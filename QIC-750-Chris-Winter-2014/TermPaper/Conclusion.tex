In conclusion, numerous researchers have published papers highlighting the feasibility of quantum state tomography using the Wigner representation of the quantum state. However, all of this being said, Wigner state representations have little experimental relevance. Most quantum computational engineers characterize their states using the density matrix representation, which is equivalent. But, considering the case of a quantum system whose states are continuous under the parameter of interest (like coherent states under the parameter $\alpha$) a density matrix representation may not be as direct as a continuous state-space representation as provided by the Wigner function.

Experimentalism aside, though, Wigner functions have been receiving a lot of attention by theoretical physicists within the last decade in that they lend themselves nicely to understanding how much ``quantum resource'' exists in a certain state \cite{Veitch}.  Coherent states of light (the appropriate description of light in the classical regime) are entirely positive in phase space. Fock states, however, must, and are always, negative over portions of the phase space. This motivates correlating the amount of ``quantum resource'' with the amount of negativity of the state's Wigner representation in phase space. So, while the Wigner representation may not be consistently practical to obtain experimentally, it is theoretically useful and is seeing much use in this domain, today (especially in the field of quantum optics).