Quantum computation relies on high-fidelity state preparation and gate operation. This is well-known as of the publishing of the oft-cited DiVincenzo criteria \cite{DIV2000}. However, characterizing the fidelity of state preparation and gate operation is non-trivial. Many metrics have been developed and applied to laboratory set-ups. Density matrix reconstruction, Rabi decay oscillations, Wigner state tomography are just a few of the most popular fidelity characterization methods (for state preparation) used in the literature today. The latter is the focus of this paper. A brief overview of the theory underlying Wigner state tomography will be given then its application to states demonstrated by various groups will be presented.

%In fact, in the current academic environment where pressure is placed on principal investigators to secure ever-scarcer funding, it is not too unreasonable to conjecture that deceiving state preparation and gate operation metrics could be utilized for the express purpose of artificially inflating one's research value. 