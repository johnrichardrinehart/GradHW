\documentclass{article}
\usepackage{dsfont,amsmath}
\begin{document}
    \section{Compression of Quantum Channels}
    The channel fidelity of a quantum channel is given by $
    F_{channel}(\Xi,\sigma) = \sqrt{\sum\limits_{k=1}^N |\langle A_k, \sigma
    \rangle |^2 }$, where $ \Xi(X) = \sum\limits_{k=1}^N A_k X A_k^* $.
    
    $ (\Phi,\Psi) $ is an $(n,\alpha,\delta)$-quantum compression scheme
    if $ F_{channel}(\Psi\Phi,\rho^{\otimes n}) > 1- \delta $.

    \begin{theorem}[Schumacher's Theorem]
        Let 
        \[
                \rho \in D(\mathds(C)^\Sigma),\quad \alpha \ge 0, \delta >
                0
        \]

        \begin{enumerate}
            \item If $ \alpha > H(\rho) $  then there exists an $
                (n,\alpha,\delta) $-compression scheme for $ \rho $  for
                all but finitely many $ n \in \mathds{N}$.
            \item If $ \alpha < H(\rho) $ then there exists an $
                (n,\alpha,\delta) $-compression scheme for $\rho$ for at
                most finitely many $ n \in \mathds{N} $.
        \end{enumerate}

        Proof sketch: Start with a spectral decomposition 
        \[ 
                \rho = \sum_{a\in \Sigma} p(a) x_a x_a^* 
        \]
        
        for some $ p \in P(\Sigma) $ and $\{x_a: a \in \Sigma \}$ an
        orthonormal basis of X.

        Remember that the typical sequences of length n are : $
        T_{n,\epsilon}(p) = \{a_1\ldots a_n \in \Sigma^n:
                2^{-n(H(\rho)+\epsilon)}<p(a_1)\ldots
        p(a_n)<2^{-n(H(p)-\epsilon)} \} $ 

        In a similar way we define a typical projector $
        \Pi_{n,\epsilon} = \sum_{a_1\ldots a_n \in T_{n,\epsilon}} x_a_1
        x_a_1^* \otimes \cdots \otimes x_a_n x_a_n^*$.

        Note that $\langle \Pi_{n,\epsilon}, \rho^{\otimes n} \rangle =
        Pr[a_1\ldots a_n \in T_{n,\epsilon} \xrightarrow{(\text{as n}\rightarrow
        \infty)} 1$, where each $ a \in \Sigma $ independently chosen
        according to p.

        Now, we define an isometry $ A = \sum\limits_{a_1 \cdots a_n \in
        T_{n,\epsilon} } e_{f(a_1\cdots a_n)} x_{a_1 \cdots a_n}^* $.
    $f: \Sigma^n \rightarrow \Sigma^m$ is an encoding function from a
    classical compression scheme.

    $ A \in L(\mathcal{X}_1 \otimes \cdots \otimes \mathcal{X}_n,
        \mathcal{Y}_1
    \otimes \cdots \otimes \mathcal{Y}_n) $.

    If we consider $ \Phi(X) = A X A^* $ we don't have a channel because
    it doesn't necessarily preserve trace. But, we can make it a channel
    by modifying it a bit $ \Phi(X) = A X A^* + \langle
    \mathds{1}-\Pi_{n,\epsilon},X\rangle\sigma $ for any $\sigma \in
    D(\mathcal{Y}_1\otimes\cdots\otimes \mathcal{Y}_n)$. If this is a
    channel then it should be the case that $Tr(\Phi(X)) = Tr(X) =
    Tr(A^*AX) + \langle \mathds{1}-\Pi_{n,\epsilon},X \rangle$. But,
    $A^*A = \Pi_{n,\epsilon}$ so $Tr(\Phi(X)) = Tr(X)$.

    We could propose some $ \Psi'(Y) = A^* Y A  $ to undo what $\Phi$
    does. However, this may not be a channel so we should add another
    correction term: $\Psi = A^* Y A + \langle \mathds{1} - A
    A^*,Y\rangle \xi$ for any state $\xi \in D(\mathcal{X}_1 \otimes
    \cdots \otimes \mathcal{X}_n)$. We should check to make sure that
    this actually undoes $\Psi$. We want to show that
    $F_{channel}(\Psi\Phi,\rho^{\otimes n})$. So let's consider what
    $(\Psi \Phi)(X)$ is.

    If I knew that $\Phi(X) = \sum_k A_k X A_k^*$ (Kraus rep) and $
    \Phi(Y) = \sum_l B_l Y B_l^* $ then 
    $ (\Psi\Phi(X) = \sum_{k,l} (B_l A_k)X(B_lA_k)^* $. 

    So, $ (\Phi\Psi)(X) = (A^*A)X(A A^*)^* + \cdots
    \text{more things like $B_k X B_k^*$} $.

    Now, \begin{align*}
        F_{channel}(\Psi\Phi, \rho^{\otimes n}) &= \sqrt{ | \langle
        A^*A,\rho^{\otimes n} \rangle |^2 + \text{non-negative terms} }
        \\
        &\ge \langle A^*A,\rho^{\otimes n} \rangle \\
        &= \langle \Pi_{n,\epsilon} ,\rho^{\otimes n}\rangle \rightarrow
1
    \end{align*}

    So, the compression is good and this works!

    \section{Quantum Entropy}
    \subsection{Properties of the Von-Neumann entropy}
        \begin{enumerate}
            \item The Shannon entropy is continuous : $H(p) = -
                \sum_{a\in \Sigma} p(a) \log(p(a))$.  The von Neumann
                entropy, $H(\rho) = H(\lambda(\rho))$, equals the entropy
                of a vector of eigenvalues. Since eigenvalues are
                continuous so the von Neumann entropy is continuous (you
                can understand this by considering constructing an
                arbitrary characteristic polynomial to obtain the
                eigenvalues. Since the von Neumann entropy is the
                composition of two continuous functions then it is
                continuous. It's easy to understand this when applied to
                hermitian operators whose eigenvalues can be ordered
                ($||\lambda(A) - \lambda(B)||_1 \le ||A-B||_1$).

            \item

        \end{enumerate}

        \begin{theorem}[Fannes-Audenaert]
            Consider two density operators $ \rho_0,\rho_1 $ and $
            \delta = .5 ||\rho_0 - \rho_1 ||_1 $ . You can conclude that
            $ |H(\rho_0)-H(\rho_1)| \le \delta \log(n-1) +
            H(\delta,1-\delta)$. This puts an upper bound on the
            difference in entropy when you perturb a density operator a
            little bit.
        \end{theorem}
        \subsection{Quantum relative entropy}
        $D(P||Q) = Tr \left(  P\log \left( P \right)-P\log \left( Q
        \right) \right) $ for $ P,Q \in Pos(\mathcal{X}) $. If $ ker(Q)
        \subseteq ker(P) $ then this is well defined and I can find
        $D(P||Q)$. However, when this is not the case we allow $ D(P||Q)
        = \infty $ .

        One way to think about this function is as one that tells you
        how much compression ``loss'' you have if you design a
        compression scheme for some $\rho$ and you used that on some
        $\sigma$.

        \begin{theorem}[Klein's inequality]
            Imagine you have $\rho,\sigma \in D(\mathcal{X})$ then
            $D(\rho||\sigma) \ge 0$ with equality if and only if $\rho =
            \sigma$.
        \end{theorem}

        Proof: Let's assume (without loss of generality) that $
        \rho,\sigma >0 $ (positive definite). Let's assume that we have
        spectral decompositions of $\rho$ and $\sigma$.  
        
        \[\rho = \sum_{a\in \Sigma} p(a) u_a u_a^*\] 
        \[\sigma = \sum_{a \in \Sigma} q(a) v_a v_a^*}\]. 
        Now,
        \begin{align*} 
            D(\rho||\sigma) &= \frac{1}{\ln(2)} \sum_{a,b} |\langle u_a,
            v_b \rangle |^2 (p(a) ln(a) - p(a) ln(b) \\
            & \intertext{It can be shown that $ln(x) < x-1$ (draw a
            2d plot} \\
            & = -p \ln(q/p) \ge -p (\frac{q}{p} -1) = p-q \\
            D(\rho||\sigma) & \ge \frac{1}{\ln(2)} \sum_{a,b} |\langle
            u_a, v_b \rangle |^2 (p(a)-q(b)) = 0
        \end{align*}

\end{document}
