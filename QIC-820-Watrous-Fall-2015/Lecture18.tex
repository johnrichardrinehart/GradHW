\documentclass{article}
\usepackage[]{amsmath,dsfont,braket,amsthm}
\newtheorem{theorem}{Theorem} 
\newtheorem{corollary}{Corollary}[theorem]
\begin{document}
\section*{LOCC and Separable Measurements}
A measurement $ \mu: \Sigma \rightarrow Pos(\mathcal{X} \otimes \mathcal{Y}) $
is LOCC if and only if the channel $\Phi \in C(\mathcal{X} \otimes \mathcal{Y},
\mathcal{Z} \otimes \mathcal{W})$ defined as:

\[ 
    \Phi(\rho) = \sum_{a\in \Sigma} \langle \mu(a) , \rho \rangle E_{a,a}
    \otimes E_{a,a}
\]

for $\mathcal{Z} = \mathds{C}^\Sigma = \mathcal{W}$, is an LOCC channel.
Thinking of a measurement as a special type of channel and requiring that to be
LOCC is another way of thinking of an LOCC measurement.

There is an analogous definition of separable measurements. That is, we can just
use the previous definition replacing the word ``LOCC'' with ``separable''. There is
a bit nicer way, though, that we can state what a separable measurement is.
We'll do that now: A measurement is known (or defined) to be separable if and
only if every measurement operator is itself separable.

Definitely, every LOCC measurement is separable. LOCC channels must be separable
channels. The converse, however, doesn't hold. There exist separable
measurements that are not LOCC. Consider $\mathcal{X} = \mathds{C}^\Sigma =
\mathcal{Y}$. Consider the following basis of $\mathcal{X} \otimes \mathcal{Y}$:
\[
    \begin{array}{ccc}
        \ket{1}\ket{1} & \ket{0}\ket{0+1} &
        \ket{2}\ket{1+2} \\
        \ket{0+1}\ket{2} & \ket{0}\ket{0-1} &
        \ket{2}\ket{1-2}  \\
        \ket{1+2}{0} & \ket{0-1}\ket{2} & \ket{1-2}\ket{0}
    \end{array}
\]
where, above, we have used the following convention:

\begin{align*}
    \ket{a+b} &= \frac{1}{\sqrt{2}} \left( \ket{a} + \ket{b} \right) \\
    \ket{a-b} &= \frac{2}{\sqrt{2}} \left( \ket{a} - \ket{b} \right)
\end{align*}

If these vectors are plotted on a coordinate system
($(\ket{0},\ket{1},\ket{2})\otimes (\ket{0},\ket{1},\ket{2})$) then these states
have a very natural interpretation.

This basis is special in that measuring with respect to this basis is a
separable measurement. However, measurement with respect to this basis is
\textbf{not} an LOCC measurement.

Imagine Bob measures first

\[ 
    \{B_a : a \in \Gamma \} \subset L(\mathcal{X}, \mathcal{Z}) 
\]

Consider the $\sum_a B_a B_a^*$.

\[ 
    \left( \mathds{1} \otimes B_a \right) \left( u_k u_k^* \otimes v_k v_k^*
    \right) \left( \mathds{1} \otimes B_a \right) ^*
\]

This can be written as:

\[ 
    \left( u_j \otimes v_j \right)^* \left( \mathds{1}\otimes B_a^* B_a \right)
    \left( u_k \otimes v_k \right) = 0 \quad,\enskip \forall a \in \Gamma
    \enskip \text{and all} \enskip j \ne k
\]

Thus, $B_a^* B_a$ is diagonal in the basis B.

In other words:

\[ 
    \mathds{1} \otimes B_a* B_a = \sum_{k=1}^a \lambda_k u_k u_k^* \otimes v_k
    v_k^* \quad,\enskip \forall a \in \Gamma 
\]

Now, if we consider some 
\begin{align*}
    \alpha &= u_1 u_1^* \otimes v_1 v_1^* \\
    \beta_1 &= u_2 u_2^* \otimes v_2 v_2^* \\
    \gamma_1 &= u_3 u_3^* \otimes v_3 v_3^*
\end{align*}

If we consider what $\mathds{1} \otimes B_a^* B_a$ must be represented by (in
terms of a matrix) you can put constraints on $\gamma$s and $\beta$s.

For example, we know that we have $\mathds{1}\otimes B_a^* B_a$. Identity tensored with
something is going to be block diagonal. So, the off-block-diagonal elements
must be zero.

We can actually show that $\mathds{1}\otimes B_a^* B_a = \delta
\mathds{1}\otimes \mathds{1}$. This means that Bob can not get information about
Alice's state and, by symmetry, Alice can't get information about Bob's state.
Thus, this set is not LOCC. There are, now, measurements that are separable but
not LOCC. There are also channels that are separable but not LOCC.

If it's LOCC then it's not just both parties acting in unison but each party
taking turns. That puts a constraint on the information and on how much of that
information can be gained about the state of the other. By trying to determine
the other state they will introduce a disturbance in the state which will
possibly cause confusion in the state. If they can somehow avoid this by
choosing particular states that won't cause confusion, then you can show that
the measurement of these states is not LOCC.

\section*{Maximally Entangled States}

Consider $\mathcal{X}$ and $ \mathcal{Y} $ be complex Euclidean spaces, both
having dimension n. What is a maximally entangled state? Well, we mean that it's
a pure state such that if I throw away some of the state I can obtain identity
in the reduced space. That is:

\[ 
    u = \frac{1}{\sqrt{n}} vec(u) \quad,\enskip \text{for} \enskip u \in
    U(\mathcal{Y},\mathcal{X})
\]

Suppose we have k maximally entangled states:

\begin{align*}
    u_1=& \frac{1}{\sqrt{n}} vec(u_1) \\ 
    \vdots& \\
    u_k =& \frac{1}{\sqrt{n}} vec(u_k)
\end{align*}

for $u_1,\ldots,u_k \in U(\mathcal{Y}, \mathcal{X})$. We have the following
proposition:

If $\mu:\{1,\ldots,k\} \rightarrow Pos(\mathcal{X}\otimes \mathcal{Y})$ is a
\textbf{separable} measurement, then:

\[ 
    \frac{1}{k} \sum^{k}_{j=1}\langle \mu(j) , u_j u_j^* \rangle  \le \frac{n}{k}
\]

\begin{proof}
    Assume that $\mu(j) = \sum_l P_l^j \otimes Q_l^j \quad \text{for each}
    \enskip j$

    Consider the probability that we get the jth guy as an output with
    probability j:

    \begin{align*}
        \langle \mu(j) , u_j u_j^* \rangle =& \frac{1}{n} vec(U_j)^* \left(
    \sum_l P_l^j \otimes Q_l^j \right) vec(U_j) \\
    =& \frac{1}{n} \sum_l Tr \left( U_j^* P_l^j U_j \left( Q_l^j \right)^T
\right) \\
\intertext{Note that $U_j^* P_l^j U_j$ is positive semidefinite so we can use
    $Tr(PQ) \le Tr(P)Tr(Q)$. This is a crude, heavy-handed inequality \dots But it's
    all we need so we'll use it. You can show this by considering that $\left|
        \langle P, Q \rangle  \right| \le \left| \left| P \right| \right| \left| \left|
        Q\right| \right|_1 \le \left| \left| P \right| \right|_1 \left| \left| Q \right|
\right|_1 = Tr(P)Tr(Q)$}
\le& \sum_l \frac{1}{n} Tr(P_l^j) Tr(Q_l^j) \\
=& \frac{1}{n} \sum_l Tr(P_l^j) Tr(Q_l^j) \\
=& \frac{1}{n} Tr \left( \sum_l P_l^j \otimes Q_l^j \right) \\
=& \frac{1}{n} Tr(\mu(j))
\end{align*}

Now, we can write:

\begin{align*}
    \frac{1}{k} \sum_{j=1}^k \langle \mu(j) , \mu_j \mu_j^* \rangle \le &
    \frac{1}{k} \sum_{j=1}^k \frac{1}{n} Tr( \mu(j)) \\
    =& \frac{1}{kn} Tr( \sum_{j=1}^k \mu(j)) \\
    =& \frac{1}{kn} Tr( \mathds{1} \otimes \mathds{1}) \\
    =& \frac{n}{k}
\end{align*}
\end{proof}

We'll now show that if you want to distinguish two pure states you can always do
this with an LOCC measurement. Suppose, then, that $x_0, x_1 \in \mathcal{X}
\otimes \mathcal{Y}$ are orthogonal unit vectors. We will show that there exists
an LOCC measurement ($\mu:\{0,1\} \rightarrow Pos(\mathcal{X}\otimes
\mathcal{Y})$) such that:

\[ 
    \langle \mu(0) , x_0 x_0^* \rangle = 1
\]

and that

\[ 
    \langle \mu(1) , x_1 x_1^* \rangle = 1
\]

That is, the two states can be distinguished perfectly with these two
measurements.

To prove this, consider we have some operator $A \in L(\mathcal{X})$ which is
not necessarily normal. We consider a set called the ``numerical range'' of A
is:

\[ 
    \mathcal{N}(A) = \{u^*Au : u \in S(\mathcal{X}) \}
\]

\[ 
    S(\mathcal{X}) = \{u \in \mathcal{X}: \left| \left| \mu \right| \right| = 1
    \}
\]

If A is indeed normal then $\mathcal{N}$ is the convex hull of the eigenvalues.
If A is not normal then you can have more interesting numerical ranges. For
example, let $A = \begin{pmatrix} 1 & 1 \\ 0 & 1 \end{pmatrix}$. The numerical
range of A can be drawn on the real axis as a circle centered at 1 with radius
1/2. We have a famous theorem that tells us something about the numerical range:

\begin{theorem}{Toeplitz-Hausdorff Theorem}
    For every $A \in L(\mathcal{X})$, the numerical range of $\mathcal{N}(A)$ is
    \underline{convex and compact}
\end{theorem}
\begin{corollary}
    Suppose $A\in L(\mathcal{X})$ and $Tr(A) = 0$. There exists an orthonormal
    basis

    \[ 
        \{u_1,\ldots,u_n\} \subset X 
    \]

    of $\mathcal{X}$ such that

    \[ 
        u_k^* A u_k = 0 \quad,\enskip \forall k=1,\ldots,n 
    \]

\end{corollary}
We will now use these two facts to show how Alice and Bob can distinguish
between any two pure states. We will now proof that there exists a pair of LOCC
measurements that I can use to distinguish any two orthogonal states.

\begin{proof}
    Let $X_0, X_1 \in L(\mathcal{Y},\mathcal{X})$ such that $x_0 = vec(X_0)$ and
    $x_1 = vec(X_1)$. Orthogonality of $x_0,x_1$ implies that $Tr(X_1^* X_0) =
    0$. But, $X_1^* X_0 \in L(\mathcal{Y})$. So, there exists an orthonormal
    basis $\{u_1,\ldots,u_n\} \subset \mathcal{Y}$ such that:

        \[ 
            u_k^* X_1^* X_0 u_k = 0  \quad,\enskip \text{for} \enskip
            k=1,\ldots,n
        \]
        
        Now, Bob measures with respect to $\{u_1,\ldots,u_n \}$ and tells Alice
        the result. Conditioned on outcome $k$, Alice holds either

        \begin{align*}
            \left( \mathds{1} \otimes \overline{u_k} \right)^* x_0 \left(
            \mathds{1} \otimes \overline{u_k} \right) =& \left( \mathds{1}
            \otimes u_k^* \right) vec(X_0) vec(X_0)^* \left( \mathds{1} \otimes
        \overline{u_k} \right) \\
        =& vec(X_0 u_k) vec(X_0 u_k)^* \\
        =& X_0 u_k u_k^* X_0^* \\
        \enskip \text{or} \enskip & X_1 u_k u_k^* X_1^* 
        \end{align*}
       
        These states are orthogonal, so Alice can distinguish them perfectly!
\end{proof}
\end{document}
