\documentclass{article}
\usepackage[]{amsmath,dsfont,amssymb}
\begin{document}
Last time we considered the concept of separability. We did this for states
and now we'd like to do something similar for channels.

Imagine an abstract scenario where Alice has access to a register X that she
uses to configure a register Z and Bob has a register Y that he uses to
configure register W. Then, imagine that Alice and Bob can communicate
classically. We call an operation that would work in this way an ``LOCC''
paradigm (``LOCC`` = Local Operations with Classical Communication). It's
also called the ``distant labs'' paradigm.

We'll have to work our way up to LOCC. Although it addresses a lot of
interesting scenarios, it's really complicated to analyze. One reason for
this is that there is no bound on the amount of communication that Alice and
Bob can have. So, we'll establish a simpler notion: A separable channel (and
maps, more generally).

\section*{Separability Extended to Maps}
A completely positive map:
\[ 
        \Xi \in CP(\mathcal{X}\otimes \mathcal{Y},\mathcal{Z} \otimes
        \mathcal{W} )
\]

is separable if it can be written:

\[ 
        \Xi = \sum_{k=1}^m \Phi_k \otimes \Psi_k 
\]

are completely positive maps $\Phi_1, \ldots, \Phi_m \in
CP(\mathcal{X},\mathcal{Z})$ and $\Psi_1, \ldots, \Psi_m \in
CP(\mathcal{Y},\mathcal{W})$.

Notation: $SepCP(\mathcal{X},\mathcal{Z}:\mathcal{Y},\mathcal{W})$.

So we will denote separable channels as
$SepC(\mathcal{X},\mathcal{Z}:\mathcal{Y},\mathcal{W}) \equiv
SepCP(\mathcal{X},\mathcal{Z}:\mathcal{Y},\mathcal{W}) \cap
C(\mathcal{X}\otimes\mathcal{Y},\mathcal{Z}\otimes \mathcal{W})$.

Simple observations:

\begin{enumerate}
    \item If $\Xi \in Sep(\mathcal{X},\mathcal{Z}:\mathcal{Y},\mathcal{W})$
        then $\Xi$ has a Kraus representation given by
        \[ 
                \Xi(X) = \sum_{k=1}^m (A_k \otimes B_k) X (A_k \otimes
                B_k)^* 
        \]

        for some choice of 
        \begin{align*}
            &A_1, \ldots, A_m \in L(\mathcal{X},\mathcal{Z}) \\
            &B_1, \ldots, B_m \in L(\mathcal{Y},\mathcal{W})
        \end{align*}
    \item \textbf{Not} every separable channel $\Xi$ looks like:
        \[ 
                \Xi = \sum_{k=1}^m p_k \Phi_k \otimes \Psi_k 
        \]

        for $(p_1,\ldots,p_m)$ a probability vector and
        \begin{align*}
            \Phi_1, \ldots, \Phi_m \in C(\mathcal{X},\mathcal{Z}) \\
            \Psi_1, \ldots, \Psi_m \in C(\mathcal{Y},\mathcal{W})
        \end{align*}

        This would be too strong of a constraint on $\Xi$. 
        The form of $\Xi$ above represents local operations with shared
        randomness (LOSR channels).
    \item   $SepC(\mathcal{X},\mathcal{Z}:\mathcal{Y},\mathcal{W})$ is a convex
        and compact set.
    \item   $SepCP(\mathcal{X},\mathcal{Z}:\mathcal{Y},\mathcal{W})$ is closed
        and convex.
\end{enumerate}

Let's define an isometry 

\[ 
V = U(\mathcal{Z}\otimes \mathcal{W} \otimes
    \mathcal{X} \otimes \mathcal{Y}, \mathcal{Z} \otimes \mathcal{X} \otimes
\mathcal{W} \otimes \mathcal{Y})
\]

as

\[ 
        V(\mathcal{z}\otimes \mathcal{w} \otimes \mathcal{x} \otimes
        \mathcal{y}) = \mathcal{z} \otimes \mathcal{x} \otimes \mathcal{w}
        \otimes \mathcal{y} \quad \forall x \in \mathcal{X}, y \in \mathcal{Y},
        z \in \mathcal{Z}, w \in \mathcal{W}
\]

Note:

\[ 
        V vec(A \otimes B) = vec(A) \otimes vec(B) \quad \forall A \in
        L(\mathcal{X},\mathcal{Z}), B \in L(\mathcal{Y},\mathcal{W})
\]

Now consider the following proposition:

\[ 
        \Xi \in CP(\mathcal{X}\otimes \mathcal{Y}, \mathcal{Z}\otimes
        \mathcal{W})
\]

it holds that 

\[ 
        \Xi \in SepCP(\mathcal{X},\mathcal{Z}:\mathcal{Y},\mathcal{W}) 
\]

if and only if

\[
        V J(\Xi) V^* \in Sep(\mathcal{Z}\otimes \mathcal{X}: \mathcal{W}\otimes
\mathcal{Y})
\]

Proof sketch: The following are equivalent:

\begin{enumerate}
    \item The channel $\Xi$ is  separable ($\Xi \in SepCP$)
    \item The Kraus representation of $\Xi$ has the special form:

        \[ 
                \Xi(X) = \sum_{k=1}^m \left( A_k \otimes B_k \right) X \left(
                A_k \otimes B_k \right)^* 
        \]
        
        for some choice of $A_1,\ldots,A_m, B_1,\ldots, B_m$.

    \item $J(\Xi) = \sum_{k=1}^m vec(A_k \otimes B_k)vec(A_k \otimes B_k)^*$
    \item $V J(\Xi) V^* = \sum_{k=1}^m vec(A_k)vec(A_k)^* \otimes
        vec(B_k)vec(B_k)^*$
    \item $V J(\Xi) V^* \in Sep(\mathcal{Z}\otimes \mathcal{X}:
        \mathcal{W}\otimes \mathcal{Y})$
\end{enumerate}

We can draw a simple corollary from this. Imagine we had the following mapping:

\[ 
        \Omega \in C(\mathcal{X}\otimes \mathcal{Y}, \mathcal{Z} \otimes
        \mathcal{W})
\]

\[ 
        \Omega(X) = Tr(X) \frac{\mathds{1}_{\mathcal{Z}}}{n} \otimes
            \frac{\mathds{1}_{\mathcal{W}}}{m}
\]

for $n = dim(\mathcal{Z})$ and $m=dim(\mathcal{W})$. All channels ``sufficiently
close'' to $\Omega$ must be separable.

Imagine there was no entanglement between registers X and Y and we want to do
some complicated protocol and generate entanglement. This can not happen with
only the exchange of classical information! We can not create entanglement
``over the telephone''. To prove this we will need to introduce some measure of
entanglement that we can use to quantify how much entanglement exists between
states.

\section*{Entanglement Rank (a.k.a. Schmidt Number)}

Suppose that 
\[ 
        P \in Sep(\mathcal{X}:\mathcal{Y}) 
\]

We can write

\[ 
        P = \sum_{k=1}^m x_k x_k^* \otimes y_k y_k^* 
\]

for normalization of $x_1,\ldots,x_m \in \mathcal{X}$ and $y_1,\ldots,y_m \in
\mathcal{Y}$ being absorbed into the vectors (not necessarily normalized). 

We would like to think about P as having ``entanglement rank equal to 1''. More
specifically, this terminology makes sense if we define:

\[ 
        X_k = x_k y_k^T \quad \text{(for each k)}
\]

and we observe the following:

\[ 
        P = \sum_{k=1}^m vec(X_k)vec(X_k)^* 
\]

and we know that the rank of these vectors is at most 1: $rank(X_k) \le 1$ for
each k.

More generally, $P \in Pos(\mathcal{X}\otimes \mathcal{Y})$ has entanglement
with rank at most r if we can write:

\[ 
        P = \sum_{k=1}^m vec(X_k)vec(X_k)^* 
\]

for $ X_1,ldots,X_m \in L(\mathcal{Y},\mathcal{X}) $ and $rank(X_k) \le r \quad
\text(for all k)$. There are a lot of ways you could construct the $X_k$s so
that's why we say ``at most rank of whatever''. We can define the entanglement
rank as the minimum r for which this is possible.

Also, we'll define:

\[ 
        Ent_r(\mathcal{X}:\mathcal{Y}) \subseteq Pos(\mathcal{X}\otimes
        \mathcal{Y})
\]

to be the set of all P having entanglement rank at most r. We'll define this to
be the set having entanglement rank at most r because we want the set to be
convex. Note that the set is a cone because we can scale P by any positive real
number and it doesn't change the rank. To be convex we can show it's a cone and
that it's closed under addition. We know it's a cone because when we add two
guys who's rank is at most r and we add another guy whose rank is at most r then
we get a result whose rank is at most r.

We'll introduce the following fact:

\[ 
        Sep(\mathcal{X}:\mathcal{Y}) \subsetneq Ent_2(\mathcal{X}:\mathcal{Y})
        \subsetneq \cdots \subsetneq Ent_n (\mathcal{X}:\mathcal{Y}) = Pos
        \left( \mathcal{X} \otimes \mathcal{Y} \right)
\]

where $n = min\{dim(\mathcal{X}), dim \left( \mathcal{Y} \right) \}$

Theorem: If $P \in Ent_r \left( \mathcal{X}:\mathcal{Y} \right)$  and $\Xi \in
SepCP \left( \mathcal{X},\mathcal{Z}:\mathcal{Y},\mathcal{W} \right)$, then
$\Xi(P) \in Ent_r \left( \mathcal{Z}:\mathcal{W} \right)$.

Proof:
\begin{align*}
    A &\in L(\mathcal{X},\mathcal{Z}) \\
    B &\in L \left( \mathcal{Y},\mathcal{W} \right) \rightarrow \left( A \otimes
B\right)vec(X) = vec(AXB^T) \\
X &\in L(\mathcal{Y},\mathcal{X})
\end{align*}

where $rank(AXB^T) \le rank(X)$

So, if 
\[ 
        \Phi(P) = \sum_k \left( A_k \otimes B_k \right) P \left( A_k \otimes B_k
        \right)^*
\]

and 

\[ 
        P = \sum_j vec(X_j) vec \left( X_j \right)^* 
\]

then

\[ 
        \Phi(P) = \sum_{j,k} vec(A_k X_j B_k^T) vec \left( A_k X_j B_k^T
        \right)^*
\]

If $rank(x_j) \le r$ for all $j$ then $rank \left( A_k x_j B_k^T \le r \right)
\quad \forall j,k$. Thus:

\[ 
        \Phi(P) \in Ent_r \left( \mathcal{Z}:\mathcal{W} \right) 
\]

Corollary:
\begin{align*}
    P &\in Sep(\mathcal{X}:\mathcal{Y}) \\
    \Phi &\in SepCP \left( \mathcal{X}:\mathcal{Z}:\mathcal{Y},\mathcal{W} \right)
\quad \Rightarrow \quad Phi(P) \in Sep(\mathcal{Z}:\mathcal{W})
\end{align*}

Proposition:
\begin{align*}
    \Phi \in SepCP \left( \mathcal{X}, \mathcal{U}:\mathcal{Y},\mathcal{V}
    \right) \\
    \Psi \in SepCP \left( \mathcal{U},\mathcal{Z}:\mathcal{V},\mathcal{W}
    \right) \\
    \Rightarrow \Psi\Phi \in
    SepCP(\mathcal{X},\mathcal{Z}:\mathcal{Y},\mathcal{W})
\end{align*}

Proof: Easy.

\section*{Definition of an LOCC Map}

Consider some channel $\Xi \in C(\mathcal{X} \otimes \mathcal{Y} , \mathcal{Z}
\otimes \mathcal{W})$.

\begin{enumerate}
    \item $\Xi$ is a one-way right (from Alice to Bob) LOCC channel if it's
        possible to write:

        \[ 
                \Xi = \sum_{a \in \Sigma} \Phi_a \otimes \Psi_a 
        \]
        
        for an alphabet $\Sigma$, $\{\Phi_a, a \in \Sigma\} \subset
        CP(\mathcal{X},\mathcal{Z})$ satisfying

        \[ 
                \sum_a \Phi_a \in C(\mathcal{X},\mathcal{Z}) 
        \]
        
        (i.e., $\{ \Phi_a : a \in \Sigma \}$ is an ``instrument'') and 

        \[ 
                \{ \Psi_a : a \in \Sigma \} \subseteq
                C(\mathcal{Y},\mathcal{W}) 
        \]
        
    \item $\Xi$ is a one-way left (from Bob to Alice) LOCC channel if it's
        possible to write:

        \[ 
                \Xi = \sum_{a \in \Sigma} \Phi_a \otimes \Psi_a 
        \]
        
        for an alphabet $\Sigma$, $\{\Psi_a, a \in \Sigma\} \subset
        CP(\mathcal{Y},\mathcal{W})$ satisfying

        \[ 
                \sum_a \Psi_a \in C(\mathcal{Y},\mathcal{W}) 
        \]
        
        (i.e., $\{ \Psi_a : a \in \Sigma \}$ is an ``instrument'') and 

        \[ 
                \{ \Phi_a : a \in \Sigma \} \subseteq
                C(\mathcal{X},\mathcal{Z}) 
        \]

    \item $\Xi$ is an LOCC channel:

        \[ 
                \Xi \in LOCC(\mathcal{X},\mathcal{Z}:\mathcal{Y},\mathcal{W}) 
        \]
        
        if 
        
        \[ 
            \Xi = \Xi_n \cdots \Xi 
        \]
        
        for one-way LOCC channels $\Xi_1, \ldots, \Xi_n$ (any $n \in
        \mathds{N}$)

\end{enumerate}

\end{document}
