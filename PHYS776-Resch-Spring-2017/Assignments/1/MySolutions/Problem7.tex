% Problem 1.7
\begin{homeworkProblem}[Problem 7]

\textbf{Problem Description.}

\begin{homeworkSection}{a)}
   \newcommand\Aop{\ensuremath{a\otimes\tilde{a}}}
   \newcommand\Adaggerop{\ensuremath{a^{\dagger}\otimes\tilde{a}^{\dagger}}}
   \newcommand\Bop{\ensuremath{a^{\dagger}a \otimes \mathds{1} + \mathds{1}
   \otimes \tilde{a}^{\dagger}\tilde{a} + \mathds{1} \otimes \mathds{1}}}
   \begin{align}
      [A,B] &= [\Aop, \Bop] \\
            &= [a \otimes \tilde{a}, a^{\dagger}a \otimes \mathds{1}] +
      [a \otimes \tilde{a}, \mathds{1} \otimes \tilde{a}^{\dagger} \tilde{a}]
      \\
      &= [a, a^{\dagger} a] \otimes \tilde{a} + a \otimes [\tilde{a},
      \tilde{a}^{\dagger} \tilde{a}]
      \intertext{Now, $ [a,a^{\dagger}a] = [a,a^{\dagger}]a + a^{\dagger}[a,a] =
      a$.}
      &= a \otimes \tilde{a} + a \otimes \tilde{a} \\
      &= 2 a \otimes \tilde{a} \\
      &= 2A \\ \nonumber \\
      [A^{\dagger}, B] &= A^{\dagger}B - B A^{\dagger} \\
                       &= \left( B^{\dagger}A - A B^{\dagger} \right)^{\dagger}
      \intertext{But, $ B^{\dagger} = B $, by inspection.}
      &= (B A - A B)^{\dagger} \\
      &= (- [A,B])^{\dagger} \\
      &= -2 a^{\dagger} \otimes \tilde{a}^{\dagger} \\
      &= -2 A^{\dagger}
   \end{align}

   \begin{align}
      [A, A^{\dagger}]
      &= [\Aop,\Adaggerop] \\
      &= a a^{\dagger} \otimes \tilde{a}\tilde{a}^{\dagger} -
      a^{\dagger}a \otimes \tilde{a}^{\dagger} \tilde{a} \\
      &= a a^{\dagger} \otimes \tilde{a}\tilde{a}^{\dagger} -
      a^{\dagger}a \otimes \tilde{a}^{\dagger} \tilde{a} +
      \underbrace{a^{\dagger}a \otimes \tilde{a}\tilde{a}^{\dagger}
      - a^{\dagger}a \otimes \tilde{a}\tilde{a}^{\dagger}}_0 \\
      &= [ a, a^{\dagger}] \otimes \tilde{a}\tilde{a}^{\dagger} +
      a^{\dagger}a \otimes \left[ \tilde{a}, \tilde{a}^{\dagger} \right] \\
      &= \mathds{1} \otimes \tilde{a}\tilde{a}^{\dagger} +
      a^{\dagger}a \otimes \mathds{1} \\
      &= \mathds{1} \otimes \left( \mathds{1}+\tilde{a}^{\dagger} \tilde{a} \right) +
      a^{\dagger}a \otimes \mathds{1} \\
      &= B
   \end{align}
\end{homeworkSection}

\begin{homeworkSection}{b)}
   \newcommand\Aop{\ensuremath{-\sigma_{-}}}
   \newcommand\Adaggerop{\ensuremath{\sigma_{+}}}
   \newcommand\Bop{\ensuremath{\sigma_{z}}}
   \begin{align}
      [A, B] &= [\Aop, \Bop] \\
             &= \frac{1}{2} [- \left( \sigma_{x} + i \sigma_{y} \right), \sigma_{z}] \\
             &= \frac{1}{2}[\sigma_{z}, \sigma_{x}] - \frac{1}{2}i [\sigma_{y}, \sigma_{z}] \\
             &= i \sigma_{y} + \sigma_{x} \\
             &= 2 \sigma_{+} \\
             &= 2 A
   \end{align}
   \begin{align}
      [A^{\dagger}, B] &= [\Adaggerop, \Bop] \\
             &= \frac{1}{2} [\left( \sigma_{x} + i \sigma_{y} \right), \sigma_{z}] \\
             &= \frac{1}{2}[\sigma_{x}, \sigma_{z}] + \frac{1}{2}i [\sigma_{y}, \sigma_{z}] \\
             &= -i \sigma_{y} - \sigma_{x} \\
             &= -2 \sigma_{+} \\
             &= -2 A^{\dagger}
   \end{align}
   \begin{align}
      [A, A^{\dagger}] &= [\Aop, \Adaggerop] \\
                       &= -\frac{1}{4} [\sigma_{x} - i\sigma_{y}, \sigma_{x} + i
      \sigma_{y}] \\
      &= -\frac{1}{4} \left( i[\sigma_{x}, \sigma_{y}] -
      i[\sigma_{y}, \sigma_{x} ] \right) \\
      &= \sigma_{z} \\
      &= B
   \end{align}
\end{homeworkSection}
\begin{homeworkSection}{c)}
   \begin{align}
      e^{\theta \left( \sigma_{+} + \sigma_{-} \right)}
      = e^{\theta \sigma_{x}}
      &= 1 + \theta \sigma_{x} + \frac{\theta^{2}}{2!} + \frac{\theta^{3}}{3!}
      \sigma_{x} + \frac{\theta^4}{4!} + \ldots \\
      &= \cosh(\theta) + \sinh(\theta) \sigma_{x} \\ \nonumber \\
      e^{\sigma_{+}\tanh(\theta)}e^{\sigma_{z}\ln \sech(\theta)}
      e^{\sigma_{-}\tanh(\theta)}
      &= \left( 1 + a \sigma_{+} + \frac{a^{2}\sigma_{+}^{2}}{2!} \ldots \right)
      e^{\sigma_{z} \ln \sech(\theta)} \left( 1 + a \sigma_{-} +
      \frac{a^{2}\sigma_{-}^{2}}{2!} + \ldots \right)
      \label{eq:7crightside}
      \intertext{\quad where, above, $ a = \tanh(\theta) $.}
      \sigma_{+}^{2} = \sigma_{+} \sigma_{+} &\propto \left( \sigma_{x} + i \sigma_{y}
      \right)\left( \sigma_{x} + i \sigma_{y} \right) = \mathds{1} + i
      \sigma_{x} \sigma_{y} + i \sigma_{y} \sigma_{x} - \mathds{1} = 0
      \intertext{Since $ \sigma_{-} $ differs only in the sign of $ \sigma_{y} $
         then $ \sigma_{-}^{2} = 0$, also. Continuing from
      Eq.~\ref{eq:7crightside}:}
      &= \left( 1 + a \sigma_{+} \right) e^{\sigma_{z} \ln \sech(\theta)}
      \left( 1 + a \sigma_{-} \right) \\
      &=
      \begin{pmatrix}
         1 & a \\
         0 & 1
      \end{pmatrix}
      \begin{pmatrix}
         e^{b} & 0 \\
         0 & e^{-b}
      \end{pmatrix}
      \begin{pmatrix}
         1 & 0 \\
         a & 1
      \end{pmatrix}
      \intertext{where $ b = \ln \sech(\theta) $.}
      &= \begin{pmatrix}
      e^{-b}a^{2} + e^{b} & a e^{-b} \\
      a e^{-b} & e^{-b}
      \end{pmatrix} \\
      &= a e^{-b} \sigma_{x} +
      \begin{pmatrix}
      e^{-b}a^{2} + e^{b} & 0 \\
      0 & e^{-b}
      \end{pmatrix} \\
      &= \tanh(\theta) \cosh(\theta) \sigma_{x} +
      \begin{pmatrix}
         \cosh(\theta) \tanh^{2}(\theta) + \sech(\theta) & 0 \\
         0 & \cosh(\theta)
      \end{pmatrix} \\
      &= \sinh(\theta) \sigma_{x} +
      \begin{pmatrix}
         \tanh(\theta) \sinh(\theta) & 0 \\
         0 & \cosh(\theta)
      \end{pmatrix} \\
      &= \sinh(\theta) \sigma_{x} + \cosh(\theta) \mathds{1}
\end{align}
\end{homeworkSection}
\begin{homeworkSection}
   Both of these expressions only involve products of exponentiated operators.
   Expressions only involving products of exponentiated operators can be
   rewritten using the Baker-Campbell-Hausdorff formula. This formula, however,
   only depends on the commutation relations between the exponentiated
   operators. Since the commutation relations between these two sets of
   operators are identical, then these two expressions can be rewritten in the
   same way. By substituting the operators $ A, A^{\dagger}, B $ (written in terms of the
   creation and annihilation operators) into the right-hand side of this
   expression we recover the normal ordered form of the thermal operator. Thus,
   the unordered and the normal ordered form of the thermal operator are
   equivalent to one another.
\end{homeworkSection}
\end{homeworkProblem}
