% Problem 1.5
\begin{homeworkProblem}[Problem 6]
   \begin{homeworkSection}{}
      If the state has \SI{10}{\deci\bel} of squeezing then that means the ratio
      of the variance in the $ q $ quadrant of the squeezed state to that of the
      vacuum is
      \[
         10 = -10 \log_{10}\left(\frac{\Delta q^{2}}{\Delta q_{0}^{2}}\right) \enskip.
      \]
      \begin{align}
         -1 &= \log_{10} \left( \frac{\Delta q^{2}}{\Delta {q_0}^{2}}
         \right) \\
         10^{-1} &= \frac{\Delta q^2}{\Delta q_{0}^{2}}
         \intertext{Now, from the notes, $ \braket{q} = 0 $ and $ \braket{q^2} =
            \frac{1}{2}\left( \cosh(2r) - \sinh(2r) \cos(\theta) \right)$ for
         squeezed states (where the squeezing parameter $ \xi = r e^{i \theta}
      $). The vacuum state can be considered a squeezed state with squeezing
   parameter $ \xi = 0 $.}
   &= \frac{\Delta q^2}{\frac{1}{2}} \\
   &= \cosh(2r) - \sinh(2r) \cos(\theta)
   \end{align}
   Now, this expression depends on both $ r $ and $ \theta $. That is, it
   depends on $ \xi $, not just its magnitude or phase. We want to consider how
   many photons are in this state. That is, we want to determine
   \[
    \braket{n} = \braket{0 | S^{\dagger}(\xi) a^{\dagger} a S(\xi) | 0}
    \enskip.
   \]
   In order to maximize the squeezing in the $ q $ quadrant $ \theta = 0 $.
   Assuming that the experimentalists wanted to determine just how much
   squeezing they could get, they would, according to this definition, squeeze
   in the $ q $ quadrant. Thus, we'll assume $ \xi $ is real.
   \[
      10^{-1} = \cosh(2r) - \sinh(2r)
   \]
   The value of $ r $ that satisfies this is given by
   \[
      e^{-2r} = 10^{-1} \enskip.
   \]
   \[
      r = -\frac{1}{2} \ln(10^{-1}) = \frac{1}{2} \ln(10)
   \]
   Using this to calculate the number of photons in the squeezed state:
   \begin{align}
      \braket{n} &= \braket{0 | S^{\dagger}(\xi) a^{\dagger} a S(\xi) | 0} \\
                 &= \braket{0 | S^{\dagger}(\xi) a^{\dagger} S
   S^{\dagger}(\xi) a S | 0}
   \intertext{Using expressions derived in the notes (3.121 and 3.121):}
   &= \braket{0 | \left( a^{\dagger}\cosh(r) - a e^{-i \theta} \sinh(r) \right)
\left( a \cosh(r) - a^{\dagger}e^{i \theta} \sinh(r) \right) | 0}
\intertext{Only the one term that acts with an annihilation operator before the
creation operator survives.}
&= \braket{0 | a a^{\dagger} \sinh^2(r) | 0} \\
&= \sinh^2(r)
   \end{align}
   So, the number of photons in the squeezed state is $ \sinh^2(r) = \sinh^2(.5
   \ln(10) \approx 2.025 $. Haha, no, I would not describe this as bright.
   \end{homeworkSection}
\end{homeworkProblem}
