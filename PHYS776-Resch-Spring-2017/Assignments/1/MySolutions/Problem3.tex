% Problem 1.3
\begin{homeworkProblem}

\textbf{Problem Description.}

\begin{homeworkSection}{}
   $ q = \frac{a^{\dagger} + a}{\sqrt{2}} $ and $ p = \frac{i(a^{\dagger} -
   a)}{\sqrt{2}} $. $ \Delta q = \sqrt{\Braket{q^2}-\Braket{q}^{2}} $. $ \Delta
   p = \sqrt{\Braket{p^2}-\Braket{p}^{2}} $.

   \begin{align}
      \Braket{q}_\theta &= \Braket{0,\tilde{0}|T^{\dagger} q T | 0, \tilde{0}}
      \\
      &\propto \Braket{0,\tilde{0}|T^{\dagger} (a + a^{\dagger}) T | 0, \tilde{0}}
      \intertext{Where the factor of $ \sqrt{2} $ has been omitted for brevity.}
      &=
      \Braket{0,\tilde{0}|T^{\dagger} a T | 0, \tilde{0}} +
      \Braket{0,\tilde{0}|T^{\dagger} a^{\dagger} T | 0, \tilde{0}}
   \end{align}
Note that the operator in the second term is the adjoint of the first operator
and, so, the second expectation value is the complex conjugate of the first.
Thus, we only consider the first expectation value.
\begin{align}
   \Braket{0,\tilde{0}|T^{\dagger} a T | 0, \tilde{0}} &= \Braket{0,\tilde{0} |
   \cosh(\theta)a + \sinh(\theta)a^{\dagger} | 0, \tilde{0}} = 0
\end{align}
Thus, $ \Braket{q} = \Braket{p} = 0 $.

So, $ \Delta q = \sqrt{\Braket{q^{2}}} $ and similarly for $ p $.

\newcommand\TdaggeraT{\ensuremath{\cosh(\theta) a + \sinh(\theta)
\tilde{a}^{\dagger}}}
\newcommand\TdaggeradaggerT{\ensuremath{\cosh(\theta)a^{\dagger} + \sinh(\theta)\tilde{a}}}

\begin{align}
   \Braket{q^{2}}_{\theta} &\propto \Braket{0,\tilde{0}|T^{\dagger} q^{2} T | 0,
   \tilde{0}} \\
   &= \Braket{0,\tilde{0}|T^{\dagger} \left( \left( a^{\dagger} \right)^{2} + a^{2} +
a^{\dagger}a + a a^{\dagger} \right) T | 0, \tilde{0}} \\
&= \Braket{0,\tilde{0}|T^{\dagger} \left( a^{\dagger} \right)^{2} T | 0, \tilde{0}} +
\Braket{0,\tilde{0}|T^{\dagger} a^{2} T | 0, \tilde{0}} +
\Braket{0,\tilde{0}|T^{\dagger} a^{\dagger} a T | 0, \tilde{0}} +
\Braket{0,\tilde{0}|T^{\dagger} a a^{\dagger} T | 0, \tilde{0}}
\intertext{Inserting identity in the form of $ T T^{\dagger} $ in between all of
the pairs of creation/annihilation operators yields:}
&= \Braket{0,\tilde{0}| T^{\dagger} a^{\dagger} T T^{\dagger} a^{\dagger} T | 0, \tilde{0}} +
\Braket{0,\tilde{0}|T^{\dagger} a T T^{\dagger} a T | 0, \tilde{0}} + \nonumber
\\
&\quad \Braket{0,\tilde{0}|T^{\dagger} a^{\dagger} T T^{\dagger} a T | 0, \tilde{0}} +
\Braket{0,\tilde{0}|T^{\dagger} a T T^{\dagger} a^{\dagger} T | 0, \tilde{0}} \\
\intertext{Using that $ T^{\dagger}(\theta) a T(\theta) = \cosh(\theta) a +
   \sinh(\theta) \tilde{a}^{\dagger} $ and $ T^{\dagger}(\theta) a^{\dagger} T(\theta) =
\cosh(\theta)a^{\dagger} + \sinh(\theta)\tilde{a}$ it's clear that the only
terms which can survive are those which are balanced with the same number of
creation and annihilation operators (the last two). Expanding these:}
&=\Braket{0,\tilde{0}|\left( \TdaggeradaggerT \right) \left( \TdaggeraT \right)
| 0, \tilde{0}} + \\
&\quad\Braket{0,\tilde{0}| \left( \TdaggeraT \right) \left( \TdaggeradaggerT \right) | 0, \tilde{0}}
\intertext{Only the terms which involve the same number of raising and lowering
operators survive. Furthermore, since the operators act on the ground state,
only those terms which involve a creation operator acting prior to an
annihilation operator survive.}
&=\Braket{0,\tilde{0}| \sinh^{2}(\theta) \tilde{a} \tilde{a}^{\dagger} | 0, \tilde{0}} +
\Braket{0,\tilde{0}| \cosh^{2}(\theta) \tilde{a} \tilde{a}^{\dagger} | 0,
\tilde{0}} \\
&= \sinh^2(\theta) + \cosh^{2}(\theta) = \cosh(2\theta)
\end{align}
So, $ \Delta q \propto \sqrt{\cosh(2\theta)} $.

To evaluate $ \Braket{p^{2}} $ we can recycle some information. $ p^{2} =
\frac{a a^{\dagger} + a^{\dagger}a - (a^{\dagger})^{2} - a^{2}}{2} $. From the
previous work it is clear that only the first two terms will contribute to the
expectation value. But, these are exactly the same terms as before. Thus, the
result is the same.


So, $ \Delta p = \sqrt{\cosh(2\theta)/2} $ and $ \Delta q \Delta p =
\frac{\cosh(2\theta)}{2} $.

This is minimum at $ \theta = 0 $. A state for which $ \theta = 0 $ corresponds
to the regime of high frequency to temperature ratio ($ \omega / T \to \infty $).
\end{homeworkSection}
\end{homeworkProblem}
