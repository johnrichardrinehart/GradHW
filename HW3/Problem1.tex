\begin{homeworkProblem}

\begin{homeworkSection}{a}
Charge multipole expansions are given in terms of an integral over the charge density as follows: \[ q_{lm} = \int\limits_{\text{all space}} Y_{lm}^*(\theta',\phi')r'^l \rho(\vec{x'})d\tau' \]

Now, this homework problem has a charge density which can be expressed as follows: \[ \rho(\vec{x'}) = \frac{q\delta(|\vec{x'}|)\delta(\theta'-\pi/2)}{r'^2sin^2\theta'} \big( \delta(\phi') + \delta(\phi'- \pi/2) - \delta(\phi - \pi) - \delta(\phi - 3\pi/2) \big) \]

Additionally, the spherical harmonics, $Y_{lm}$s, can be written in terms of associated Legendre polynomials as: \[Y_{lm}^*(\cos\theta) = \gamma_{lm}P_{lm}(\cos\theta)\exp(-im\phi) \]

Given all of this.

\begin{align*}
	q_{lm} = \int\limits_0^{2\pi} \int\limits_0^\pi \int\limits_0^\infty& Y_{lm}(\theta',\phi')r'^l \\ &\bigg( \frac{q\delta(|\vec{x'}|)\delta(\theta'-\pi/2)}{r'^2sin^2\theta'} \big( \delta(\phi') + \delta(\phi'- \pi/2) - \delta(\phi - \pi) - \delta(\phi - 3\pi/2) \big) \bigg) r'^2 sin^2\theta' \de r' \de\theta' \de\phi' \\
\end{align*}
\begin{align*}
	 &q_{lm} = q a^l Y_{lm}^*(\pi/2,0)+Y_{lm}^*(\pi/2,\pi/2) -Y_{lm}^*(\pi/2,\pi)-Y_{lm}^*(\pi/2,3\pi/2) \\
   &Y_{lm}(\theta,\phi) = \gamma_{lm} P_{lm}(\cos\theta)\exp(-i m \phi) \textit{,}\quad \textit{So,} \quad Y_{lm}(\pi/2,a) = \gamma_{lm} P_{lm}(0)\exp(-i m a)
\end{align*}
For a = $0,\pi/2,\pi,3\pi/2$ the expression for $\exp(-i m a)$ can be reduced as $1,(-i)^m,(-1)^m,i^m$, respectively. Given this, the expression for the multipoles can be recast as follows:

\[
    q_{lm} = q a^l \gamma_{lm} P_{lm}(0)\big( (1-(-1)^m) - (i^m - (-i)^m) \big)
\]

This can be easily seen to be zero for even $m$. Consider $f(m) =  (1-(-1)^m) - (i^m - (-i)^m $. $f(1) = 2-2i,\, f(3) = 2+2i,\, f(5) = 2-2i,\, f(7) = 2+2i,\, \cdots$. Additionally, though, $P_{lm}(x)$ is odd for any combination of $l$ and $m$ that is also odd. Since $m$ is constrained to be odd for nonzero $q_{lm}$ then $l$ must be constrained to be odd as well for nonvanishing $q_{lm}$ Thus, the final expression for $q_{lm}$ can be reduced:

\[
	q_{2j+1,2k+1} = 2q a^{2j+1} \big\{ 1+(-i)^{2k+1} \big\}Y_{2j+1,2k+1}P_{2j+1,2k+1}(0)
\]
\end{homeworkSection}

\begin{homeworkSection}{b}
	In a similar fashion as before, we can construct $\rho(
	
	q_{lm} = 
\end{homeworkSection}
\end{homeworkProblem}