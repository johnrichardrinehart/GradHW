\begin{homeworkProblem}[Fluctuation-Dissipation Theorem]
   \problemStatement{
      \begin{equation}
         m \frac{dv(t)}{dt} = \mathcal{F}(t) + F(t) \label{eq:eq_of_motion}
      \end{equation}
   }

   % Problem 2.1
   \subsection{Integration}
   \label{sub:integration}
%   \problemStatement{
%   }

   I'll start by writing the integral of Eq.~\ref{eq:eq_of_motion}.

   \begin{align*}
      \int_{t'}^{t'+\tau} m \frac{dv(t)}{dt} dt &= \int_{t'}^{t'+\tau}
      \bigl(\mathcal{F}(t) + F(t)\bigr) dt \\
      m \int_{t'}^{t'+\tau} \frac{dv(t)}{dt} dt &= \int_{t'}^{t'+\tau}
      \bigl(\mathcal{F}(t) + F(t)\bigr) dt
      \intertext{Then, by the second fundamental theorem of calculus, the left
      hand side of the above can be expressed as $ v(t) $ evaluated at the
      bounds of the integral.}
      m \Bigl(v(t'+\tau) - v(t')\Bigr)&= \int_{t'}^{t'+\tau}
      \bigl(\mathcal{F}(t) + F(t)\bigr) dt
      \intertext{Now, over a macroscopic (large) time interval $ \tau $, the
      external force $ \mathcal{F}(t) $ does not vary appreciably; we will
      assume that it does not change at all over this time scale. Thus, $
      \mathcal{F}(t) $ can be pulled out of the integral. However, we can not do
      the same with $ F(t) $, only because it is assumed to be fluctuating much
      over $ \tau $. Thus, we obtain the desired form, below.}
      m \Bigl(v(t'+\tau) - v(t')\Bigr) &= \mathcal{F}(t')\tau + \int_{t'}^{t'+\tau} F(t) dt
   \end{align*}

   % Problem 2.2
   \subsection{Decomposition of \texorpdfstring{$F(t)$}{F(t)}}
   \label{sub:decomposition_of_f_t}
%   \problemStatement{
%   }

   The component that will make the particle ultimately return to the
   equilibrium value of the velocity is $ f(t) $. It must be negative such that
   the velocity at some later time is less than that of some earlier time.
   Consider the recently derived expression in the context of an absent external
   force.

   \begin{align*}
      m \Bigl(v(t'+\tau) - v(t')\Bigr) &= \mathcal{F}(t')\tau + \int_{t'}^{t'+\tau} F(t) dt
      \\
      &= \int_{t'}^{t'+\tau} \Bigl( \langle F(t) \rangle + f(t) \Bigr) dt
      \intertext{But, $ \langle F(t) \rangle $ is $ 0 $, by construction.}
      &= \int_{t'}^{t'+\tau} f(t) dt
   \end{align*}
   The particles must slow down since there is no external force
   supplying energy to the system of particles. In order that the particles
   slow down, they must have acceleration that is oppositely signed relative
   to their velocity. Another way to say this is that the velocity at some
   greater time must be less than that at some later time so that $ v(t'+\tau) -
   v(t') < 0 $. Without loss of generality, we can assume that the
   velocity is in the positive direction. Then, the right hand side must be
   negative, which means that $ f(t) $ must be somewhere (if not everywhere)
   negative over $ t' $ to $ t' + \tau $. If one breaks $ f(t) $ into a slowly
   varying component and a quickly varying component, the slowly varying
   component would carry the information that returns the particles to the
   equilibrium velocity.

   % Problem 2.3
   \subsection{Restoring force Description \texorpdfstring{$F(t)$}{F(t)}}
   \label{sub:restoring_force_description_$f(t)$_f_t_}
%   \problemStatement{
%   }
   The typical form a restoring force is one which is negatively signed with
   respect to the coordinate of the thing that is moving. For example, in the
   case of a spring, the restoring force has the following form
   \[
      F_{\textrm{spring}} = - k x
   \]
   , where k represents the stiffness of the spring and $ x $ represents the
   position of the end of the spring relative to its unstretched position
   (stretched being in the positive direction and compressed being in the
   negative direction). In this case, however, it is not the position of the
   spring that we are considering but the velocity of a particle. Thus, the
   force would have the following velocity dependence,
   \[
      F(v) = -a v
   \]
   for some non-zero, positive $a$. The size of $ a $ determines the rate at
   which the particle returns to its equilibrium velocity (like the spring
   constant determines the rate at which the spring returns to its equilibrium
   position).

   % Problem 2.4
   \subsection{Ensemble-averaged Force - (I)}
   \label{sub:ensemble_averaged_force_i_}
%   \problemStatement{
%   }

   We have
   \[
      \langle F(t)  \rangle = \sum^{}_{r} P_{r}(t+\tau') F_{r}\enskip.
   \]
   But, we can write $ P_{r}(t+\tau') = P_{r}(t) \exp(\beta \Delta E) $. So,
   now,
   \[
      \langle F(t)  \rangle = \sum^{}_{r} P_{r}(t) \exp(\beta \Delta E) F_{r}\enskip.
   \]
   But, $ \exp(x) = 1 + x + x^{2} + x^{3} + \mathcal{O}(x^{4}) $. So, finally,
   to first order in $ \beta \Delta E $,
   \[
      \langle F(t)  \rangle \approx \sum^{}_{r} P_{r}(t) \Bigl( 1 + \beta \Delta
      E \Bigr) F_{r}\enskip.
   \]
   As mentioned in the problem statement $ \sum^{}_{r} P_{r}(t+\tau')F_{r} = 0
   $. So,
   \[
      \langle F(t)  \rangle \approx \sum^{}_{r} P_{r}(t) \beta \Delta E
      F_{r}\enskip.
   \]

   % Problem 2.5
   \subsection{Ensemble-averaged Force - (II)}
   \label{sub:ensemble_averaged_force_ii_}
%   \problemStatement{
%   }

   The integral is a function of $ t'' $. So, $ t \to t - t' $ and $ t' \to 0 $.
   Also, $ dt'' \to ds'' $. Thus the integral should be written as
   \[
      -\beta \langle v(t) \rangle \int_{t-t'}^{0} \langle F(t') F(t'+s) \rangle ds''
   \]

   % Problem 2.6
   \subsection{Fluctuation-Dissipation Theorem}
   \label{sub:fluctuation_dissipation_theorem}
%   \problemStatement{
%   }

   The argument that the double integral can be approximated by the single
   integral above is somewhat complicated. First, the limits of the integral
   form a domain that is integrated over. Remembering that $ s $ is a function
   of $ t' $, one can discover that the lower limit of the $ s $ integral is
   over a line through the $ st' $ plane: $ s(t') = t - t' $. The upper bound is
   the line $ s = 0 $. The domain of integration over $ s $ is between these two
   lines. The lower limit of the $ t' $ integral is $ t $ (which corresponds to
   $ s(t) = 0 $, for the lower bound of the $ s $ integral) and the upper limit
   of the $ t' $ integral is $ t+\tau $ (which corresponds to $ s(t+\tau) = -
      \tau $, for the lower bound of the $ s $ integral). Thus, the region of
      integration over the $ st' $ plane is a triangle with points $ (0,t) $, $
      (0,t+\tau)$ and $ (t+\tau,-\tau) $.

      Realizing that $ \langle F(t') F(t'+s) \rangle_{0} $ is not dependent on $
      t'$, but only on $ s $, it would be great to integrate over $ t' $ first
      and then over $ s $. So, we exchange the order of integration. Taking the
      above result and casting the points into the $ t's $ plane instead of the
      $ st' $ plane yields
      \[
         \int_{-\tau}^{0} \langle F(t') F(t'+s) \rangle_{0} ds
         \int_{t-s}^{t+\tau} dt'
         = \int_{-\tau}^{0} \langle F(t') F(t'+s) \left( \tau + s \right) ds
            \enskip.
      \]
      Now, it's important to consider the scale of the quantities in the
      problem. $ \tau $ is the macroscopic time scale, which is very large. The
      correlation function drops off very quickly for large $ s $. This means
      that the integrand contributes very little when $ |s| \ge \tau $. However,
      when $ |s| \le \tau $, it can be ignored in comparison to $ \tau $ itself.
      So, it's now reasonable to turn $ \tau + s $ into $ \tau $ in the previous
      integral. Since the correlation function drops off very quickly for large
      $ s $ beyond $ \tau $ it is reasonable, also, to allow the lower limit to
      extend to $ -\infty $. Finally, since the correlation function is even, we
      can extend the integral to $ + \infty $ if we take half of that integral.
      So, we have
      \[
         \frac{\tau}{2} \int_{-\infty}^{\infty} \langle F(t') F(t'+s) \rangle ds
      \]

 \end{homeworkProblem}
