\documentclass{article}
\newcommand{\hmwkTitle}{Qubit Dynamics in Presence of Thermal Noise} % Assignment title
\newcommand{\hmwkDueDate}{\today} % Due date
\newcommand{\courseTitle}{QIC 890 - Intro. to Noise Processes} % Course/class
\newcommand{\hmwkClassInstructor}{Dr. Na Young Kim} % Teacher/lecturer
\newcommand{\hmwkAuthorName}{John Rinehart} % Your name
\newcommand{\sudentNumber}{20503440} % Your name
\newcommand{\position}{Ph.D. student, Physics-IQC}
%---Packages------------------------------------------------------------------
 %------------------------------------------------------
\usepackage[T1]{fontenc}
\usepackage{lmodern}
\usepackage[english]{babel}
%\usepackage[utf8,latin1]{inputenc}
\usepackage[T1]{fontenc}
\usepackage[usenames,dvipsnames]{pstricks}
\usepackage{epsfig}
\usepackage{pst-grad} % For gradients \usepackage{pst-plot} % For axes
\usepackage{pifont}
\graphicspath{{IMG/}}
\usepackage[absolute,overlay]{textpos}
\usepackage{hyperref}
\usepackage{xcolor}
\usepackage{calc}
\usepackage{chngcntr}
\usepackage{microtype}
\usepackage[parfill]{parskip}
\usepackage{fancyhdr} % Required for custom headers
\usepackage{lastpage} % Required to determine the last page for the footer
\usepackage{extramarks} % Required for headers and footers
\usepackage{graphicx} % Required to insert images
\usepackage{lipsum} % Used for inserting dummy 'Lorem ipsum' text into the template
\usepackage{amsmath,amsthm,amsxtra,amsfonts}
\usepackage[]{dsfont}
\usepackage[toc,page]{appendix}
\usepackage[framemethod=TikZ]{mdframed}
\usepackage[]{siunitx}
\usepackage{circuitikz}
\usepackage{caption}
\usepackage[]{natbib}
\usetikzlibrary{arrows}
%\usepackage{tikz}
%\usetikzlibrary[]{external}
%\tikzexternalize{prefix=tikz/}
% Margins
\topmargin=-0.45in
\evensidemargin=0in
\oddsidemargin=0in
\textwidth=6.5in
\textheight=9.0in
\headsep=0.25in

\linespread{1.1} % Line spacing

% Set up the header and footer
\pagestyle{fancy}
\lhead{\hmwkAuthorName} % Top left header
\rhead{\courseTitle\ : \hmwkTitle} % Top center header
%\rhead{\firstxmark} % Top right header
\lfoot{\lastxmark} % Bottom left footer
\cfoot{} % Bottom center footer
\rfoot{Page\ \thepage\ of\ \pageref{LastPage}} % Bottom right footer
\renewcommand\headrulewidth{0.4pt} % Size of the header rule
\renewcommand\footrulewidth{0.4pt} % Size of the footer rule

\setlength\parindent{0pt} % Removes all indentation from paragraphs

%----------------------------------------------------------------------------------------
%	DOCUMENT STRUCTURE COMMANDS
%	Skip this unless you know what you're doing
%----------------------------------------------------------------------------------------

% Defines the problem answer command with thecontent as the only argument
\newcommand{\problemStatement}[1]{
\noindent
\begin{mdframed}[roundcorner=3pt]{
\begin{minipage}{0.98\columnwidth}
\begin{flushleft}#1\end{flushleft}
\end{minipage}}
\end{mdframed}
}

% Header and footer for when a page split occurs within a problem environment
\newcommand{\enterProblemHeader}[1]{
\nobreak\extramarks{#1}{#1 continued on next page\ldots}\nobreak
\nobreak\extramarks{#1 (continued)}{#1 continued on next page\ldots}\nobreak
}

% Header and footer for when a page split occurs between problem environments
\newcommand{\exitProblemHeader}[1]{
\nobreak\extramarks{#1 (continued)}{#1 continued on next page\ldots}\nobreak
\nobreak\extramarks{#1}{}\nobreak
}

%\renewcommand{\thesection}{}
%\renewcommand{\thesubsection}{\arabic{section}.\arabic{subsection}}
%\makeatletter
%\def\@seccntformat#1{\csname #1ignore\expandafter\endcsname\csname the#1\endcsname\quad}
%\let\sectionignore\@gobbletwo
%\let\latex@numberline\numberline
%\def\numberline#1{\if\relax#1\relax\else\latex@numberline{#1}\fi}
%\makeatother

% Try to hide the first digit in the section number
%\makeatletter
%\renewcommand{\thesection}{}
%\makeatother
%\setcounter{secnumdepth}{0} % Removes default section numbers
\newcounter{homeworkProblemCounter} % Creates a counter to keep track of the number of problems
\newcommand{\homeworkProblemName}{}
\newenvironment{homeworkProblem}[1][Problem \arabic{homeworkProblemCounter}]{ % Makes a new environment called homeworkProblem which takes 1 argument (custom name) but the default is "Problem #"
\stepcounter{homeworkProblemCounter} % Increase counter for number of problems
\renewcommand{\homeworkProblemName}{#1} % Assign \homeworkProblemName the name of the problem
\section*{\homeworkProblemName} % Make a section in the document with the custom problem count
\addcontentsline{toc}{section}{\homeworkProblemName}
\stepcounter{section} % Increase counter for section
\enterProblemHeader{Problem \arabic{homeworkProblemCounter}} % Header and footer within the environment
}{
\exitProblemHeader{Problem \arabic{homeworkProblemCounter}} % Header and footer after the environment
}

\newcommand{\problemAnswer}[1]{ % Defines the problem answer command with the content as the only argument
\noindent\framebox[\columnwidth][c]{\begin{minipage}{0.98\columnwidth}\begin{center}#1\end{center}\end{minipage}} % Makes the box around the problem answer and puts the content inside
}

\newcommand{\homeworkSectionName}{}
\newenvironment{homeworkSection}[1]{ % New environment for sections within homework problems, takes 1 argument - the name of the section
\renewcommand{\homeworkSectionName}{#1} % Assign \homeworkSectionName to the name of the section from the environment argument
\subsection*{\homeworkSectionName} % Make a subsection with the custom name of the subsection
\addcontentsline{toc}{subsection}{\homeworkSectionName}
\enterProblemHeader{\homeworkProblemName} % Header and footer within the environment
}{
\enterProblemHeader{\homeworkProblemName} % Header and footer after the environment
}

%%%%%%%%%%
\newcommand{\blu}[1]{\textcolor[rgb]{0,0,1}{#1}}
\newcommand{\bs}[1]{\boldsymbol{#1}}
%\newcommand{\V}[1]{\bm{#1}}
\newcommand{\V}[1]{\Vec{#1}}
\newcommand{\A}[1]{\Hat{#1}}
\newcommand{\W}[1]{\widehat{#1}}
\newcommand{\T}[1]{\widetilde{#1}}

\newcommand{\pd}[2]{\dfrac{\partial #1}{\partial #2}}
\newcommand {\ppds}[2]{\dfrac{\partial^2 {#1}}{\partial {#2}^2}}
\newcommand{\ppdss}[2]{\dfrac{\partial^2}{\partial #1 \partial #2}}
\newcommand{\pdtt}[3]{\dfrac{\partial^2 {#1}}{\partial {#2} \partial {#3}}}

\newcommand{\fpd}[2]{\frac{\partial #1}{\partial #2}}
\newcommand{\fpds}[1]{\frac{\partial}{\partial #1}}

\newcommand{\ignore}[1]{}

\newcommand{\der}[2]{\frac{d{#1}}{d{#2}}}
\newcommand{\vt}[1]{\Vec{\mathcal{#1}}}
\newcommand{\VP}[1]{\Vec{\mathbf{#1}}}
\newcommand{\vp}[1]{\mathbf{#1}}
\newcommand{\phas}[1]{\angle{#1}^{\circ}}
\newcommand{\er}{\epsilon_{r}}
\newcommand{\mr}{\mu_{r}}
\newcommand{\Lrw}{\Longrightarrow}
\newcommand{\refeq}[1]{(\ref{#1})}
\newcommand{\abs}[1]{\left| #1\right|}
%\newcommand{\ket}[1]{|#1\rangle}
%\newcommand{\bra}[1]{\langle #1| }
\newcommand{\intas}{\int\limits_{all\;space}}
\newcommand{\intsradial}[3]{\int\limits_{#1}^{#2} #3 r^2 \mathrm{d} r}
\newcommand{\intspolar}[3]{\int\limits_{#1}^{#2} #3 sin(\theta) \mathrm{d} \theta}
\newcommand{\intsazim}[3]{\int\limits_{#1}^{#2} #3 \mathrm{d} \phi}
\newcommand{\intcz}[3]{\int\limits_{#1}^{#2} #3 \mathrm{d} z}
\newcommand{\intcx}[3]{\int\limits_{#1}^{#2} #3 \mathrm{d} x}
\newcommand{\intcy}[3]{\int\limits_{#1}^{#2} #3 \mathrm{d} y}
\newcommand{\intavcart}[1]{\int \limits_{all\; space} #1 \, \mathrm{d} x \mathrm{d} y \mathrm{d} z}
\newcommand{\bracket}[2]{\langle#1|#2\rangle }
\newcommand*{\myalign}[2]{\multicolumn{1}{#1}{#2}}

\DeclareMathOperator{\Tr}{Tr}
% Not included in amsmath
\DeclareMathOperator{\sech}{sech}
\DeclareMathOperator{\csch}{csch}
\DeclareMathOperator{\arcsec}{arcsec}
\DeclareMathOperator{\arccot}{arcCot}
\DeclareMathOperator{\arccsc}{arcCsc}
\DeclareMathOperator{\arccosh}{arcCosh}
\DeclareMathOperator{\arcsinh}{arcsinh}
\DeclareMathOperator{\arctanh}{arctanh}
\DeclareMathOperator{\arcsech}{arcsech}
\DeclareMathOperator{\arccsch}{arcCsch}
\DeclareMathOperator{\arccoth}{arcCoth} 

% New definition of square root:
% it renames \sqrt as \oldsqrt
\let\oldsqrt\sqrt
% it defines the new \sqrt in terms of the old one
\def\sqrt{\mathpalette\DHLhksqrt}
\def\DHLhksqrt#1#2{%
\setbox0=\hbox{$#1\oldsqrt{#2\,}$}\dimen0=\ht0
\advance\dimen0-0.2\ht0
\setbox2=\hbox{\vrule height\ht0 depth -\dimen0}%
{\box0\lower0.4pt\box2}}

%%%---
\newcommand\ointint{\begingroup
\displaystyle \unitlength 1pt
\int\mkern-7.2mu
\begin{picture}(0,3)
\put(0,3){\oval(10,8)}
\end{picture}
\mkern-7mu\int\endgroup}
%%%----
\providecommand{\abs}[1]{\lvert#1\rvert}
\providecommand{\norm}[1]{\lVert#1\rVert}

%%%%%%%%%%%%%%%


% Matlab code section obtained from StackExchange: http://tex.stackexchange.com/questions/75116/what-can-i-use-to-typeset-matlab-code-in-my-document
\usepackage{listings}
\usepackage{color} %red, green, blue, yellow, cyan, magenta, black, white
\definecolor{mygreen}{RGB}{28,172,0} % color values Red, Green, Blue
\definecolor{mylilas}{RGB}{170,55,241}

\lstset{language=Matlab,%
    %basicstyle=\color{red},
    breaklines=true,%
    morekeywords={matlab2tikz},
    keywordstyle=\color{blue},%
    morekeywords=[2]{1}, keywordstyle=[2]{\color{black}},
    identifierstyle=\color{black},%
    stringstyle=\color{mylilas},
    commentstyle=\color{mygreen},%
    showstringspaces=false,%without this there will be a symbol in the places where there is a space
    numbers=left,%
    numberstyle={\tiny \color{black}},% size of the numbers
    numbersep=9pt, % this defines how far the numbers are from the text
    emph=[1]{for,end,break},emphstyle=[1]\color{red}, %some words to emphasise
    %emph=[2]{word1,word2}, emphstyle=[2]{style},
}

%----------------------------------------------------------------
\numberwithin{equation}{section}
%\numberwithin{equation}{chapter}
%\renewcommand{\theequation}{\arabic{equation}}



%%------------------------------------------------------------------------------------------
%	TITLE PAGE
%----------------------------------------------------------------------------------------

\title{
\vspace{2in}
\textmd{\textbf{\courseTitle\\ \vspace{0.5in}\hmwkTitle}}\\
\vspace{0.5in}\large{{\hmwkClassInstructor}}
\vspace{3in}
}
\author{\textbf{\hmwkAuthorName}\\
\date{\hmwkDueDate}}

\begin{document}
\maketitle

\newpage
\tableofcontents
\newpage

\section{Noise of Finite-Length Resistor Subject to Thermal Gradient}
\label{sec:noise_of_finite_length_resistor_subject_to_thermal_gradient}

Consider the problem of a resistor $ R $ of length $ l $ subject to a thermal
gradient induced by the resistor being connected to two thermal baths at each of
its ends, $ T_{1} $ at one end and $ T_{2} $ at the other (see
Fig.~\ref{fig:resistor_at_two_temps}). This is relevant to the experimentalist
working with resistive cabling in a dilution refrigerator. Using the result of
the noise generated by a resistor at a single finite temperature we will
determine the noise generated by this resistor. In the effort to maintain
complete generality we will use the following expression for the spectral
density of a noisy resistor derived by Nyquist in 1928 which applies over every
frequency and temperature\cite{Nyquist1928}:
\begin{equation}
   S_{V}(f) = \frac{4 h f R}{e^{\frac{hf}{k_{B}T}}-1} \enskip.
   \label{eq:nyqist_noise_expression}
\end{equation}

The expression for the spectral density has units of
\SI{}{\volt\squared\per\hertz}. The above expression is applicable to a resistor
$ R $ anchored at both ends to a bath at a single temperature. We are
considering a resistor anchored to baths at two (possibly) different
temperatures. In order to solve this problem, the first thing that we will need
to do is determine the distribution of temperature along the resistor. The
temperature distribution of the resistor must satisfy the heat diffusion
equation
\begin{equation}
   \frac{\partial T(x,y,z,t)}{\partial t} - \alpha \nabla^{2}T(x,y,z) = 0
   \enskip.
\end{equation}
Above, $ T(x,y,z,t) $ is the temperature of the object at the position $
\left( x,y,z \right)$ at time $ t $ and $ \alpha $ is some diffusion constant
associated with the materials of the object.
Now, in this case, the resistor is assumed to be a one-dimensional object.
Assuming the resistor to be oriented along a straight line in the x-direction,
the above equation can be recast as
\begin{equation}
   \frac{\partial T(x,t)}{\partial t} - \alpha \frac{\partial^2 T(x)}{\partial x^{2}} = 0
   \enskip. \label{eq:one-dimensional-diffusion-equation}
\end{equation}
If the resistor is allowed to come to a steady state then all time variations of
the temperature vanish. This simplifies
Eq.~\ref{eq:one-dimensional-diffusion-equation} to
\begin{equation}
   \frac{\partial^2 T(x)}{\partial x^{2}} = 0
   \enskip. \label{eq:one-dimensional-diffusion-equation-in-steady-state}
\end{equation}
The solutions to the above equation are simple and are given by $ T(x) = Ax+B $ with $ A
$ and $ B $ being determined by the boundary conditions of the problem. In this
case, the boundary conditions of the problem determine that
\[
   T(0) = T_{1}
\]
and
\[
   T(l) = T_{2} \enskip.
\]
Applying these boundary conditions to the above expression determine that
\[
   A = \frac{T_{2}-T_{1}}{l}
\]
and
\[
   B = T_{1} \enskip.
\]

\begin{figure}[h]
   \centering
%      \captionsetup{width=.8\textwidth}
      \input{resistorattwotemps.tikz}
      \caption{A resistor of length $l$ and resistance $R$ anchored at its ends
      to thermal baths at temperatures $ T_{1} $ and $ T_{2} $.}
   \label{fig:resistor_at_two_temps}
\end{figure}
Thus $ T(x) = \frac{T_{2}-T_{1}}{l}x +T_{1} $. Now, we will adopt the following
approach to determine the noise generated by the entire resistor: Discretize the
resistor into many resistive segments connected in series; then, express the
noise of the entire resistor in terms of a Riemann sum which will be recast as
an integral in the limit of an infinite number of segments. This is a valid
approach under the following considerations: The noise of adjacent segments is
independent (and, so, are uncorrelated). That is, considering the voltage noise
generated by a segment $S_{a}$, $ V_{S_{a}} $, and another segment $ S_{b} $, $
V_{S_{b}} $, $ \langle V_{S_{a}} V_{S_{b}} \rangle = \langle V_{S_{a}} \rangle
\langle V_{S_{b}} \rangle$ (independent) and $ \langle V_{S_{a}} V_{S_{b}}
\rangle - \langle V_{S_{a}} \rangle \langle V_{S_{b}} \rangle = 0 $
(uncorrelated).

Considering the physical origin of such noise (thermal agitation of electrons)
these assumptions may seem unreasonable. It may be the case that the agitation
of one electron exerts a force on nearby electrons affecting their agitation.
This would imply that the position of one electron (a random variable affecting
the potential/voltage of the electron) is correlated (alternatively stated, that
it shares information) with other electrons. However, for the sake of
mathematical tractability, this will be ignored. If physical observations of
thermal noise do not correspond to the derived expression, then this should be
one of the first assumptions to reconsider. However, assuming for now that this
assumption is not unreasonable we can express the noise of the ith resistor
section, $ R_{i} $, as
\begin{equation}
   S_{V}(f) = \frac{4 h f R_{i}}{e^{\frac{hf}{k_{B}T_{i}}}-1} \enskip.
   \label{eq:nyqist_noise_expression_discretized}
\end{equation}
According to the expression for the temperature distribution of the conductor,
we have
\begin{equation}
T(R_{i}) = \frac{T_{2}-T_{1}}{l}x_{i} +T_{1} \enskip.
\label{eq:resistive_section_temperature}
\end{equation}
If we assume that the resistor is discretized into $ N $ sections of length $
l/N $ and we index each section by $ i $ such that the ith section is at position
$ x_{i} = li/N $ then we can rewrite Eq.~\ref{eq:resistive_section_temperature}
as
\begin{equation}
T(R_{i}) = \frac{T_{2}-T_{1}}{l}\frac{li}{N} +T_{1} =  \frac{T_{2}-T_{1}}{N}i +T_{1}\enskip.
\label{eq:resistance_of_ith_resistive_section}
\end{equation}
Note that we have chosen to index the sections from $ 1,\ldots,N $ such that
there are $ N $ sections and the last section is the Nth section. Thus, the
temperature of the first section is
\begin{equation}
   T(R_{1}) = \frac{T_{2}- T_{1}}{N} + T_{1} = T_{1} \left( 1 - \frac{1}{N} \right)
   + \frac{T_{2}}{N} \enskip.
\end{equation}
Clearly the first section is not at temperature $ T_{1} $ for any finite number
of sections. Thus, it is requisite that we take the limit as $ N \rightarrow
\infty $. In order to write an expression for the noise of the entire resistor
we must consider how to combine the noise of adjacent resistors. Clearly if we
consider the noise of two resistors $ R_{1} $ and $ R_{2} $ both at the same
temperature $ T $, the noise generated from the series connection of these two
resistors should be the same as that generated from a resistor of value $ R =
R_{1} + R_{2} $.
\begin{equation}
   S_{V_{R}}(f) = \frac{4 h f R}{e^{\frac{hf}{k_{B}T}}-1}=
   \frac{4 h f \left( R_{1} + R_{2} \right)}{e^{\frac{hf}{k_{B}T}}-1}
   = S_{V_{R_{1}}} + S_{V_{R_{2}}} \enskip.
\end{equation}
Maybe unsuprisingly, the power spectral densities of individual resistors are
additive. So, the noise spectral density from the entire resistor is just the
sum of all of the spectral densities of each of the sections:
\begin{equation}
   S_{V_{R}}(f) = \lim_{N \rightarrow \infty}\sum^{N}_{i=1}
   S_{V_{R_{i}}}(f)
   = \lim_{N \rightarrow \infty}\sum^{N}_{i=1} \frac{4 h f
   R_{i}}{e^{\frac{hf}{k_{B}T_{i}}}-1}
   = \lim_{N \rightarrow \infty}\sum^{N}_{i=1}\frac{1}{N} \frac{4 h f
   R}{e^{\frac{hf}{k_{B}T_{i}}}-1} \label{eq:spectral_density_riemann}
\end{equation}
where, above, we made the assumption that all N resistive sections are of value $
R/N$. Now, in order to express this in terms of an integral we need to take the
limit as some quantity (representing some interval) gets very small. In this
case, we can imagine the temperature difference across each resistive section
shrinking to zero as the number of sections increases to infinity. Considering
this, then, the temperature difference across the ith resistive section, $
\Delta T $ is
\begin{equation}
   \Delta T = \left(\frac{i+1}{N} \left( T_{2} - T_{1} \right) + T_{1}\right) - \left( \frac{i}{N}
\left( T_{2} - T_{1} \right) + T_{1} \right) = \frac{T_{2} - T_{1}}{N} \enskip.
\end{equation}
This allows us to replace instances of $ \frac{1}{N} $ in
Eq.~\ref{eq:spectral_density_riemann} with $ \Delta T $ as
\begin{equation}
   S_{V_{R}}(f) = \lim_{\Delta T \rightarrow 0}\sum^{N}_{i=1} \frac{\Delta
   T}{T_{2} - T_{1}}\frac{4 h f R}{e^{\frac{hf}{k_{B}T_{i}}}-1} \enskip.
   \label{eq:spectral_density_intermediate_form}
\end{equation}
Rewriting our expression for the temperature of the ith resistive section in
Eq.~\ref{eq:resistance_of_ith_resistive_section} terms of $ \Delta T $, now,
yields
\begin{equation}
   T(R_{i}) = i\Delta T +T_{1} \enskip.
\end{equation}
Substituting the above expression into
Eq.~\ref{eq:spectral_density_intermediate_form} yields
\begin{equation}
   S_{V_{R}}(f) = \lim_{\Delta T \rightarrow 0}\sum^{N}_{i=1} \frac{\Delta
   T}{T_{2} - T_{1}}\frac{4 h f R}{\exp\left(\frac{hf}{k_{B} \left( T_{1} + i \Delta T \right)}\right)-1}
   = \frac{4 h f R}{T_{2} - T_{1}}\lim_{\Delta T \rightarrow 0}\sum^{N}_{i=1}
   \frac{\Delta T}{\exp\left(\frac{hf}{k_{B} \left( T_{1} + i
   \Delta T \right)}\right)-1} \enskip.
   \label{eq:spectral_density_final_intermediate_form}
\end{equation}
Now, the definition of an integral in terms of a Riemann sum is exactly
\begin{equation}
   \int_{a}^{b} f(x) dx = \lim_{\Delta x \rightarrow 0} \sum^{N}_{i=1} f(a +
   i \Delta x) \Delta x \enskip.
\end{equation}
So, the expression Eq.~\ref{eq:spectral_density_final_intermediate_form} can be
cast as
\begin{equation}
   S_{V_{R}}(f) = \frac{4 h f R}{T_{2} - T_{1}} \int_{T_{1}}^{T_{2}} \left(
   \exp\left( \frac{hf}{k_{B}T}\right) - 1 \right)^{-1} dT \enskip.
   \label{eq:spectral_density_of_resistor_in_integral_form}
\end{equation}
In the event that the resistor $ R $ has the property that the resistance
changes as a function of temperature, it is easy to modify the above expression
to account for this. Assuming the resistor has a temperature dependence $ R(T)
$, the spectral density of the entire resistor can be calculated as
\begin{equation}
   S_{V_{R}}(f) = \frac{4 h f}{T_{2} - T_{1}} \int_{T_{1}}^{T_{2}} R(T) \left(
   \exp\left( \frac{hf}{k_{B}T}\right) - 1 \right)^{-1} dT \enskip.
   \label{eq:spectral_density_of_entire_resistor_in_integral_form}
\end{equation}
In the appendix, the expression of the noise generated by a resistor at one
temperature in the classical limit is obtained. If we want the autocorrelation
of this noisy process, the Wiener-Khintchine theorem allows us to obtain this
through a (inverse) Fourier transform.
\begin{equation}
   \phi_{V_{R}}(t) = \int_{-\infty}^{\infty} S_{V_{R}}(f) e^{i
   2 \pi f t} df
\end{equation}
Note that the spectral density diverges in the limit of large $ f $. Thus, the
proper Fourier integral can not be taken. Rather, it must be windowed for any
application.

\section{Deriving the Spin-Boson Master Equation}
\label{sec:deriving_the_spin_boson_master_equation}

One stochastic model of particular interest to experimentalists building quantum
computers is the spin-boson model. It describes the
interaction of a two-level system (the spin) with a bath of harmonic
oscillators (the bosons). In the present context, the spin is meant to take the place of a
quantum bit, or qubit, and the bath of harmonic oscillators is the thermal bath
that is presented to the qubit by the resistive network. Having knowledge of the
resistive network it may be possible to describe the time evolution of a state
including decoherence (loss of phase information) and dissipation (loss of
energy).

This noisy process is of particular interest to the developers of
superconducting quantum computers which are designed to interface with
(approximate) two-level system through the use of resistive cabling. The
resistive cables introduce thermal noise. Coming to a detailed understanding of
how the resistive network affects the two-level system is of particular interest
to the designers of such systems in order that they can reduce the effects the
environment induces on the qubit (the spin or the two-level system).

\section{Deriving the Born-Markov Equation}
\label{sec:deriving_the_born_markov_equation}

The problem of solving for the time evolution of the state of the two-level
system coupled to a thermal bath has been solved (see \cite{Ithier2005},
\cite{Wilhelm2006}) but remains a problem of significant pedagogical interest. It
is the goal of this work for the author to elucidate the dynamics of the
two-level system subject to a thermal environment.

The dynamics of the two-level system coupled to a thermal environment relies on
the derivation of a master equation (differential equation describing a quantum
state subject to noise) called the Born-Markov equation. It describes the
evolution a quantum system (of any size) coupled to a noisy environment. Once
that has been determined it will be applied to the specific case of a qubit
(two-level system) coupled to a thermal environment (thermal bath of bosons).
The approach shown will be very similar to the work of \cite{Goddard2010} as it
was the primary reference for this work.

It is a known principle of quantum mechanics (Stinespring's dilation theorem)
that if the state space is chosen to be ``large enough'' (by adding ancilla
systems) that it evolves unitarily (this is the quantum version of the statement
``closed systems neither gain nor lose energy'').  Unitary evolution of a
quantum state $ \rho $, initialized as $ \rho(0) $, would then be given by the
unitary evolution of that state: $ \rho(t) = \hat{U}(t) \rho(0) U^{\dagger}(t)
$. This is easily justified by considering the Schr\"{o}dinger equation in the
context of a density matrix evolving according to the Hamiltonian of the system
represented by $ \rho $. Now, imagine that $ \rho $ is a state which exists in a
space existing as a combination of a ``system'' whose dynamics we are interested
in and an ``environment'' whose dynamics are immaterial. We can obtain the
evolution of only the state, $ \rho_{\textrm{sys}}(t) $ by tracing out the
environment as
\begin{equation}
   \rho_{\textrm{sys}}(t) = Tr_{\textrm{env}}\left(\rho_{\textrm{sys} \otimes
   \textrm{env}}\left(t\right)\right) \enskip.
\end{equation}
Henceforth, $ \textrm{sys} $ and $ \textrm{env} $ will be abbreviated in favor
of the unambiguous form $ s $ and $ e $. The details of how to
numerically/computationally perform a partial trace are immaterial to the
following discussion and, as such, will not be described here.  The above
notation of $ \textrm{sys} \otimes \textrm{env} $ describes a tensor product
space (a mathematical space determined by the combination of the system and the
environment). The reader is referred to \cite{Nielsen2011} for more information
regarding elementary quantum information operations like computing the partial
trace and the tensor product.

Determining the time evolution of the density matrix can be expressed as above
but, often, it is difficult, if not impossible, to determine the unitary
operator that generates time translations of the system-environment state space.
Plus, if we are only interested in the system, the computation of such a time
evolution is wasteful. It would be convenient to describe the time evolution of
the state in terms of something resembling the Schr\"{o}dinger equation.
However, the dissipation and decoherence introduced by the environment has an
effect on the system dynamics. Simply put, it would be nice to find an evolution
of the form
\begin{equation}
   \frac{d}{dt}\rho_{s}(t) = -i \left[ H, \rho_{s}(t) \right] + D \left(
   \rho_{s}(t) \right) \enskip.
\end{equation}
The form, above, describes unitary evolution according to \textit{some}
Hamiltonian (not necessarily, and probably not, the free system's Hamiltonian).
The Hamiltonian above only acts on the system state and ignores the state of the
environment. The evolution is described purely in terms of some effective system
Hamiltonian, the state of the system and an operator $ D $ which clearly must
contain information regarding the environment. Note that in the absence of $ D $
the dynamics are that of a closed system under the influence of its Hamiltonian
(the quantum Liouville equation). Clearly, the operator $ D $ addresses the
dissipation and decohorence (the choice of the letter $ D $ now being clear) and
includes the information regarding the environment. Another noteworthy feature
of the above expression is the system evolution's independence of the history of
$ \rho_{s}(t' < t) $. This can only be accomplished if the system and
environment have no ``memory''. Such an assumption implies the use of the Markov
assumption which is exactly this: A system is Markovian if its dynamics depend
on nothing other than the current time (in the case of continuous time) or the
immediately previous time (in the discrete time case).

To introduce a form from which this (and the Born) assumption can be made we
will consider, first, the time evolution of the combined system-environment
state. Then, we will trace out the environment making the Born and Markov
approximations, as needed. We will then obtain the Born-Markov master equation.
In general, the Hamiltonian of the combined system and environment, $
H_{\textrm{total}} $ can be
expressed as
\begin{equation}
   H_{\textrm{total}} = H_{\textrm{s}} \otimes \mathds{1}_{\textrm{e}} +
   \mathds{1}_{\textrm{s}} \otimes H_{\textrm{e}} + H_{\textrm{s} \otimes \textrm{e}}
\end{equation}
where $ H_{\textrm{s}} $ represents the system Hamiltonian absent the
environment (similarly with $ H_{\textrm{e}} $) and $ H_{\textrm{s} \otimes
\textrm{e}} $ represents the Hamiltonian exchanging information between the
system and the environment. The notation $ \mathds{1}_{\textrm{space}} $ will be
used to denote the identity operator over some space. It can be helpful to work
in a domain (picture) where the system and environment Hamiltonians are absent
and only the interaction Hamiltonian is relevant. This can be accomplished
through the transformation
\[
   \rho^{\textrm{int}}_{\textrm{s} \otimes \textrm{e}}(t)
   = e^{\frac{i}{\hbar} \left( H_{\textrm{s}} \otimes \mathds{1}_{\textrm{e}} +
      \mathds{1}_{\textrm{s}} \otimes H_{\textrm{e}}\right) t }
      \underbrace{\left( e^{- \frac{i}{\hbar} H_{\textrm{total}} t } \rho_{\textrm{s} \otimes \textrm{e}}(0)
      e^{ \frac{i}{\hbar} H_{\textrm{total}} t }\right)}_{\rho_{\textrm{s}\otimes
      \textrm{e}}(t)}
      e^{-\frac{i}{\hbar}\left( H_{\textrm{s}} \otimes \mathds{1}_{\textrm{e}} +
   \mathds{1}_{\textrm{s}} \otimes H_{\textrm{e}}\right) t } \enskip.
\]
The above transformation transforms $ \rho_{\textrm{s} \otimes \textrm{e}}(0) $ (of the combined system
and environment) according to the combined system and environment Hamiltonian;
then, this is transformed according to both the system and environment
Hamiltonian (the so-called ``free part'' of the Hamiltonian). Taking the time
derivative of this expression leaves us with a sum of four terms, each of which
is a product similar to the above (the time derivative of $
\rho_{\textrm{s}\otimes \textrm{e}}(0) $ is zero). The derivative of the first and last terms
determine a commutation relationship between the free part of the Hamiltonian
and the state in the interaction picture. The derivative of the second and
fourth terms determine a commutation relationship between the total Hamiltonian
and the time-evolved state $ \rho_{\textrm{s}\otimes \textrm{e}}(t) $. So, we
have
\begin{equation}
   \frac{d}{dt} \rho^{\textrm{int}}_{\textrm{s} \otimes \textrm{e}}(t)
   = \frac{i}{\hbar} \left[   H_{\textrm{s}} \otimes \mathds{1}_{\textrm{e}} +
      \mathds{1}_{\textrm{s}} \otimes H_{\textrm{e}},
   \rho^{\textrm{int}}_{\textrm{s} \otimes \textrm{e}}(t) \right]
      -
      \frac{i}{\hbar} e^{\frac{i}{\hbar} \left( H_{\textrm{s}} \otimes \mathds{1}_{\textrm{e}} +
      \mathds{1}_{\textrm{s}} \otimes H_{\textrm{e}}\right) t }
      \left[ H_{\textrm{total}}, \rho_{\textrm{s}\otimes \textrm{e}}(t) \right]
      e^{-\frac{i}{\hbar} \left( H_{\textrm{s}} \otimes \mathds{1}_{\textrm{e}} +
      \mathds{1}_{\textrm{s}} \otimes H_{\textrm{e}}\right) t } \enskip.
\end{equation}
Expanding $ H_{\textrm{total}} $ into its two parts, the free and interaction
parts, and inserting identity in the form of $ e^{-\frac{i}{\hbar} \left(
   H_{\textrm{s}} \otimes \mathds{1}_{\textrm{e}} + \mathds{1}_{\textrm{s}}
   \otimes H_{\textrm{e}}\right) t } e^{\frac{i}{\hbar} \left( H_{\textrm{s}}
   \otimes \mathds{1}_{\textrm{e}} + \mathds{1}_{\textrm{s}} \otimes
H_{\textrm{e}}\right) t } $ yields
\begin{equation}
   \frac{d}{dt} \rho_{\textrm{s} \otimes \textrm{e}}^{\textrm{int}}(t) =
   - \frac{i}{\hbar} \left[ H^{\textrm{int}}_{\textrm{s} \otimes \textrm{e}}(t),
   \rho_{\textrm{s} \otimes \textrm{e}}^{\textrm{int}}(t) \right]
   \label{eq:interaction_picture_state_commutator}
\end{equation}
where $ H^{\textrm{int}}_{\textrm{s} \otimes \textrm{e}}(t) $ is the time evolution of $
H_{\textrm{s} \otimes \textrm{e}} $ according to the interaction picture (that is, it is
evolved just like $ \rho_{\textrm{s} \otimes \textrm{e}}(t) \rightarrow
\rho^{\textrm{int}}_{\textrm{s} \otimes \textrm{e}}(t) $). We can integrate
Eq.~\ref{eq:interaction_picture_state_commutator} to solve for $
\rho_{\textrm{s} \otimes \textrm{e}}^{\textrm{int}}(t) $ and back-substitute this solution for $
\rho_{\textrm{s} \otimes \textrm{e}}^{\textrm{int}}(t) $ into Eq.~\ref{eq:interaction_picture_state_commutator}
to obtain
\begin{equation}
   \frac{d}{dt} \rho_{\textrm{s} \otimes \textrm{e}}^{\textrm{int}}(t) =
   -\frac{i}{\hbar} \left[ H_{\textrm{s} \otimes \textrm{e}}^{\textrm{int}}(t),
   \rho_{\textrm{s} \otimes \textrm{e}}^{\textrm{int}}(0) \right] -
   \frac{1}{\hbar^{2}}
   \int_{0}^{t} dt' \left[ H_{\textrm{s} \otimes \textrm{e}}^{\textrm{int}}(t),
      \left[ H_{\textrm{s} \otimes \textrm{e}}^{\textrm{int}}(t'),
   \rho_{\textrm{s} \otimes \textrm{e}}^{\textrm{int}}(t') \right] \right] \enskip.
   \label{eq:interaction_picture_integrodifferential_form_intermediate}
\end{equation}
Now, $ \left[ H_{\textrm{s} \otimes \textrm{e}}^{\textrm{int}}(t), \rho_{\textrm{s} \otimes \textrm{e}}^{\textrm{int}}(0) \right] $ is an
operator. Allow the result of the commutator to be some operator $ A $. If $ A $
were added to the interaction part of the total Hamiltonian and subtracted from
the interaction part of the total Hamiltonian (such that $ H_{\textrm{total}} $
remains unchanged) then the previous argument would still hold and the first
commutator of
Eq.~\ref{eq:interaction_picture_integrodifferential_form_intermediate} would be
zero. Thus, without loss of generality, that term can be dropped from the previous
expression. Now, we can trace over the environment in the above expression to
obtain $ \rho^{\textrm{int}}_{\textrm{s}} $ (the state of only the system in the
interaction picture) as
\begin{equation}
   \frac{d}{dt} \rho^{\textrm{int}}_{\textrm{s}}(t) =
    - \frac{1}{\hbar^{2}}
    Tr_{\textrm{e}} \left( \int_{0}^{t} dt' \left[ H_{\textrm{s} \otimes \textrm{e}}^{\textrm{int}}(t) ,
          \left[ H_{\textrm{s} \otimes \textrm{e}}^{\textrm{int}}(t') ,
       \rho_{\textrm{s} \otimes \textrm{e}}^{\textrm{int}}(t') \right] \right]
       \right) \enskip.
   \label{eq:interaction_picture_integrodifferential_form_final}
\end{equation}
So, before any approximations, we have the master equation of $
\rho_{\textrm{s}}^{\textrm{int}}(t) $. But, this expression has two undesirable properties:
\begin{enumerate}
   \item It depends on the combined state of the system and environment.
   \item It depends on the history of both the interaction Hamiltonian and the
      combined state of the system and environment.
\end{enumerate}
These problems can be alleviated by introducing the Born-Markov assumptions. The
Born assumption claims that the system is weakly coupled to the
environment. This has two effects:
\begin{enumerate}
   \item The initial combined state can be expressed as a product state: $
      \rho_{\textrm{s} \otimes \textrm{e}}(0) = \rho_{\textrm{s}}(0) \otimes
      \rho_{\textrm{e}}(0) $.
   \item The time evolution of the system is independent of that of the
      environment which does not evolve: $ \rho _{\textrm{s} \otimes
      \textrm{e}}(t)
      = \rho_{\textrm{s}}(t) \otimes \rho_{\textrm{e}}(0) $.
\end{enumerate}
Because the free Hamiltonian is is composed of parts which affect the system and
environment separately, it is clear that the time evolution of the combined
state in the interaction picture is the same as that in the Schr\"{o}dinger
picture; so, $ \rho_{\textrm{s} \otimes \textrm{e}}^{\textrm{int}} =
\rho^{\textrm{int}}_{\textrm{s}}(t) \otimes
\rho^{\textrm{int}}_{\textrm{e}}(0)$. Another tool that we will use to simplify
the above expression is the diagonal block expansion of the interaction
Hamiltonian. It can easily be shown that for some operator $ A \in
\mathcal{L}(S_{1} \otimes S_{2}) $ has a block decomposition in terms of the
canonical basis of $ S_{1} $. The canonical basis of $ S_{1} = E_{a,b} $, where
$ E_{a,b} $ is represented as a matrix in $ \mathcal{L}(S_{1}) $ with a $ 1 $ in
the $ (a,b) $ entry and $ 0 $ elsewhere. The decomposition can be expressed as
\begin{equation}
   A = \sum^{N}_{a=1} \sum^{N}_{b=1} E_{a,b} \otimes P_{a,b}
\end{equation}
where $ E_{a,b} \in \mathcal{L}(S_{1}) $
and $ P_{a,b} \in \mathcal{L}(S_{2}) $. Analogously, this can be written as
\begin{equation}
   B = \sum^{N}_{a=1} \sum^{N}_{b=1} P_{a,b} \otimes E_{a,b}
\end{equation}
where $ P_{a,b} \in \mathcal{L}(S_{2}) $ and $ E_{a,b} \in \mathcal{L}(S_{1}) $.
The tensor product is not commutative; in general $ A \ne B $. However, all
normal operators are diagonalizable; Hermitian matrices are diagonalizable.
Thus, there must exist some decomposition
of $ H_{\textrm{s} \otimes \textrm{e}} $ as
\begin{equation}
    H_{\textrm{s} \otimes \textrm{e}} = \sum^{N}_{a=1} S_{a} \otimes E_{a}
\end{equation}
where $ S_{a} \in \mathcal{L}(\textrm{system}) $ and $ E_{a} \in
\mathcal{L}(\textrm{environment}) $. The time evolution of this Hamiltonian in
the interaction picture is easily determined by the independent actions of the
system and environment Hamiltonians,
\begin{equation}
   H_{\textrm{s} \otimes \textrm{e}}^{\textrm{int}}(t)
   = e^{\frac{i}{\hbar} \left(
      H_{\textrm{s}}
      \otimes \mathds{1}_{\textrm{e}} + \mathds{1}_{\textrm{s}} \otimes
   H_{\textrm{e}} \right)t}
   \left( \sum^{N}_{a=1} S_{a} \otimes E_{a} \right)
   e^{-\frac{i}{\hbar} \left(
      H_{\textrm{s}}
      \otimes \mathds{1}_{\textrm{e}} + \mathds{1}_{\textrm{s}} \otimes
   H_{\textrm{e}} \right)t}
   = \sum^{N}_{a=1} S_{a}(t) \otimes E_{a}(t)
\end{equation}
with $ S_{a}(t) = e^{\frac{i}{\hbar} H_{s}  t } S_{a} e^{-\frac{i}{\hbar} H_{s}
t } $ and
$ E_{a}(t) =  e^{\frac{i}{\hbar}
H_{\textrm{e}} t } E_{a} e^{-\frac{i}{\hbar}
H_{\textrm{e}} t } $.

We can now use the Born assumption and the block expansion of the composite
system-environment Hamiltonian ($ H_{\textrm{s} \otimes \textrm{e}} $) to
rewrite Eq.~\ref{eq:interaction_picture_integrodifferential_form_final} as
\begin{align}
   \frac{d}{dt} \rho^{\textrm{int}}_{\textrm{s}}(t)
   &= - \frac{1}{\hbar^{2}}
    \Tr_{\textrm{e}} \left( \int_{0}^{t} dt' \left[ \sum^{N}_{i=1} S_{i}(t)
          \otimes E_{i}(t) , \left[ \sum^{N}_{j=1} S_{j}(t') \otimes E_{j}(t'),
    \rho^{\textrm{int}}_{\textrm{s}}(t') \otimes \rho^{\textrm{int}}_{\textrm{e}}(0) \right] \right] \right) \\
   &= - \frac{1}{\hbar^{2}}
    \Tr_{\textrm{e}} \left( \int_{0}^{t} dt' \sum^{N}_{i,j} \left[ S_{i}(t)
          \otimes E_{i}(t) , \left[ S_{j}(t') \otimes E_{j}(t'),
   \rho^{\textrm{int}}_{\textrm{s}}(t') \otimes
   \rho^{\textrm{int}}_{\textrm{e}}(0) \right] \right] \right) \enskip. \\
   \intertext{Now, expand the nested commutators and identify $
   \Tr_{\textrm{e}}\left(E_{i}(t) E_{j}(t') \rho_{\textrm{e}}(0)\right) $ as a
   function of $ t $ and $ t' $ indexed by $ i $ and $ j $, say $ F_{i,j}(t,t')
   $. Doing all of this yields}
   = - \frac{1}{\hbar^{2}} \Tr_{\textrm{e}} \biggl(  \int_{0}^{t} dt' \sum^{N}_{i,j}
       \Bigl(
          &\bigl( S_{i}(t) \otimes E_{i}(t) \bigr)
          \bigl( S_{j}(t') \otimes E_{j}(t') \bigr)
          \bigl( \rho^{\textrm{int}}_{\textrm{s}}(t') \otimes
          \rho^{\textrm{int}}_{\textrm{e}}(0) \bigr) \\
          -& \bigl( S_{i}(t) \otimes E_{i}(t) \bigr)
          \bigl( \rho^{\textrm{int}}_{\textrm{s}}(t') \otimes
          \rho^{\textrm{int}}_{\textrm{e}}(0) \bigr)
          \bigl( S_{j}(t') \otimes E_{j}(t') \bigr) \nonumber \\
          -& \bigl(  S_{j}(t') \otimes E_{j}(t')\bigr)
          \bigl(  \rho^{\textrm{int}}_{\textrm{s}}(t') \otimes
   \rho^{\textrm{int}}_{\textrm{e}}(0)\bigr)
          \bigl(  S_{i}(t) \otimes E_{i}(t)\bigr) \nonumber \\
          +& \bigl(  \rho^{\textrm{int}}_{\textrm{s}}(t') \otimes
   \rho^{\textrm{int}}_{\textrm{e}}(0)\bigr)
          \bigl(  S_{j}(t') \otimes E_{j}(t')\bigr)
          \bigl(  S_{i}(t) \otimes E_{i}(t)\bigr)
       \Bigr) \biggr)  \nonumber \\
       = -\frac{1}{\hbar^{2}} \Tr_{\textrm{e}} \biggl(  \int_{0}^{t} dt' \sum^{N}_{i,j}
          \Bigl(
             &\bigl( S_{i}(t) S_{j}(t') \rho^{\textrm{int}}_{\textrm{s}}(t')
          \otimes E_{i}(t) E_{j}(t') \rho^{\textrm{int}}_{\textrm{e}}(0) \bigr)
          \label{eq:born_equation_condensed_algebraic_form} \\
          -& \bigl( S_{i}(t) \rho^{\textrm{int}}_{\textrm{s}}(t') S_{j}(t')
         \otimes E_{i}(t) \rho^{\textrm{int}}_{\textrm{e}}(0) E_{j}(t') \bigr)
         \nonumber \\
         -& \bigl( S_{j}(t')\rho^{\textrm{int}}_{\textrm{s}}(t') S_{i}(t)
         \otimes E_{j}(t') \rho^{\textrm{int}}_{\textrm{e}}(0) E_{i}(t) \bigr)
         \nonumber \\
         +& \bigl( \rho^{\textrm{int}}_{\textrm{s}}(t') S_{j}(t') S_{i}(t)
         \otimes \rho^{\textrm{int}}_{\textrm{e}}(0) E_{j}(t') E_{i}(t) \bigr)
          \Bigr) \biggr) \nonumber \\
       = -\frac{1}{\hbar^{2}} \biggl( \int_{0}^{t} dt' \sum^{N}_{i,j}
          \Bigl(
             & F_{i,j}(t,t') S_{i}(t) S_{j}(t') \rho^{\textrm{int}}_{\textrm{s}}(t')
          - F_{j,i}(t',t) S_{i}(t) \rho^{\textrm{int}}_{\textrm{s}}(t') S_{j}(t')
          \label{eq:born_equation_cyclic_property_of_trace} \\
          -& F_{i,j}(t,t') S_{j}(t')\rho^{\textrm{int}}_{\textrm{s}}(t') S_{i}(t)
          + F_{j,i}(t't) \rho^{\textrm{int}}_{\textrm{s}}(t') S_{j}(t') S_{i}(t)
          \Bigr) \biggr) \nonumber \\
       = -\frac{1}{\hbar^{2}} \biggl( \int_{0}^{t} dt' \sum^{N}_{i,j}
          & F_{i,j}(t,t') \Bigl(
             S_{i}(t) S_{j}(t') \rho^{\textrm{int}}_{\textrm{s}}(t')
          - S_{j}(t')\rho^{\textrm{int}}_{\textrm{s}}(t') S_{i}(t)
       \Bigr) \\
       + & F_{j,i}(t',t) \Bigr(
          \rho^{\textrm{int}}_{\textrm{s}}(t') S_{j}(t') S_{i}(t)
          - S_{i}(t) \rho^{\textrm{int}}_{\textrm{s}}(t') S_{j}(t')
            \Bigr) \biggr) \nonumber
\end{align}
Note that the transition from
Eq.~\ref{eq:born_equation_condensed_algebraic_form} to
Eq.~\ref{eq:born_equation_cyclic_property_of_trace} relies on the cyclic
property of the trace. Now, some assumptions are going to be made about these
autocorrelation functions $ F_{i,j}(t,t') $. First, it will be assumed that the
environment is stationary. This means that all of the statistical properties of
the system depend not on the particular times $ t_{1}, t_{2}, \ldots t_{n} $ of
interest but only on the differences of time under consideration. So, $
F_{i,j}(t,t') = F_{i,j}(t'-t) $. This is a reasonable assumption often made of
noisy systems.  Here, the environment stores all of the noisy information and,
as such, can reasonably be considered to have this property. Another assumption
that will be made is that the noise is relatively uncorrelated. That is, $
F_{i,j}(t'-t) \approx 0 $ for $ \left| t' \right| \gg \left| t \right| $. This
is the Markovian assumption (the environment has no ``memory''). It may prove to
be an invalid assumption. As always with science, one has to compare theoretical results
with experimentation to confirm the theory.

The assumption that the system is memoryless implies the two following important
statements. First, the state's dynamics, which are assumed to change over intervals greater
than the memory time of the environment, are relatively fixed over $ t' \approx
t $. This allows for the replacement of $ t' \rightarrow t $ as the argument for
$ \rho_{\textrm{s}}^{\textrm{int}} $ in the last expression. Second, the
autocorrelation functions are assumed to be negligible for all $ t' \ll t $.
Since the integral runs from 0 up to time $ t $ there is no harm in extending
the lower limit of the integral to $ -\infty $. The integrand contributes little
over that domain. Using these ideas to rewrite the last expression we have
\begin{align}
   \frac{d}{dt} \rho_{\textrm{s}}^{\textrm{int}}(t)
       = -\frac{1}{\hbar^{2}} \biggl( \int_{-\infty}^{t} dt' \sum^{N}_{i,j}
          & F_{i,j}(t'-t) \Bigl(
             S_{i}(t) S_{j}(t') \rho^{\textrm{int}}_{\textrm{s}}(t)
          - S_{j}(t')\rho^{\textrm{int}}_{\textrm{s}}(t) S_{i}(t)
       \Bigr) \\
       +  & F_{j,i}(t-t') \Bigr(
          \rho^{\textrm{int}}_{\textrm{s}}(t) S_{j}(t') S_{i}(t)
          - S_{i}(t) \rho^{\textrm{int}}_{\textrm{s}}(t) S_{j}(t')
            \Bigr) \biggr) \nonumber
\end{align}
It will be useful to express everything in terms of the time difference $ t'-t
\equiv \tau $. Note that, $ dt' = d\tau $. Now,
\begin{align}
   \frac{d}{dt} \rho_{\textrm{s}}^{\textrm{int}}(t)
       = -\frac{1}{\hbar^{2}} \biggl( \int_{-\infty}^{0} d\tau \sum^{N}_{i,j}
          & F_{i,j}(\tau) \Bigl(
             S_{i}(t) S_{j}(\tau+t) \rho^{\textrm{int}}_{\textrm{s}}(t)
          - S_{j}(\tau+t)\rho^{\textrm{int}}_{\textrm{s}}(t) S_{i}(t)
       \Bigr) \label{eq:born_markov_master_equation_in_interaction_picture} \\
       +  & F_{j,i}(-\tau) \Bigr(
          \rho^{\textrm{int}}_{\textrm{s}}(t) S_{j}(\tau+t) S_{i}(t)
          - S_{i}(t) \rho^{\textrm{int}}_{\textrm{s}}(t) S_{j}(\tau+t)
            \Bigr) \biggr) \nonumber
         \end{align}
This is exactly what we wanted! We have an expression for the time rate of
change of our state in terms of operators of the Hamiltonian we would be
provided (time advanced in the interaction picture). Crucially, the above result
does not depend on the combined state of the system and environment and, also,
the above result is no longer in integrodifferential form. The history of the
state is no longer being integrated over. However, we have an expression for the
state in the interaction picture. It would be nice to have this in the
Schr\"{o}dinger picture. As derived in the appendix, we will now use the
transformation of
Eq.~\ref{eq:time_evolution_of_schrodinger_state_in_terms_of_interaction_state}.
Performing the transformation on the expression in
Eq.~\ref{eq:born_markov_master_equation_in_interaction_picture} has the effect
of shifting the arguments back by $ t $. The interaction transformation is being
undone by an amount $ t $. So, the final form of the Born-Markov master equation
is
\begin{align}
   \frac{d}{dt} \rho_{\textrm{s}}(t)
   = - \frac{i}{\hbar} \left[ H_{\textrm{s}}, \rho_{\textrm{s}}(t) \right] -
   \frac{1}{\hbar^{2}} \biggl( \int_{-\infty}^{0} d\tau \sum^{N}_{i,j}
          & F_{i,j}(\tau) \Bigl(
          S_{i} S_{j}(\tau) \rho_{\textrm{s}}(t)
          - S_{j}(\tau) \rho_{\textrm{s}}(t) S_{i}
       \Bigr) \label{eq:born_markov_master_equation_in_schrodinger_picture} \\
       +  & F_{j,i}(-\tau) \Bigr(
       \rho_{\textrm{s}}(t) S_{j}(\tau) S_{i}
          - S_{i} \rho_{\textrm{s}}(t) S_{j}(\tau)
            \Bigr) \biggr) \nonumber \enskip.
\end{align}

\section{Spin-Boson Master Equation}
\label{sec:spin_boson_master_equation}
We have now derived a master equation for a system weakly coupled to an
environment. The validity of this master equation holds under some conditions. One of those
conditions was that of weak coupling or, alternatively said, of environment persistence (the absence
of environmental dynamics). This allowed us to express the state of the system
as the tensor product of the initial state of the environment and the (dynamic)
state of the system. Let us imagine, now, that we are presented with the
Hamiltonian of a qubit (a transmon; refer to \cite{Bishop2010}) coupled to a
bath of harmonic oscillators in a thermal state (a thermal bath, like a noisy
resistor). This can be written as
\begin{equation}
   H_{\textrm{transmon, bath}} = \hbar \omega_{q} \sigma_{z} \otimes
   \mathds{1}_{\textrm{bath}} + \hbar \mathds{1}_{\textrm{qubit}} \otimes
   \sum^{N}_{i=1} \omega_{i} a_{i}^{\dagger}a_{i}
   + \hbar \sum^{N}_{i=1} g_{i} \sigma_{x} \otimes \bigl(a_{i} +
   a_{i}^{\dagger}\bigr) \enskip.
\end{equation}
The master equation derived in
Eq.~\ref{eq:born_markov_master_equation_in_schrodinger_picture} required
knowledge of the autocorrelation function of the environment operators $
F_{i,j}(\tau) $. In addition to that, it required knowledge of the
time-evolution of the system operators. We will establish both of those
quantities, now. We will begin with the autocorrelation function of the
environment operators.

The autocorrelation of the environment operators only involves the
environment operators used to describe the interaction part of the Hamiltonian.
These are $ g_{i} \left( a_{i} + a_{i}^{\dagger} \right) $. They were calculated
as an average over the measurable quantities and were assumed to rely only on
the operators used to represent the environment part of the interaction part of
the Hamiltonian and the (initial) state of the environment. Thus, calculating $
F_{i,j}(t,t') $ looks like
\begin{equation}
   F_{i,j}(t,t') = \hbar^{2} g_{i} g_{j} \Tr_{\textrm{e}}\left( \left( a_{i}(t) + a_{i}^{\dagger}(t)
   \right) \left( a_{j}(t') + a_{j}^{\dagger}(t') \right) \rho_{\textrm{e}}\right) \enskip.
\end{equation}
The time evolution of the operators is calculated as $ E_{i}(t) =
e^{\frac{i}{\hbar} \left( \mathds{1}_{\textrm{s}} \otimes H_{\textrm{e}}
\right) t} E_{i} e^{-\frac{i}{\hbar} \left( \mathds{1}_{\textrm{s}} \otimes H_{\textrm{e}}
\right) t} $. Substituting in the operators for this case
\begin{equation}
   a_{i}(t) = e^{\frac{i}{\hbar} \hbar \omega_{i} \sum^{N}_{i=1} a_{i}^{\dagger} a_{i}  t}
a_{i} e^{-\frac{i}{\hbar} \hbar \omega_{i} \sum^{N}_{i=1} a_{i}^{\dagger} a_{i} t} \enskip.
\end{equation}
But, the above is just the evolution of the annihilation operator in the
Heisenberg picture. So, we can say that
\begin{equation}
   \frac{d}{dt} a_{i}(t) = \frac{i}{\hbar} \left[ H_{\textrm{e}} , a_{i} \right]
\end{equation}
Since different modes commute ($ \left[ a_{i}, a_{j} \right] = 0 $), we can say
\begin{align}
   \frac{d}{dt} a_{i}(t) &= \frac{i}{\hbar} \omega_{i} \left[ a_{i}^{\dagger} a_{i}  , a_{i}
\right] \\
   &= \frac{i}{\hbar} \omega_{i} \left( a_{i}^{\dagger} a_{i} a_{i} - a_{i}
a_{i}^{\dagger} a_{i} \right) \enskip.
\intertext{Using the fact that $ \left[ a_{i}, a_{i}^{\dagger} = 1 \right] $,}
&= \frac{i}{\hbar} \omega_{i} \left( a_{i}^{\dagger} a_{i} a_{i} - \left( 1 +
a_{i}^{\dagger} a_{i} \right) a_{i} \right) \\
&= - \frac{i}{\hbar} \omega_{i} a_{i} \enskip.
\end{align}
Solving this first order differential equation yields
\begin{equation}
   a_{i}(t) = a_{i} e^{-i \omega_{i} t} \enskip.
\end{equation}
Performing a similar analysis for $ a_{i}^{\dagger}(t) $ yields
\begin{equation}
   a_{i}^{\dagger}(t) = a_{i}e^{i \omega_{i}t} \enskip.
\end{equation}
Note that the fact that $ \left[ a_{i}, a_{j} \right] $ commute also implies
that $ F_{i,j}(t,t') = 0 $ for all $ i \ne j $. We are now prepared to solve for
$ F_{i,i}(t,t') $ assuming that $ \rho_{\textrm{e}} $ is given. Assume that $
\rho_{\textrm{e}} $ is a thermal state. It can be shown that the value of $
F_{i,j}(t,t') $ under these conditions is exactly
\begin{equation}
   F_{i,i}(t,t') = \hbar^2 g_{i}^{2} \Bigl(\coth\left(\frac{\hbar \omega_{i}}{2 k_{b} T}\right)
      \cos\left(\omega_{i} \left( t' -
   t \right)\right) - i \sin\left(\omega_{i} \left( t'-t \right)\right)\Bigr)
\end{equation}
This is exactly what we need. However, this form has the disadvantage of
requiring knowledge of $ g_{i} $, the system-environment coupling parameters. A
better approach to determining the autocorrelation functions is to determine, in
the context of an quantized electrodynamics situation, what the thermal noise
generated by a thermal state is. Once the spectral density is determined in the
context of quantum electrodynamics, the result can be transformed, using the
Wiener-Khintchine theorem, to determine the autocorrelation function, in time.
Now, it has been shown (see, for example, \cite{Callen1951}) that the quantum
spectral density of a noisy resistor is
\begin{equation}
   S_{V_{R}}(\omega) = \frac{4 R(\omega,T) \hbar \omega}{e^{\hbar \omega / k_{B} T
   } - 1} \enskip.
\end{equation}
This is very similar to the expression that was derived earlier. However, it
lacks an integral over temperature because the system is assumed to be at a
given temperature in this form.

It would be great to relate this to the autocorrelation function just derived.
However, this spectral density, like the one that I derived in the classical
case, also diverges without bound with increasing $\omega$. Thus, a proper
Fourier integral does not exist. However, if an assumption is made that only the
frequencies that are ``close'' to the frequency of the qubit affect the qubit
dynamics, then the Fourier integral can be windowed and a result can be
obtained. Concretely,
\begin{align}
   \phi_{V_{R}}(\tau) &= \frac{1}{2 \pi} \int_{-\omega_{b}}^{-\omega_{a}} S_{V_{R}}(\omega) e^{i
   \omega \tau} d\omega + \frac{1}{2 \pi} \int_{\omega_{a}}^{\omega_{b}} S_{V_{R}}(\omega) e^{i
   \omega \tau} d\omega \enskip.
   \intertext{Changing the sign of $ \omega $ in the first integral,}
   &= -\frac{1}{2 \pi} \int_{\omega_{b}}^{\omega_{a}} S_{V_{R}}(-\omega) e^{- i
   \omega \tau} d\omega + \frac{1}{2 \pi} \int_{\omega_{a}}^{\omega_{b}} S_{V_{R}}(\omega) e^{i
   \omega \tau} d\omega \\
   &= \frac{ \hbar }{\pi} \int_{\omega_{b}}^{\omega_{a}} \frac{ R(\omega) \omega
   e^{-i \omega \tau}}{e^{-\hbar \omega /
   k_{B} T}-1} d\omega + \frac{ \hbar }{\pi} \int_{\omega_{a}}^{\omega_{b}} \frac{R(\omega)  \omega e^{i \omega \tau}}{e^{\hbar \omega /
   k_{B} T} - 1} d\omega \\
   &= -\frac{ \hbar }{\pi} \int_{\omega_{a}}^{\omega_{b}} \frac{R(\omega)
   \omega e^{-i \omega \tau}}{ e^{ -\hbar \omega /
   k_{B} T} - 1} d\omega + \frac{\hbar }{\pi} \int_{\omega_{a}}^{\omega_{b}} \frac{
   R(\omega) \omega e^{i \omega \tau}}{e^{\hbar \omega /
   k_{B} T}-1} d\omega \enskip.
\end{align}
Unfortunately, this integral does not have a nice closed-form solution. However,
this integral can be numerically approximated for small values of $ \tau $. In
fact, depending on the temperature, $ T $, $ \frac{\hbar \omega}{e^{\hbar
\omega / k_{B} T } - 1} \approx k_{B} T $. This is the case at room temperature
for frequencies below $ \approx \SI{1}{\tera\hertz} $. However, at
\SI{10}{\milli\kelvin} (the coldest temperature in a state-of-the-art dilution
refrigerator) the approximation breaks validity at $ \approx \SI{1}{\mega\hertz}
$. Accounting for temperature variations in the resistive network which
generates the noise, the final expression for the windowed spectral density can
be expressed as
\begin{align}
   \phi_{V_{R}}(\tau) = F(\tau) = -&\frac{ \hbar }{\pi} \int_{\omega_{a}}^{\omega_{b}}
   \omega e^{-i \omega \tau}
   \left(\frac{1}{T_{\textrm{high}}- T_{\textrm{low}}}\int_{T_{\textrm{low}}}^{T_{\textrm{high}}}
   \frac{R(\omega,T) }{ e^{ -\hbar \omega / k_{B} T} - 1} dT\right)
   d\omega \\
   + & \frac{\hbar }{\pi} \int_{\omega_{a}}^{\omega_{b}} \omega e^{i \omega \tau}
   \left(\frac{1}{T_{\textrm{high}}- T_{\textrm{low}}}\int_{T_{\textrm{low}}}^{T_{\textrm{high}}}
   \frac{R(\omega,T) }{ e^{ \hbar \omega / k_{B} T} - 1} dT\right)
    d\omega
\end{align}

So, were a simulation of the noise delivered to a two-level system to be desired, it
is now only necessary to evaluate this integral over the appropriate $ \omega $
and $ T $. This should be evaluated for $ \tau $ smaller than the scale of the system
dynamics. Then, all that needs to be done is to evolve the system operators
under the system Hamiltonian. In this case (of the transmon) there is only one system
operator, the Pauli-Z spin operator, which does not time evolve according to its
own Hamiltonian. Thus, $ \sigma_{z}(\tau) = \sigma_{z} $. Since $
\sigma_{z} \sigma_{z} = \mathds{1}_{2} $, the Born-Markov master equation which
governs a transmon's decoherence and dissipation is given by the following
expression:
\begin{equation}
   \frac{d}{dt} \rho_{\textrm{qubit}}(t) = -i \omega_{q} \left[ \sigma_{z} ,
      \rho_{s}(t) \right] - \frac{1}{\hbar^{2}}
      \int_{-\infty}^{0} \biggl( F(\tau) \Bigl(
         \rho_{s}(t) - \sigma_{z} \rho_{s} \sigma_{z} \Bigr)
         +
         F(-\tau) \Bigl( \rho_{s}(t) - \sigma_{z} \rho_{s}(t) \sigma_{z}(t)
      \Bigr) \biggr) d\tau \enskip.
\end{equation}
This concludes the analysis of the dynamics of a  two-level system exposed to a
thermal bath.

\section{Acknowledgements}
\label{sec:acknowledgements}
This work could not have been accomplished without the ``live'' interaction with
the following three people: T.C. Fraser, Hammam Qassim and Arnaud
Carignan-Dugas.

T.C. Fraser took much of his time to develop and walk me through the noise
generated by a resistor anchored to two thermal baths. He also spent much of his
time reading and understanding Timothy Goddard's work ``Quantum Decoherence and
the Spin-Boson Model''. His help in understanding Mr. Goddard's work was
invaluable to the creation of this work.

Hammam Qassim was helpful in (very painstakingly) explaining the block
decomposition of an operator living in a tensor-product space.

Arnaud Carignan-Dugas was instrumental in understanding the time derivative of
the tensor product of two (time-varying) operators.

The author would also like to thank Timothy Goddard for his time spent in
writing a very pedagogical introduction to quantum decoherence and dissipation.

\appendix
\renewcommand{\theequation}{\Alph{section}.\arabic{equation}}
\section{Classical Thermal Noise of a Resistor at One Temperature}
\label{sec:classical_thermal_noise_of_a_resistor_at_one_temperature_}

Considering this expression in the limit that $ T_{2} \rightarrow T_{1} $:
\begin{align}
   S_{V_{R}}(f) &= \lim_{T_{2} \rightarrow T_{1}} \frac{4 h f}{T_{2} - T_{1}} \int_{T_{1}}^{T_{2}} R(T) \left(
   \exp\left( \frac{hf}{k_{B}T}\right) - 1 \right)^{-1} dT \\
   & = \lim_{T_{2} \rightarrow T_{1}} \frac{4 h f R(T_{1})}{T_{2} - T_{1}}
   \int_{T_{1}}^{T_{2}} \left( \exp\left( \frac{hf}{k_{B}T}\right) - 1
   \right)^{-1} dT
   \intertext{For $ h f \ll k_{B} T $ the integrand, $ \left(\exp(h f / k_{B} T)
-1\right)^{-1} $ is approximately $ \frac{k_{B} T }{h f} $. In this regime, we
are only integrating over $ T $ and the result is,}
   &= \lim_{T_{2} \rightarrow T_{1}} \frac{4 k_{B} R(T_{1})}{T_{2} - T_{1}}
   \frac{T_{2}^{2} - T_{1}^{2}}{2} \\
   &= \lim_{T_{2} \rightarrow T_{1}} 4 k_{B} R(T_{1}) \frac{T_{2}+T_{1}}{2} \\
   &= 4 k_{B} R(T_{1}) T_{1}
   \enskip.
\end{align}
This is the classical noise temperature of a resistor at temperature $ T_{1} $.

\section{Derivation of Time Evolution of System State in Schr\"{o}dinger Picture}
\label{sec:derivation_of_time_evolution_of_system_state_in_schr"_o_dinger_picture}
The time evolution of a state is determined by the Hamiltonian that governs it.
For a closed system, a state $ \rho(t) $ evolves according to its Hamiltonian $
H $ according to the Schr\"{o}dinger equation
\[
   \frac{d}{dt} \rho(t) = - \frac{i}{\hbar} \left[ H, \rho(t) \right] \enskip.
\]
If the Hamiltonian has the form $ H_{A} \otimes \mathds{1}_{B} + \mathds{1}_{A}
\otimes H_{B} + H_{AB} $ and the time evolution of $ \rho_{A}(t) =
\Tr_{B}\left(\rho(t)\right) $ is desired then it can be found as
\begin{align}
   \frac{d}{dt} \rho_{A}(t) = \frac{d}{dt} \Tr_{B}\left(\rho(t)\right)
   &= \Tr_{B}\left( \frac{d}{dt} \rho(t) \right) \\
   &= -\frac{i}{\hbar} \Tr_{B} \left( \left[ H, \rho(t) \right] \right) \\
   &= -\frac{i}{\hbar} \Tr_{B} \left( \left[  H_{A} \otimes \mathds{1}_{B} + \mathds{1}_{A}
\otimes H_{B} + H_{AB} , \rho(t) \right]\right) \\
   &= -\frac{i}{\hbar} \Tr_{B} \left( \left[  H_{A} \otimes \mathds{1}_{B} ,
\rho(t) \right] + \left[ \mathds{1}_{A}
\otimes H_{B} , \rho(t) \right] + \left[ H_{AB} , \rho(t) \right]\right)
\enskip.
\end{align}
Now, in the case where $ \rho(t) $ can not be expressed in tensor
product form this is all that can be said. However, if $ \rho(t) $ can be
expressed as $ \rho_{A}(t) \otimes \rho_{B}(t) $ then we have
\begin{equation}
= -\frac{i}{\hbar} \Tr_{B} \Bigl( \bigl[  H_{A} \otimes \mathds{1}_{B} ,
 \rho_{A}(t) \otimes \rho_{B}(t)  \bigr]
 + \bigl[ \mathds{1}_{A} \otimes H_{B} , \rho_{A}(t) \otimes \rho_{B}(t)
 \bigr]
+ \bigl[ H_{AB} ,  \rho_{A}(t) \otimes \rho_{B}(t) \bigr] \Bigr) \enskip.
\end{equation}
Using the fact that the trace of any density matrix is 1, we can simplify the
first term above as
\begin{equation}
= -\frac{i}{\hbar} \bigl[  H_{A} , \rho_{A}(t) \bigr]
 -\frac{i}{\hbar} \Tr_{B} \Bigl( \bigl[ \mathds{1}_{A} \otimes H_{B} , \rho_{A}(t) \otimes \rho_{B}(t)
 \bigr]
+ \bigl[ H_{AB} ,  \rho_{A}(t) \otimes \rho_{B}(t) \bigr] \Bigr) \enskip.
\end{equation}
Factoring out a $ \rho_{A} $ from the second term, we have
\begin{equation}
= -\frac{i}{\hbar} \bigl[  H_{A} , \rho_{A}(t) \bigr]
-\frac{i}{\hbar} \rho_{A}(t) \otimes \Tr_{B} \Bigl( \bigl[ H_{B} , \rho_{B}(t)
      \label{eq:evolution_of_density_matrix_in_interaction_picture_intermediate_form}
\bigr] \Bigr)
- \frac{i}{\hbar} \Tr_{B} \Bigl( \bigl[ H_{AB} ,  \rho_{A}(t) \otimes \rho_{B}(t) \bigr] \Bigr) \enskip.
\end{equation}
Now, again, this is all we can say unless we admit more assumptions to shape our
mathematics. So, assume that $ \left[H_{B}, \rho_{B}(t)\right] = 0 $ (a
statement that $ \rho_{B}(t) $ doesn't evolve under the action of $ H_{B} $;
it is a stationary state of $ H_{B} $). This eliminates the second term of
the above. The last term represents the evolution of the combined state
subject to the interaction Hamiltonian. We know from
Eq.~\ref{eq:interaction_picture_state_commutator} that the commutator can be
expressed as
\begin{equation}
   \left[ H_{AB}, \rho_{A}(t) \otimes \rho_{B}(t) \right] = i \hbar e^{- i (H_{A} \otimes \mathds{1}_{B} + \mathds{1}_{A}
      \otimes H_{B}) t } \left(\frac{d}{dt} \rho^{\textrm{int}}_{A \otimes B}(t) \right) e^{i (H_{A} \otimes \mathds{1}_{B} + \mathds{1}_{A}
   \otimes H_{B}) t }
\end{equation}
If we assume that the combined state in the interaction picture is separable as
$ \rho_{A \otimes B}^{\textrm{int}}(t) =
\rho_{A}^{\textrm{int}}(t) \otimes \rho_{B}^{\textrm{int}}(t) $ then, by product
rule of the tensor product, we have
\begin{equation}
   \left[ H_{AB}, \rho_{A}(t) \otimes \rho_{B}(t) \right] = i \hbar e^{- i (H_{A} \otimes \mathds{1}_{B} + \mathds{1}_{A}
      \otimes H_{B}) t } \left(\frac{d}{dt} \rho^{\textrm{int}}_{A}(t) \otimes
      \rho_{B}^{\textrm{int}}(t) + \rho^{\textrm{int}}_{A}(t) \otimes \frac{d}{dt}
      \rho_{B}^{\textrm{int}}(t) \right) e^{i (H_{A} \otimes \mathds{1}_{B} + \mathds{1}_{A}
   \otimes H_{B}) t } \enskip.
\end{equation}
By commutativity of the Hamiltonians $ H_{A} \otimes \mathds{1}_{B} $ and $ \mathds{1}_{A}
   \otimes H_{B} $ those can be expressed as
\begin{align*}
   \left[ H_{AB}, \rho_{A}(t) \otimes \rho_{B}(t) \right] = i \hbar
   & e^{- i (H_{A} \otimes \mathds{1}_{B}) t }
   \frac{d}{dt} \rho^{\textrm{int}}_{A}(t)
   e^{i (H_{A} \otimes \mathds{1}_{B}) t }
   \otimes
   e^{-i \left( \mathds{1}_{A} \otimes H_{B} \right) t }
      \rho_{B}^{\textrm{int}}(t)
      e^{i \left( \mathds{1}_{A} \otimes H_{B} \right) t }
      \\
      + i \hbar & e^{- i (H_{A} \otimes \mathds{1}_{B}) t }
      \rho^{\textrm{int}}_{A}(t)
      e^{ i (H_{A} \otimes \mathds{1}_{B}) t }
      \otimes
      e^{-i \left( \mathds{1}_{A} \otimes H_{B} \right) t }
      \frac{d}{dt} \rho_{B}^{\textrm{int}}(t)
      e^{i \left( \mathds{1}_{A} \otimes H_{B} \right) t }
      \enskip.
\end{align*}
Now, using one more assumption, if one of those states, say $
\rho^{\textrm{int}}_{B}(t) $ is independent of time, then its time derivative
will be zero. This is the case for the environmental bath. It is assumed to be
stationary in time. Thus, only the first term of the previous expression would
survive. Taking the trace of the first term over space $ B $ would yield 1 since
the trace of a density matrix is 1 and, as can be seen by the cyclic property of
the trace, the exponentials have no effect on the trace. So, we could now write
\begin{equation}
   \Tr_{B} \bigl( \left[ H_{AB}, \rho_{A}(t) \otimes \rho_{B}(t) \right] \bigr)
   =
   i \hbar
   e^{- i (H_{A} \otimes \mathds{1}_{B}) t }
   \frac{d}{dt} \rho^{\textrm{int}}_{A}(t)
   e^{i (H_{A} \otimes \mathds{1}_{B}) t } \enskip .
\end{equation}
Substituting this expression in
Eq.~\ref{eq:evolution_of_density_matrix_in_interaction_picture_intermediate_form}
(and throwing away the second term in accordance with the previous expression)
\begin{equation}
\frac{d}{dt} \rho_{A}(t) = -\frac{i}{\hbar} \bigl[  H_{A} , \rho_{A}(t) \bigr]
+ e^{- i (H_{A} \otimes \mathds{1}_{B}) t }
   \frac{d}{dt} \rho^{\textrm{int}}_{A}(t)
   e^{i (H_{A} \otimes \mathds{1}_{B}) t } \enskip.
   \label{eq:time_evolution_of_schrodinger_state_in_terms_of_interaction_state}
\end{equation}
So, armed with the knowledge of the reduced state in the interaction picture
allow we can write the time rate of change of the reduced state in the
Schr\"{o}dinger picture.

\bibliographystyle{alpha}
\bibliography{bib}
\end{document}
