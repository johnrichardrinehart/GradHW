\section*{Future Work}
\addcontentsline{toc}{section}{Future Work}

There are a number of problems with the code base that was used to replicate the
circuit studied in reference \cite{opt} (this circuit was also used as an example
in \cite{wcd}). The first main problem with the code is the reliance on fmincon
, a MATLAB\textsuperscript{\textregistered} function that is able to solve
nonlinear optimization problems with applied constraints. In this case, the
constraint of decreasing resistance with increasing frequency was applied.

However, nonlinear optimization is subject to the unfortunate problem that
solutions can arise that are not, in general, the best solution. The
optimization procedures, in general, work by determining the best way in which
to ``tweak''/adjust the variables under consideration such as to minimize the
function at hand. The solution relies, then, heavily on the initial conditions
given to the solver. A large improvement to the algorithm would be to express
the problem as that of a linear optimization problem. Then, the sensitivity to
initial conditions is gone and a global minimum is guaranteed.

Next, the Gewertz method that is used in conjunction with Darlington Synthesis
is undesirable. The amount of steps that are undertaken to obtain $Z(s)$ from
$R(-s^2)$ are so many that the chance for numerical or implementation error are
great. A better way to approach the problem would be to obtain $X(s)$ directly
from the piecewise $\tilde{R}(s)$ obtained from the optimization procedure.
Then, once $X(s)$ is known it is easy to construct $Z(s) = R(s) + X(s)$.
Then, a rational function fit of $Z(s)$ can be obtained (to arbitrary numerator
and denominator precision).

Furthermore, this tool, right now, does not accommodate the ``double matching``
problem where both the source and the load are fixed and the best network is to
be obtained. In this tool, the load is fixed and the rest of the network is
allowed to vary.

Finally, a great edition to this tool would be to increase the level of
interactivity and usability. Right now, the script Example9\_2\_1.m uses the
supplied functions in a very particular way that make it easy to adjust the
load, the number of breakpoints, etc. However, it should be possible to ship the
entire SRFT library without the dependence on such a particular script.
