\section*{Conclusion}
\addcontentsline{toc}{section}{Conclusion}

In summary, a case study from the literature was studied in an attempt to write
a SRFT tool in MATLAB\textsuperscript{\textregistered}. A beta version of such a
tool was produced and is under active revision. Currently, this code is
available at \url{https://github.com/fuzzybear3965/SRFT}. A fork of this
code base will be written in Julia and in Java. This will make the adoption of
such a tool much easier. The SRFT allows for optimal filter design which
utilizes components to the best degree possible and attains on optimal design,
which is too often difficult to achieve elsewhere in microwave engineering. 

One further note, the entire assumption of the SRFT is that the system is
linear. Based on this assumption, the principles of elementary circuit theory and
the relations between the real and imaginary parts of the transfer function
could be used to obtain the results given. However, in many cases, the
circuit or system under consideration behaves nonlinearly. In this case, the
theory given is no longer applicable. Another goal of this project, then, is to
extend the usefulness of SRFT into the domain of nonlinear signals. That has
great applicability to the design world and should be of great interest to
industry and to academia.
