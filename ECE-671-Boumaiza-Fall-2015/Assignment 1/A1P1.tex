\section{Power and Waves}
\subsection{Available Power}
The maximum power available from the source is determined by the condition where
the source is conjugately matched to the load (this is when $Z_s = Z_{in}^*$).
Thus, a naive analysis of the power available from the source can be calculated
as follows.

\subsubsection{Available Power: Naive Approach}
The source impedance is specified as 100~$\Omega$. Thus, a conjugately matched
input impedance would be given by (100~$\Omega$)$^*$. This load will divide the
voltage of the source. Thus, the power available
from the source is given by:
\begin{align*}
    P_{avl} &= \Re(V_{in}I_{in}^*) \\
            &= \Re \left(\frac{|V_{in}|^2}{Z_{in}^*}\right) \\
            &= \left| V_{in} \right|^2 \Re \left( \frac{1}{Z_s} \right) \\
            &= \frac{\left| V_{in} \right|^2}{|Z_s|^2} \Re \left( Z_s^* \right) \\
            &= \frac{\left| V_{in} \right|^2}{|Z_s|^2} \Re \left( Z_s \right)
\end{align*}

But, $V_{in} = V_s \frac{Z_s}{Z_s+Z_{in}}$. Using this:

\begin{align*}
    P_{avl} &= \frac{\left| V_s \right|^2}{\left| Z_s \right|^2} \frac{\left|
Z_s\right|^2}{\left| Z_s + Z_{in}\right|} \Re \left( Z_s \right) \\
&=\frac{\left| V_s \right|^2}{\left| Z_s + Z_{in} \right|^2} \Re \left( Z_s
\right) \\
&= \frac{\left| V_s \right|^2}{4 \left( \Re \left( Z_s \right) \right)^2} \Re
\left( Z_s \right) \\
&= \frac{\left| V_s^2 \right|}{4 \Re \left( Z_s \right)}
\end{align*}

Now, this makes sense and we could use this to obtain an expression for the
power available from the source. Let's consider that and obtain an answer, for
this problem, for the power available from the source. The derivation above
assumed that the voltages and currents were specified as RMS quantities. If they
are provided as peak values (like in this problem) we need to scale both the
current and voltage by $2^{-1/2}$ placing a $\frac{1}{2}$ in front of the
previous expression. Given that the source impedance is specified as a real
\SI{100}{\ohm} we can solve for the power available from the source:

\[ 
        P_{avl} = \frac{\left| \SI{20}{\volt} \right|^2}{2 \cdot 4 \cdot \SI{100}{\ohm}}
        = \SI{500}{\milli\watt}
\]

This is the naive solution. 

\subsubsection{Available Power: Alternative Solution}
It makes sense. Another way to solve this problem is to consider the voltage
that would be incident on the network $V_1^+$ and realize that the power
transmitted from the source down a transmission line of characteristic impedance $Z_{c_1}$ is:

\begin{equation*}
    P_{trans} = \Re \left( V_1 I_1^* \right) = \Re \left( \left(  V_1^+ + V_1^- \right) \frac{\left( V_1^+ - V_1^-
        \right)^*}{Z_{c_1}^*} \right) 
\end{equation*}

Then, realizing that the wave reflected off of the interface between the source
impedance, $Z_s$, and the input impedance $Z_{in}$ is

\[ 
        V_1^- = \Gamma_{in} V_1^+ = \frac{Z_{in} - Z_{s}}{Z_{in} + Z_{s}} V_1^+
\]

we can rewrite the previous expression:

\begin{equation}
    P_{trans}= \left| V_1^+ \right|^2 \Re \left( \left( 1  + \Gamma_{in} \right)
        \frac{\left(1 - \Gamma_{in}^* \right)}{Z_{c_1}^*} \right) = \left| V_1^+
        \right|^2 \left( 1-\left| \Gamma_{in} \right|^2 \right) \frac{\Re \left(
    Z_{c_1} \right)}{\left| Z_{c_1} \right|^2}
\end{equation}

If we can find an expression for $V_1^+(V_s)$ then we have something with which
to compare our naive solution. This is not too hard. The voltage across the
input of the network can be expressed through a voltage divider as:

\[ 
        V_{in} = V_1^+ + V_1^- = V_1^+ (1 + \Gamma_{in}) = V_s \frac{Z_{in}}{Z_s
        + Z_{in} }
\]

Thus,

\[ 
        V_1^+ = V_s\frac{1}{1+ \Gamma_{in}} \frac{Z_{in}}{Z_s + Z_{in}}
\]

Finally, we can write that the power delivered, in general, from a source with
complex impedance $Z_c$ to a network with impedance $Z_{in}$ is:

\[ 
        P_{del} = \left| V_s \right|^2 
        \frac{\left| Z_{in}\right|^2}{\left| Z_s + Z_{in} \right|^2} 
        \frac{ 1 - \left| \Gamma_{in} \right|^2}{\left| 1+\Gamma_{in} \right|^2}
        \frac{\Re \left( Z_{c_1} \right)}{\left| Z_{c_1} \right|^2}
\]

The case in which the maximum amount of power is delivered to the network is
when the network is conjugately matched to the source $Z_{in} = Z_s^*$. This
constrains $\Gamma_{in} = 0$ (for real $Z_{in}$ and $Z_c$). Since, in this
problem, $Z_{in} \in \mathds{R}$. Then we can write the following for the
$P_{avl}$, the maximum amount of deliverable power to the network:

\begin{align*}
        P_{avl} = P_{del}|_{Z_s = Z_{in}^*} =  
        P_{del} &= \left| V_s \right|^2 
        \frac{Z_s^2}{\left( 2 \Re \left( Z_s \right) \right)^2} 
        \frac{ 1 - \left| 0 \right|^2}{\left| 1 + 0 \right|^2}
        \frac{\Re \left( Z_s \right)}{\left| Z_{s} \right|^2} \\
        &= \left| V_s \right|^2 \frac{1}{4 \Re \left( Z_s \right)}
\end{align*}

Again, if the voltage is given in peak values and not in RMS quantities then a
factor of $\frac{1}{2}$ will have to be inserted to account for this. Note that
this solution is the same as the naive solution presented earlier. It has the
distinct advantage of presenting us with the forward and reverse travelling
waves which we will need later.
\subsection{Calculating Port Waves and Powers (Try \#1: $Z_c$-less)}
\subsubsection{Port 1 Waves}
I will now proceed to calculate the port 1 power waves. Please note that the
assumptions made in this problem set are different than the assumptions made in
most reference texts regarding this material. The most significant departure in
this solution set is the assumption that the input reflection coefficient is not
referenced to a characteristic impedance but to the source impedance. The reason
for this is because the problem does not explicitly state that a transmission
line connects the source to the two port network and there is no reason to
assume that there is one. The two port network scattering parameters could, if
desired, be referred back to the source impedance such as to allow for the
following analysis to be perfectly justified. So, to calculate the port 1 power
waves I will use the following relationships:
\begin{align}
    \sqrt{Z_c}\left( a_1 + b_1 \right) &= V_s \frac{Z_{in}}{Z_{in}+Z_s}
    \label{eq1} \\
    b_1 &= \Gamma_{in}a_1 \label{eq2} \\
    \Gamma_{in} &=\frac{Z_{in}-Z_s}{Z_{in}+Z_s} \label{eq3}
\end{align}

Note that \ref{eq3} differs from the typical definition of $\Gamma_{in}$ because
I have assumed that the input impedance has been referred back to the reference
plane at the source impedance. That is, there is no explicit transmission line
that connects the source impedance to the two port network. Thus, the two port
network is ``directly connected'' to the source and this is the first reflection
plane in the signal propagation from the source. $Z_c$ in \ref{eq1} refers to
some choice of a ``reference impedance'' for the system under consideration. It
does not change the voltage values under consideration. This arbitrary choice
of a reference impedance just affects the normalization of these voltages.
Combining (\ref{eq1}) and (\ref{eq2}) yields:

\begin{equation}
    a_1 = \frac{V_s}{\sqrt{Z_c}} \frac{1}{( 1+\Gamma_{in} )}
    \frac{Z_{in}}{Z_{in}+Z_s} \label{prea1}
\end{equation}

Rearranging (\ref{eq3}) for $Z_{in}$ produces $Z_{in} = Z_s \frac{1+\Gamma_{ in
}}{1-\Gamma_{ in }}$. Substituting this into (\ref{prea1}) yields:

\begin{align}
    a_1 &= \frac{V_s}{\sqrt{Z_c}( 1+\Gamma_{in} )} \frac{Z_s \frac{1+\Gamma_{ in }}{1-\Gamma_{
    in }}}{Z_s \frac{1+\Gamma_{ in }}{1-\Gamma_{ in }} + Z_s} \nonumber \\
    &= \frac{V_s}{\sqrt{Z_c}( 1+\Gamma_{in} )} \frac{Z_s \left( 1+\Gamma_{in}
    \right)}{Z_s \left( 1+ \Gamma_{in} \right) + Z_s \left( 1 - \Gamma_{in}
    \right)} \nonumber \\
    &= \frac{V_s}{2 \sqrt{Z_c}} \label{a1}
\end{align}

This result is surprising to anyone who studies microwave circuits. The result
for the incident wave usually relies on the characteristic impedance of the
transmission line and the source impedance. However, because the input impedance
has been referred back to the plane of the source impedance the dependence on
the transmission line drops out. It is surprising (even to the author) that
there exists no dependence on the source impedance. But, this is what the math
dictates. The reflected wave (and, thus, the total voltage across the input network)
definitely does depend on the relationship between the source and the input
impedance. The reflected wave $b_1$ is easily obtained from this using
(\ref{eq2}). 
\begin{equation}        
    b_1 = \frac{V_s \Gamma_{in}}{2 \sqrt{Z_c}} \label{b1}
\end{equation}

Notice that if $\Gamma_{in} = 1$ (an open) that 
$a_1 = \frac{V_s}{2 \sqrt{Z_c}}$
and 
$b_1 = \frac{V_s}{2 \sqrt{Z_c}}$ such
that 
$V_{load} = \sqrt{Z_c} (a_1+b_1) = V_s$ (reflects completely in phase). This
makes sense; $\Gamma_{in} = 1$ corresponds to $Z_{in}$ being an open. This would
mean no current flows through the circuit and $V_{in} = V_s$.

If $\Gamma_{in} = -1$ (a short)
\begin{gather*}
    a_1 = \frac{V_s }{2\sqrt{Z_c}} \\
    b_1 = -\frac{V_s}{2\sqrt{Z_c}} \\
    V_{load} = 0
\end{gather*}

If $\Gamma_{in} = 0$ (a matched load)
\begin{gather*}
    b_1 = 0 \\
    a_1 = \frac{V_s}{2 \sqrt{Z_c}} \\
    V_{load} = \frac{V_s }{2}
\end{gather*}

Thus, this recovers the well-known results in the limiting cases.
\subsubsection{Port 2 Waves}
Calculating the port 2 power waves I will use the following relationships:

\begin{align}
    b_2 &= S_{21}a_1 + S_{22}a_2 \label{eq1a}\\
    a_2 &= \Gamma_l b_2 \label{eq2a} \\
    a_1 &= \frac{V_s}{2 \sqrt{Z_c}} \label{eq3a}
\end{align}

Combining (\ref{eq1a}) and (\ref{eq2a}) yield:

\begin{equation}
b_2 = \frac{S_{21}a_1}{1-S_{22}\Gamma_l} \label{eq4a}
\end{equation}

Combining (\ref{eq4a}) and (\ref{eq3a}) yields:

\begin{equation}
    b_2 =  \frac{V_s}{2 \sqrt{Z_c}}\frac{S_{21}}{1-S_{22}\Gamma_l} \label{b2}
\end{equation}

Obtaining $a_2$ from (\ref{eq2a}) and (\ref{b2}) is trivial:

\begin{equation}
    a_2 = \frac{V_s}{2 \sqrt{Z_c}}\frac{S_{21}\Gamma_l}{1-S_{22}\Gamma_l} \label{a2}
\end{equation}

Note that $\Gamma_l$ is usually defined as $\Gamma_l = \frac{Z_l - Z_c}{Z_l +
Z_c}$ and requires some notion of a
characteristic impedance. Why can we not do the same thing we did with the input
impedance and refer the impedance of the load to port 2 of the two-port
network. We could! In fact, this is usually what is done. Based on the output
impedance of the two-port network and the impedance of the load we could
determine this reflection coefficient. Thus, instead of defining the reflection
coefficient of the load in this way, we will consider looking back from the load
to determine the output impedance of the network, $Z_{out}$. Then, we will
calculate the reflection coefficient at the interface between the two-port
network and the load as $\Gamma_l = \frac{Z_l - Z_{out}}{Z_l + Z_{out}}$. Note that
$\Gamma_{out} = -\Gamma_l = \frac{Z_{out} - Z_l}{Z_{out} + Z_l}$ (because we're
looking the other way). To determine what $\Gamma_{out}$ is in terms of the
parameters of the problem consider that:

\[ 
        \frac{b_2}{a_2} = \Gamma_{out} = S_{21}\frac{a_1}{a_2} + S_{22}
\]

by \ref{eq1a}. We get a relationship between $a_1$ and $a_2$ by considering the
following three things:

\begin{align*}
    b_1 = a_1 S_{11} + a_2 S_{12} \\
    a_1 = \Gamma_s b_1 \\
    \Gamma_s = \frac{Z_s-Z_{in}}{Z_s+Z_{in}}
\end{align*}

Substituting the second of the above two equations into the first and solving
for $\frac{a_1}{a_2}$ yields:

\[ 
        \frac{a_1}{a_2} = \frac{S_{12} \Gamma_l}{1-\Gamma_s S_{11}} 
\]

so that:

\[ 
        \Gamma_{out} = S_{22} + \frac{S_{12}S_{21} \Gamma_s}{1-S_{11}\Gamma_s} 
\]

This implies (and it can be shown that)

\[ 
        \Gamma_{in} = S_{11} + \frac{S_{12}S_{21} \Gamma_l}{1-S_{22}\Gamma_l} 
\]

Note that we don't have an expression that allows us to calculate $\Gamma_{out}$
or $\Gamma_{in}$, yet. $\Gamma_{out} = -\Gamma_l$ depends on $\Gamma_s = -\Gamma_{in}$ but
$\Gamma_{in}$ depends on $\Gamma_l$. Thus, substituting our expression for
$\Gamma_s = -\Gamma_l$ in to $\Gamma_{in}$ will allow us to solve for
$\Gamma_{in}$ in terms of everything else. This turns out to be very difficult
to solve by hand (as I've experienced). Using Mathematica to help with
generating the algebraic solution yields not one but two solutions for both
$\Gamma_{in}$ and $\Gamma_{out}$ which are as follows:

\[ 
        \begin{array}{|c|c|}
            \hline & \\ \text{Quantity} & \text{Expression}\\[.3cm]
            \hline & \\ \Gamma_{in_1} & -\frac{-S_{12}^2 S_{21}^2+2 S_{11}
                    S_{12} S_{22} S_{21}-\sqrt{4 \left(S_{12} S_{21} S_{22}-S_{11}
                    \left(S_{22}^2+1\right)\right){}^2+\left(-\left(S_{22}^2+1\right)
                    S_{11}^2+2 S_{12} S_{21} S_{22} S_{11}-S_{12}^2
                    S_{21}^2+S_{22}^2+1\right){}^2}-\left(S_{11}^2-1\right)
                    \left(S_{22}^2+1\right)}{2 \left(S_{11} \left(S_{22}^2+1\right)-S_{12}
                    S_{21} S_{22}\right)} \\[.5cm]
            \hline & \\ \Gamma_{in_2} &-\frac{-S_{12}^2 S_{21}^2+2 S_{11} S_{12}
                    S_{22} S_{21}+\sqrt{4 \left(S_{12} S_{21} S_{22}-S_{11}
                    \left(S_{22}^2+1\right)\right){}^2+\left(-\left(S_{22}^2+1\right)
                    S_{11}^2+2 S_{12} S_{21} S_{22} S_{11}-S_{12}^2
                    S_{21}^2+S_{22}^2+1\right){}^2}-\left(S_{11}^2-1\right)
                    \left(S_{22}^2+1\right)}{2 \left(S_{11} \left(S_{22}^2+1\right)-S_{12}
                    S_{21} S_{22}\right)} \\[.5cm]
            \hline & \\ \Gamma_{out_1} & \frac{S_{12}^2 S_{21}^2-2
                    S_{11} S_{12} S_{22} S_{21}+\sqrt{4 \left(S_{12} S_{21}
                    S_{22}-S_{11}
                    \left(S_{22}^2+1\right)\right){}^2+\left(-\left(S_{22}^2+1\right) S_{11}^2+2
                    S_{12} S_{21} S_{22} S_{11}-S_{12}^2
                    S_{21}^2+S_{22}^2+1\right){}^2}+\left(S_{11}^2+1\right)
                    \left(S_{22}^2-1\right)}{2 \left(S_{11}^2+1\right) S_{22}-2 S_{11} S_{12}
                    S_{21}} \\[.5cm]
            \hline & \\ \Gamma_{out_2} & \frac{S_{12}^2 S_{21}^2-2 S_{11} S_{12} S_{22}
                    S_{21}-\sqrt{4 \left(S_{12} S_{21} S_{22}-S_{11}
                    \left(S_{22}^2+1\right)\right){}^2+\left(-\left(S_{22}^2+1\right)
                    S_{11}^2+2 S_{12} S_{21} S_{22} S_{11}-S_{12}^2
                    S_{21}^2+S_{22}^2+1\right){}^2}+\left(S_{11}^2+1\right)
                    \left(S_{22}^2-1\right)}{2 \left(S_{11}^2+1\right) S_{22}-2 S_{11} S_{12}
                    S_{21}} \\[.5cm] \hline
        \end{array}
\]

Although these expressions are extremely complicated you can see that the main
difference between the two solutions for $\Gamma_{in}$ and $\Gamma_{out}$ is 
the sign in front of the radical. These are the two roots of a quadratic
solution. In the event that the two solutions are different, numerically, the
proper $\Gamma_{in}$ and $\Gamma_{out}$ must be chosen in such a way that
problem still makes sense. For passive circuits, it must be the case that
$\left| \Gamma_{in} \right| \le 1$ and $ \left| \Gamma_{out} \right| \le 1$. 
\subsubsection{Power to the Load}
To determine the power delivered to the load we begin with the definition of
real power (note that I assume, immediately, that the voltages and currents are
provided as peak quantities):

\[ 
        P_{load} =  \frac{1}{2}\Re\left( V_{load}I^*_{load} \right) 
\]

In this case, $V_{load} = \sqrt{Z_c} \left( a_2 + b_2 \right) $ and $I_{load}
=\sqrt{Z_c} \frac{b_2+a_2}{\sqrt{Z_l}}$. So $P_{load}$ can be rewritten as
follows:

\begin{align*}
    P_{load} &= \frac{1}{2}\Re\Big(Z_c \left( a_2+b_2 \right)\frac{\left( b_2^*+a_2^*
    \right)}{Z_l}\Big) \\
\end{align*}

We have expressions for both $b_2$ and $a_2$ so we're done.

To calculate the power reflected to the source I need to take the incident power
and subtract the amount that is delivered to the input network.

\[ 
        P_{source} = \frac{1}{2}\Re \left( V_{source}I^*_{source} \right) 
\]

$$V_{source} = V_s \frac{Z_s}{Z_s+Z_{in}}$$ and $$I_{source} =\frac{V_s}{Z_s +
Z_{in}}$$.

Substituting these two expressions into $P_{source}$ yields:

\begin{align*}
    P_{source} &= \frac{1}{2} \Re \left( \frac{V_s Z_s}{Z_s + Z_{in}}
\frac{V_s^*}{Z_S^* + Z_{in}^*} \right) \\
&= \frac{\left| V_S \right|^2}{2} \Re \left( \frac{Z_s}{|Z_s+Z_{in}|^2} \right)
\\
&= \frac{\left| V_s \right|^2}{2 \left| Z_s + Z_{in} \right|^2} \Re(Z_s)
\end{align*}

Note that in a similar way I could determine the amount of power delivered to
the input network and, also, to the circuit as a whole.

\begin{align*}
    P_{network} &= \frac{1}{2} \Re \left( \left( V_s \frac{Z_{in}}{Z_{in}+Z_s}
    \right) \left( \frac{V_s}{Z_{in}+Z_s} \right)^*  \right) \\
    P_{del} = P_{network} + P_{source} &= \frac{1}{2} \Re \left( V_s \left(
\frac{\left(a_1 + b_1  \right)\sqrt{Z_c}}{Z_{in}} \right)^*  \right)
\end{align*}

\subsubsection{Plugging in Numbers}

We will now plug the following numbers into equations \ref{a1}, \ref{b1}, \ref{b2}, \ref{a2}
 to obtain $a_1$, $b_1$, $b_2$ and $a_2$, respectively.

\[
        \def\arraystretch{1.5}
        \begin{array}{|c|c|}
            \hline  \text{\quad Variable \quad}  & \text{\quad Value \quad } \\
            \hline \hline Z_{l} = Z_c \text{(by choice)} & \SI{50}{\ohm} \\
            \hline Z_{s} & \SI{100}{\ohm} \\
            \hline \Gamma_{in} = -\Gamma_{s} = \frac{Z_s - Z_{in}}{Z_s+Z_{in}} &
            \Gamma_{in_1} \approx -8.79 + j 458\cdot 10^{-3} \text{\enskip
            (invalid) \quad or \quad} \Gamma_{in_2} \approx 114\cdot10^{-3} +j
            5.92 \cdot 10^{-3} \\
            \hline \Gamma_{out} = -\Gamma_{l} = \frac{Z_l-Z_{out}}{Z_l+Z_{out}}  &
            \Gamma_{out_1} \approx -5.96 + j 433 \cdot 10^{-3} \text{\enskip (invalid) \quad or \quad}
            \Gamma_{out_2} \approx 167\cdot 10^{-3} + j 12.1 \cdot 10^{-3} \\
            \hline S_{11} & .1 \phase{- 30 \degree} \\
            \hline S_{12} & .4 \phase{- 75 \degree} \\
            \hline S_{21} & .95 \phase{- 45 \degree} \\
            \hline S_{22} & .15 \phase{- 10 \degree} \\
            \hline V_{s} & \SI{20}{\volt}\phase{0 \degree} \\
            \hline Z_{in} = Z_s \frac{1+\Gamma_{in}}{1 - \Gamma_{in}} &  \approx
            \SI{126 + j 1.51}{\ohm} \\ \hline
        \end{array}
\]

Plugging these numbers in yields:

\[
        \def\arraystretch{1.5}
        \begin{array}{|c|c|}
            \hline \text{ \quad Quantity \quad } & \text{ \quad Value \quad }
            \\
            \hline \hline a_1 &
            \SI[parse-numbers=false]{\sqrt{2}}{\volt\per\sqrt{\ohm}}
            \text{\quad (exactly)} \\
            \hline b_1 & \approx \SI[parse-numbers=false]{160 + j 8.37}{\milli\volt\per\sqrt{\ohm}} \\
            \hline a_2 & \approx \SI[parse-numbers=false]{-166+j 143}{\milli\volt\per\sqrt{\ohm}} \text{\quad
        (exactly)} \\
        \hline b_2 & \approx \SI[parse-numbers=false]{929 -j 924}{\milli\volt\per\sqrt{\ohm}} \\ 
        \hline P_{load} & \approx \SI{596}{\milli\watt} \\
        \hline P_{source} & \approx \SI{393}{\milli\watt} \\
        \hline P_{network} & \approx \SI{493}{\milli\watt} \\ 
        \hline P_{del} & \approx \SI{886}{\milli\watt} \\ \hline
        \end{array}
\]

Note that more the source produced $\approx \SI{946}{\milli\watt}$ of power.
Note that for $Z_{in} = \SI{126}{\ohm}$ and $Z_s = \SI{100}{\ohm}$ that
$P_{src} = \SI{885}{\milli\watt}$. Looking at the table above this is also the
sum of the amount of reflected power, $P_{ref}$, and the power delivered to the
network, $P_{network}$ (as it should be). However, the amount of power delivered
to the load is greater than that delivered to the input network. This implies
that the two-port network was supplying power to the load. This can not be. I
don't know why this is happening. I have checked my formulae multiple times. Any
assistance that could be shed on this matter would be greatly appreciated. 

\subsection{Calculating Port Waves and Powers ``Properly'' (Try \#2: With $Z_c$)}

The approach we will take, now, will follow the more traditional paths that are
taken to solve these problems. We will begin by calculating each of the\ incident
and reflected waves.
\subsubsection{Port 1 Waves}
$V_1 = (a_1 + b_1)\sqrt{Z_c} = V_s \frac{Z_{in}}{Z_s + Z_{in}}$. $b_1 =
\Gamma_{in} a_1$.

\[ 
        a_1 = \frac{V_s}{\sqrt{Z_c}} \frac{Z_{in}}{Z_s + Z_{in}} \frac{1}{1+\Gamma_{in}} 
\]

Now,

\[ 
        b_1 = \Gamma_{in} a_1 = \frac{V_s}{\sqrt{Z_c}} \frac{Z_{in}}{Z_s +
        Z_{in}} \frac{\Gamma_{in}}{1+ \Gamma_{in}} 
\]

Note that the expressions given for $\Gamma_{in}$ provided earlier in terms of
the scattering parameters and $\Gamma_l$ are still valid. But, we must change
$\Gamma_{in}$ to refer to an characteristic impedance $Z_{c_1}$ at the input and we must
change $\Gamma_l$ to refer to a characteristic impedance $Z_{c_2}$ at the
output. These may be the same, but in general they do not have to be.

\[ 
        \Gamma_{in} = \frac{Z_{in}-Z_{c_1}}{Z_{in}+Z_{c_1}} \text{\quad and
        \enskip} \Gamma_{l} = \frac{Z_{l}-Z_{c_2}}{Z_{l}+Z_{c_2}} 
\]

\subsubsection{Port 2 Waves}
Now, $a_1$ and $b_1$ are completely defined and are able to be calculated. The
reference impedance $Z_c$ defined in both of them should be taken with respect
to the characteristic impedance at the input side, $Z_{c_1}$. $b_2$ can be
calculated in a similar manner as before:

\[ 
        b_2 = a_1 S_{21} + a_2 S_{22}
\]

But, $a_2 = \Gamma_l b_2$.

\[ 
        b_2 = a_1 \frac{S_{21}}{1-\Gamma_l S_{22}}  = \frac{V_s}{\sqrt{Z_{c_2}}}
        \frac{Z_{in}}{Z_s + Z_{in}} \frac{1}{1+\Gamma_{in}}
    \frac{S_{21}}{1-\Gamma_l S_{22}}
\]

\[ 
        a_2 = \Gamma_l b_2 = a_1\frac{S_{21}\Gamma_l}{1-\Gamma_l S{22}}  =  \frac{V_s}{\sqrt{Z_{c_2}}}
        \frac{Z_{in}}{Z_s + Z_{in}} \frac{\Gamma_l}{1+\Gamma_{in}}
    \frac{S_{21}}{1-\Gamma_l S_{22}}
\]

\subsubsection{Plugging in Numbers (These Actually Work)}

Performing a similar thing as before, I will fill the table below with known
quantities. Below that I will supply the calculated quantities.


\[
        \def\arraystretch{1.5}
        \begin{array}{|c|c|}
            \hline  \text{\quad Variable \quad}  & \text{\quad Value \quad } \\
            \hline \hline Z_{l} = Z_{c_1} \text{(by choice)} & \SI{50}{\ohm} \\
            \hline Z_{s} & \SI{100}{\ohm} \\
            \hline \Gamma_{in} & 86.6 \cdot 10^{-3} - j 50 \cdot 10^{-3} \\
            \hline S_{11} & .1 \phase{- 30 \degree} \\
            \hline S_{12} & .4 \phase{- 75 \degree} \\
            \hline S_{21} & .95 \phase{- 45 \degree} \\
            \hline S_{22} & .15 \phase{- 10 \degree} \\
            \hline V_{s} & \SI{20}{\volt}\phase{0 \degree} \\
            \hline Z_{in} = Z_{c_1} \frac{1+\Gamma_{in}}{1 - \Gamma_{in}} &  \approx
            \SI{118 - j 12.0}{\ohm} \\ \hline
        \end{array}
\]

Plugging these numbers (see Mathematica in appendix) in yields:

\[
        \def\arraystretch{1.5}
        \begin{array}{|c|c|}
            \hline \text{ \quad Quantity \quad } & \text{ \quad Value \quad }
            \\
           \hline \hline a_1 &
            \SI[parse-numbers=false]{1}{\volt\per\sqrt{\ohm}} \\
            \hline b_1 & \approx \SI[parse-numbers=false]{86.6 - j 50.0}{\milli\volt\per\sqrt{\ohm}} \\
            \hline a_2 & \approx \SI[parse-numbers=false]{0 +j 0}{\milli\volt\per\sqrt{\ohm}} \text{\quad
        (exactly)} \\
        \hline b_2 & \approx \SI[parse-numbers=false]{672 -j 672}{\milli\volt\per\sqrt{\ohm}} \\ 
        \hline & \\ P_{load} & \frac{1}{2} \Re \left( \sqrt{Z_{c_2}} \left( a_2 + b_2
        \right) \left( \sqrt{Z_{c_2}}\frac{a_2 + b_2}{Z_l} \right)^*
    \right)\approx \SI{451}{\milli\watt} \\ & \\
        \hline & \\ P_{source} & \frac{1}{2} \Re \left( \left( \frac{V_s Z_{s}}{Z_{in}
        + Z_s} \right) \left( \frac{V_s}{Z_{in}+Z_s} \right)^* \right)\approx
        \SI{418}{\milli\watt} \\ & \\ 
        \hline & \\ P_{network} & \frac{1}{2} \Re\left( \left( V_s
\frac{Z_{in}}{Z_{in}+Z_s} \right) \left( \frac{V_s}{Z_{in}+Z_s} \right)^*
\right) \approx \SI{495}{\milli\watt} \\ & \\
\hline & \\ P_{del} & \approx \frac{1}{2}\Re \left( V_s \left( \frac{V_s}{Z_{in}
+ Z_s} \right)^* \right) \approx \SI{913}{\milli\watt} \\ & \\ \hline
        \end{array}
\]

Note that, now, $P_{source} + P_{network}$ (the power delivered to the source
plus that delivered to the load and the two port network) is equal to the
delivered power, as it should be, \textbf{and $P_{load} < P_{network}$}, as it
should be.
