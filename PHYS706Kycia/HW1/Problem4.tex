\begin{homeworkProblem}

\textbf{The time-averaged potential of a neutral hydrogen atom is given by }

\begin{equation}
\Phi = \frac{q}{4\pi\epsilon_0}\frac{e^{-\alpha r}}{r}(1+\frac{\alpha r}{2})
\end{equation}
 
\textbf{where q is the magnitude of the electronic charge, and $a^{-1} = a_0/2$, $a_0$ being the 
Bohr radius. Find the distribution of charge (both continuous and discrete) that will 
give this potential and interpret your result physically. }
\\ \par
What I will do to solve this will be to take the Laplacian of both sides of this equation. I will use the following identities: $\nabla^2 (f g) = g \nabla^2 f + f \nabla^2 g + 2 \nabla f \cdot \nabla g $ and $\nabla^2 \frac{1}{r} = -4\pi \delta^3(\vec{r})$ and $\nabla \frac{1}{r} = -\frac{\hat{r}}{r^2}$.

\begin{align}
	\rho &= -\epsilon_0 \nabla^2 \Phi \nonumber \\
	&= A \big(\frac{1}{r}\nabla^2 e^{-\alpha r} + e^{-\alpha r}\nabla^2 \frac{1}{r} + 2 \nabla {1}{r} \cdot \nabla e^{-\alpha r} + \frac{\alpha}{2}\nabla^2 e^{-\alpha r}\big) \nonumber \\
	\intertext{A here has absorbed the quantities $\frac{-\epsilon_0 q}{4\pi \epsilon_0}$.Using the identities above and the following results, $\nabla^2 e^{- \alpha r} = \frac{e^{-\alpha r}\alpha}{x}(x\alpha -2)$ and $\nabla e^{-\alpha r} = - \alpha e^{-\alpha r} \hat{r}$ :}
		&= A \big(\frac{1}{r}(\frac{e^{-\alpha r}\alpha}{r}(r\alpha -2)) - e^{-\alpha r}4\pi \delta^3(\vec{r})-2	\frac{\hat{r}}{r^2} \cdot (-\alpha e^{-\alpha r} \hat{r}) + \frac{\alpha}{2}\frac{e^{-\alpha r}\alpha}{r}(r\alpha -2)\big) \nonumber \\
	&= A \big( \frac{e^{-\alpha r}\alpha}{r^2}(r\alpha -2) - e^{-\alpha r}4\pi \delta^3(\vec{r}) + 2 \alpha \frac{e^{-\alpha r}}{r^2} + \frac{\alpha^3}{2} e^{-\alpha r} -\frac{e^{-\alpha r} \alpha^2}{r} \big) \nonumber \\
	&= A (-e^{-\alpha r} 4\pi \delta^3(\vec{r}) + \frac{\alpha^3}{2}e^{-\alpha r}) \nonumber \\
	&= -\epsilon_0 \frac{q}{4\pi \epsilon_0} (-e^{-\alpha r} 4\pi \delta^3(\vec{r}) + \frac{\alpha^3}{2}e^{-\alpha r}) \nonumber \\
	&= q \delta^3(\vec{r}) - q e^{-\alpha r} \frac{1}{a_0^3 \pi} \nonumber
\end{align}
\\ \par
I have managed to eliminate the exponential tacked on the $\delta^3$ term because it has no effect on the integral over which the $\delta^3$ would have significance. There, the exponential just reduces to the constant 1.
\\ \par
The first term looks like a positive point charge located at the origin. The second term looks like a negative charge distribution that is smeared over all of space. If I integrate that second term over all space I get -q: $4\pi \frac{q}{\pi a_0^3} \int_0^{\infty} e^{-\alpha r} r^2 dr = -q$. So, this looks like the electron term.
\\ 
\par
Actually, as an aside, the ground state probability distribution of the electron's position in space is $\Psi(r) = \frac{1}{\sqrt{\pi} a_0^{1.5}}e^{\frac{r}{a_0}}$. The probability density of the electron is given by $\rho(x) = |\Psi(x)|^2 (-q) = -q e^{-\alpha r} \frac{1}{a_0^3 \pi}$. This is the same answer as I obtained above, through analyzing classical electrodynamics equations. This is an example of the oft-cited Ehrenfest theorem in which averages over the quantum regime approach the results obtained in the classical regime.

\end{homeworkProblem}