\section{Lossy Matching Networks}
\addtocounter{section}{1}
\setcounter{equation}{0}

The first thing to do is to find the amount of loss in the matching network.
This loss will be due to two non-ideal conditions for the circuit elements
\begin{enumerate}
    \item The inductor is lossy with series lossy element $R_s$
    \item The capacitor is lossy with parallel lossy element $R_p$
\end{enumerate}

Thus, the loss in $R_s$ can be found by driving the now-matched input network
with a voltage $V_{in}$ and finding the following:

\[ 
        P_{s} = \frac{1}{2} \Re \left( V_{s} I_s^* \right) 
\]

$V_s = V_{in} \frac{R_s}{Z_{in}}$ and $I_s = \frac{V_{in}}{Z_{in}}$ 

The power delivered to the series resistor is now 
\[ 
    P_s = \frac{1}{2} \frac{\left|V_{in}\right|^2 R_s}{\left|Z_{in}\right|^2}
\]

The power delivered to the parallel resistor is a bit more tricky. We can
consider the voltage across the parallel branch (with $R_p$, $R_l$ and $Z_C$).
The voltage across that branch is 

\[ 
        V_p = V_{in} \frac{Z_c||R_l||R_p}{Z_{in}}
\]

The current through $R_p$, $I_p$, is:
\[ 
    I_p = \frac{V_p}{R_p} 
\]

Thus, we can write the power delivered to this resistor as:

\[ 
    P_p = \frac{1}{2}  \frac{\left|V_p \right|^2 }{R_p}
\]

And the total power delivered to these two circuits elements is just

\begin{align}
    P_s + P_p = \frac{1}{2} \frac{\left| V_{in} \right|^2 R_s}{\left| Z_{in}
    \right|^2} + \frac{1}{2} \frac{\left| V_{in} \right|^2 \Big|
Z_c||R_l||R_p \Big|^2}{R_p\left| Z_{in} \right|^2} \label{power}
\end{align}

This is exact. However, if we consider the case where we are close to the
matching frequency then $Z_A \approx - Z_B$ and most of the energy is dissipated
in the resistors (not reflected or stored in the inductor and capacitor). Also,
if the inductors and capacitors are any good, it should be the case that $ 
 R_p >> R_l >> R_s$ (a large shunt resistor on the capacitor implies that it's
 very good). This means that if we write an expression for the power
lost near the matching frequency it would look like:

\[ 
        P_{lost} = \left( V_s \frac{R_s}{R_s + R_p||R_l} \right)^2 \frac{1}{R_s}
        + \left( V_s \frac{R_p || R_l}{R_s + R_p ||R_l} \right)^2 \frac{1}{R_p}
\]

But, because $R_p >> R_l$ then $R_p || R_l \approx R_l$.

\[ 
        P_{lost} \approx V_s^2 \left( \frac{R_s}{R_l + R_s} \right)^2
        \frac{1}{R_s} + V_s^2 \left( \frac{R_l}{R_s+R_l} \right)^2
        \frac{1}{R_p} = V_s^2 / \left( R_s +R_l \right)
\]

Expressing this in terms of the transformation ratio $ \eta = R_l/R_{in}$ (where $R_{in}
\approx R_s + R_l$) yields:
\[ 
        P_{lost} \approx V_s^2 \frac{1}{R_{in}} = V_s^2 \frac{\eta}{R_l} 
\]

Notice that as the transformation ratio increases, the loss increases.

We know the value of $R_l$ is \SI{50}{\ohm} since this type of matching network
is designed to lower the impedance of the load. Now, we get to pick the sign of
the susceptance. We will select it to be positive such as to maintain the
geometry of the capacitor shunting the load. Realizing that $Z_c =
\SI{10}{\ohm}$ and $R_l = \SI{50}{\ohm}$ we have

\[ 
        B =\frac{1}{50} \sqrt{5-1} = \frac{2}{50}~\text{S}
\]

At 3 GHz this corresponds to a capacitor of value: $\frac{B}{\omega} = C$:

\[ 
        C = \frac{\frac{2~\text{S}}{50}}{2\pi \cdot 3~\text{GHz}} \approx
        \SI{2.12}{\pico\farad}
\]

Using Pozar's 5.3b to find X, the series impedance of the matching network
yields:

\begin{align*}
    X &= \frac{1}{B} + \frac{X_l Z_c}{R_l} - \frac{Z_0}{B R_l} \\
    &= \frac{1}{B} - \frac{Z_c}{B R_l} \\
    &= \SI{25}{\ohm} - \frac{10~\Omega}{\frac{2S}{50}50~\Omega} \\
    &= \SI{20}{\ohm}
\end{align*}

This is an inductor of value $L = \frac{X}{\omega}$:
\[ 
        L = \frac{20~\Omega}{2\pi \cdot 3 \text{~GHz}} \approx
        \SI{1.06}{\nano\henry}
\]

If the Q factors of the inductor and capacitor are 15 and 30, respectively, this
implies a value of $R_s = \frac{X_l}{Q} = \frac{20~\Omega}{15} =
\frac{4~\Omega}{3}$ and $R_p = Q_c X_c = 20 \cdot 25~\Omega = 500~\Omega$. 

Plugging in the values I have, the power delivered to the load is:

\[ 
        P_{load} = \frac{1}{2}\left( |V_s|^2 \frac{\Big| Z_{c}||R_l||R_p \Big|^2
                }{\Big| Z_{in} \Big|^2} \right) \frac{1}{R_l} 
\]

Dividing this by the power delivered to the whole network:

\[ 
        P_{del} = \frac{V_s^2}{2} \Re \left( \frac{1}{Z_{in}^*} \right)
\]

Calculating the following on Mathematica I obtain:

\[ 
        \text{Insertion Loss~} =-10 \log_{10} \frac{P_{load}}{P_{del}} \approx .93~\text{dB}
\]

This aligns well with the simulated value $-20 \log_{10}\left| S_{21} \right|$
which (apart from ignoring reflection) demonstrates the same idea. At the matching
frequency, the amount of reflection is very small ($< 20 \text{~dB}$). The
amount of forward gain reported by ADS is $-.968 \text{~dB}$, i.e. a loss of
$.968 \text{~db}$. Apart from the reflection, these values are consistent with
one another.
