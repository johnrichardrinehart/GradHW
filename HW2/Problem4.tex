\begin{homeworkProblem}{Problem 4 (Jackson ed. 3 Problem 3.4a)}

The first thing to realize is that no longer is there azimuthal symmetry in this problem. Thus, $\Phi(r,\theta,\phi) = \sum\limits_0^\infty (A_l r^l + B_l r^{-(l+1)})Y_{lm}(\theta,\phi)$. I have been given $V(R,\phi)$. Now, I need to use orthogonality and slick mathematical tricks to get my answer into a tractable form. $V(R,\phi)$ alternates sign based on how many ``slices'' the sphere has in it (there are 2n wedges for n slices). Thus, the potential can be expressed in the following form if the sphere is aligned such that at $\phi = 0$ the dividing line between V and -V is along the x-axis and for $\phi \in (0,\pi/n)$ the potential is positive.

\[
\Phi(\Phi,R) =
\begin{cases}
V \quad \phi \in \big(\frac{\pi 2j}{n},\frac{\pi (2j+1)}{n}\big) \\
V \quad \phi \in \big(\frac{\pi (2j+1)}{n},\frac{\pi (2j+2)}{n}\big)
\end{cases}
\text{for $j = 0,1,2,...,n-1$}
\]

\end{homeworkProblem}