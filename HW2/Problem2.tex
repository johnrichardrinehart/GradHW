\begin{homeworkProblem}
\textbf{Two concentric spheres have radii a, b $(b > a)$ and each is divided into two hemi- 
spheres by the same horizontal plane. The upper hemisphere of the inner sphere 
and the lower hemisphere of the outer sphere are maintained at potential V. The 
other hemispheres are at zero potential 
Determine the potential in the region $a < r < b$ as a series in Legendre poly- 
nomials. Include terms at least up to $l = 4$. Check your solution against known 
results in the limiting cases $b \rightarrow \infty$, and $a \rightarrow 0$. } \\
\par
The boundary conditions are expressed in terms of potentials. Thus, I should use Laplace's equation and the Legendre polynomials to determine the potential everywhere between the two spheres. Due to the azimuthal symmetry of the problem I can assume that ``$m = 0$'' for my associated Legendre polynomials.

\[
\Phi(r,\theta) = \sum\limits_{l = 0}^{\infty} \left( A_l r^l + B_l r^{-(l+1)} P_l(\cos\theta) \right)
\]

Using orthogonality of the Legendre polynomials we can solve for $A_l$ and $B_l$. We have two boundary conditions. At $r = a$ $\Phi = V$ and at $r = b$ $\Phi = $. This sets up the two equations (note that $\int\limits_0^\pi P_l(\cos \theta)P_m(\cos \theta) d\cos\theta = \frac{2}{l + 1} \delta_{l m}$). 

\begin{align}
\int_{\theta = 0}^{\theta = \pi} V(a,\theta) P_k(\cos\theta) d\cos\theta &= A_k a^k + B_k a^{-(k+1)} \nonumber \\
\int_{\theta = 0}^{\theta = \pi} V(b,\theta) P_k(\cos\theta) d\cos\theta &= A_k b^k + B_k b^{-(k+1)} \nonumber \\
\intertext{These two integrals can be substantially reduced by realizing that the potential (hence, the integrand) is zero for half of the integration.} \nonumber
V\int_{\theta = 0}^{\theta = \pi/2} P_k(\cos\theta) d\cos\theta &= A_k a^k + B_k a^{-(k+1)} \nonumber \\
V\int_{\theta = \pi/2}^{\theta = \pi} P_k(\cos\theta) d\cos\theta &= A_k a^k + B_k a^{-(k+1)} \nonumber \\
\intertext{In general, these integrals have no closed form analytic solution (truth be told, these integrals can be expressed in terms of gamma functions, but this is unnecessarily complicated and not particularly enlightening. I will, at this time, change integration variables from $\cos\theta \rightarrow x$.} \nonumber 
V\int_{1}^{0} P_k(x) dx &= A_k a^k + B_k a^{-(k+1)} \nonumber \\
V\int_{0}^{-1} P_k(x) dx &= A_k a^k + B_k a^{-(k+1)} \nonumber \\
\intertext{Now, a useful property for evaluating Legendre polynomials is the following: $P_l(x) = \dfrac{1}{2l+1} \der{}{x}\left(P_{l+1}(x)+P_{l-1}(x)\right)$. One might be concerned regarding the $P_0(x)$ case. However, $P_0(x)$ = 1 from $x = -1 \rightarrow 1$. Thus, the expression for the $P_0(x)$ case is given below. After, the $P_l(x)$ cases will be evaluated (where $l>1$).} \nonumber \\
V\int_{1}^{0} P_0(x) dx &= A_k a^k + B_k a^{-(k+1)} = -V \nonumber \\
V\int_{0}^{-1} P_0{x} dx &= A_k a^k + B_k a^{-(k+1)} = -V \nonumber
\end{align}

\begin{align}
\intertext{For $k>1$ we consider the first integral: $V\int_{1}^{0} P_k(x) dx = A_k a^k + B_k a^{-(k+1)}$:} \nonumber \\
V\int_{1}^{0} P_k(x) dx &= \gamma_k \der{}{x}\left(P_{k+1}(x)+P_{k-1}(x)\right) \nonumber \\
\intertext{Throwing in a couple more Legendre polynomial properties (thank you, Wikipedia!) $P_l(1) = 1 \forall l$ and $(-1)^l P_l(-x) = P_l(x) \forall l$. In plain English, the second identity states that if $l$ is odd that $P_l$ is odd. So, $P_l(-1) = -1$ for odd $l$. For even $l$, $P_l(-1) = 1$. Considering even $l = 2p \; p = 0,1,2,...$} \nonumber 
% Even l's
V\int_{1}^{0} P_{2p}(x) d x &= \gamma_{2p} \int_1^0 \der{}{x}\left(P_{2p+1}(x)+P_{2p-1}(x)\right)dx \nonumber \\
&= \gamma_{2p} \big(P_{2p+1}(1)-P_{2p-1}(1) + P_{2p-1}(0) - P_{2p+1}(0) \big) \nonumber
\intertext{According to the properties above this can be easily seen to be zero for all $p$. Now, considering odd $l = 2p+1 \; p = 0,1,2,...$} \nonumber
V\int_{1}^{0} P_{2p+1}(x) d x &= \gamma_k \int_1^0 \der{}{x}\left(P_{2p+2}(x)+P_{2p}(x)\right)dx \nonumber \\
&= \gamma_{2p+1} \big(P_{2p+2}(0)-P{2p}(0) + P_{2p}(1) - P_{2p+2}(1) \big) \nonumber \\
\intertext{By the properties listed above we can reduce this to the following expression:} \nonumber
&= \gamma_{2p+1} \big(P_{2p+2}(0) - P_{2p}(0) \big) \nonumber \\
\intertext{As there is no closed form solution for this expression (outside of the use of gamma functions) this integral is for all intents and purposes completed. For the case where $l = 0$ the integral simplifies to $-1$ since $P_0(x) = 1$ from $x = -1 \rightarrow$ 1.} \nonumber
\end{align}

Performing similar analysis on the other integral yields the following:
\[
\int_{-1}^0 P_l(x) dx = \begin{cases} 0 & \text{l is even and $l > 0$} \\ -\gamma_l \big(P_{l+1}(0) - P_{l-1}(0) \big) & \text{l is odd} \\ -\frac{1}{2} & \text{l is 0} \end{cases}
\]

Succinctly put, the first result is:

\[
\int_{0}^{-1} P_l(x) dx = \begin{cases} 0 & \text{l is even and $l > 0$} \\ \gamma_l \big(P_{l+1}(0) - P_{l-1}(0) \big) & \text{l is odd} \\ -\frac{1}{2} & \text{l is 0} \end{cases}
\]


Given the similarity of the two expressions I can express 

\end{homeworkProblem}