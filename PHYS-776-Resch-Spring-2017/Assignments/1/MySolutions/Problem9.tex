% Problem 1.9
\begin{homeworkProblem}[Problem 9]
   \begin{homeworkSection}{a)}
      To show this, we'll determine the matrix elements of $ E $ in the number
      basis.
      \begin{align}
         E &\equiv (n+1)^{-1/2}a \label{eq:Ecreation}\\
         E_{i,j} &= \bra{i} (n+1)^{-1/2} a \ket{j} \\
                 &= \bra{i} (n+1)^{-1/2} \sqrt{j} \ket{j-1}
         \intertext{By expanding $ (n+1)^{-1/2} $ into a power series of $ n $ it can
         be shown that $ (\hat{n}+1)^{-1/2}\ket{m} = (m+1)^{-1/2} \ket{m} $.}
         &= \bra{i} \frac{\sqrt{j}}{\sqrt{j}} \ket{j-1} \\
         &= \bra{i} \ket{j-1}
      \end{align}
      So, the $ i $,$ j $th element is only non-zero for those values of $ j = i+1 $.
      The eigenvalue associated with the $ \ket{n} $ Fock state of the $ E $ operator
      is 1 for all $ n $. So, this operator can be written as
      \[
         E = \sum_{n=0}^{\infty} \ket{n}\bra{n+1}
      \]
      since this operator has the same action on the (complete) basis of Fock states
      as that given in Eq.~\ref{eq:Ecreation}.
   \end{homeworkSection}
   \begin{homeworkSection}{b)}
      \begin{align}
         E \ket{\phi} &= \sum_{n=0}^{\infty} \ket{n}\bra{n+1} \sum_{m}^{\infty}
         e^{i m \phi} \ket{m} \\
         &= \sum_{n=0}^{\infty} e^{i (n+1) \phi} \ket{n} \\
         &= e^{i \phi} \sum_{n=0}^{\infty} e^{i n \phi} \ket{n} \\
         &= e^{i \phi} \ket{\phi}
      \end{align}
      \begin{align}
         \braket{\phi_{1} | \phi_{2}} &= \sum^{\infty}_{m=0} e^{-i m \phi_1}
         \bra{m} \sum^{\infty}_{n=0} e^{i n \phi_{2}} \ket{\phi_{n}} \\
         &= \sum^{\infty}_{m = 0} e^{i m \left( \phi_2 - \phi_1 \right)}
         \braket{m | m} \\
         &= \sum^{\infty}_{m = 0} e^{i m \left( \phi_2 - \phi_1 \right)} \\
      \end{align}
      So, for $ \theta_2 = \theta_1 $ this sum diverges. For $ \theta_2 \ne
      \theta_1 $ this is not zero. So, these states are neither orthogonal nor
      normalizable.
   \end{homeworkSection}
   \begin{homeworkSection}{c)}
      \begin{align}
         P(\phi) &= \frac{1}{2 \pi} \braket{\phi | \rho_{\ket{n}} | \phi} \\
                 &= \frac{1}{2 \pi}
         \left(\sum^{\infty}_{i=0} e^{-i n \phi} \bra{i}\right)
         \ket{n} \bra{n}
         \left(\sum^{\infty}_{j=0} e^{j n \phi} \ket{j}\right) \\
         &= \frac{1}{2 \pi}
      \end{align}
      %The uncertainty of the phase can be calculated as $ \sqrt{\braket{\phi^{2}} -
      %\braket{\phi}^{2}} $.
      %\begin{align}
      %\braket{\phi} &= \braket{n | \phi | n } \\
      %&= \braket{n | \sum^{\infty}_{j=0} \ket{j} \bra{j+1} | n} \\
      %&= 0
      %\end{align}
      %\begin{align}
      %\braket{\phi^{2}} &= \braket{n | \phi^{2} | n } \\
      %\end{align}
      The uncertainty (variance) of the phase can be calculated as $ \left(\int_{0}^{2
            \pi} P(\phi) \phi^{2} d \phi - \left( \int_{0}^{2 \pi} P(\phi) d \phi
      \right)^{2}\right)^{1/2}$. Since $ P(\phi) = \frac{1}{2\pi}$ both integrals are easy
      to calculate.
      \[
         \int_{0}^{2 \pi} P(\phi) \phi^{2} d \phi  =
         \frac{1}{2\pi}\frac{(2\pi)^3}{3} = \frac{(2\pi)^2}{3}
      \]
      and
      \[
         \int_{0}^{2 \pi} P(\phi) \phi d \phi  =
         \frac{1}{2\pi}\frac{(2\pi)^2}{2} = \pi \enskip.
      \]
      So, the variance is
      \[
         \sigma_{\phi}^{2} = \frac{(2\pi)^2}{3} - \pi^2 = \frac{\pi^2}{3} \enskip.
      \]
      This is the same variance as that of a uniform distribution for some random
      variable X uniformly distributed over $ [a,b] $ which is $ \sigma^2_{X}(b-a)^2/12 $.

      This, and the fact that the probability distribution over $ \phi $ has no
      dependence on $ \phi $ indicates that the uncertainty of the phase for a Fock
      state is uniform over the phase (it's completely uncertain).

      To calculate the same uncertainty for the coherent states I can repeat the
      above approach where $ P(\phi) $ is now given by the corresponding expression
      for coherent states.
      %\begin{align}
      %P_{\alpha}(\phi) &= \frac{1}{2 \pi} \braket{\phi | \alpha | \phi}  \\
      %=&
      %\frac{1}{2 \pi}
      %\left(\sum^{\infty}_{n=0} e^{-i n \phi} \bra{n}\right)
      %\left(\sum^{\infty}_{k=0} e^{{-r^2}/2} \frac{r^k e^{i k \theta}}{\sqrt{k!}}
      %\ket{k}\right)
      %\left(\sum^{\infty}_{l=0} e^{{-r^2}/2} \frac{r^l e^{-i l \theta}}{\sqrt{l!}}
      %\bra{l}\right)
      %\left(\sum^{\infty}_{m=0} e^{i m \phi} \ket{m}\right) \\
      %&=
      %\frac{1}{2 \pi}
      %\left(\sum^{\infty}_{n=0} e^{-i n \phi}
      %e^{{-r^2}/2} \frac{r^n e^{i n \theta}}{\sqrt{n!}}
      %\right)
      %\left(\sum^{\infty}_{m=0} e^{{-r^2}/2} \frac{r^m e^{-i m \theta}}{\sqrt{m!}}
      %e^{i m \phi}\right) \\
      %&=
      %\frac{1}{2 \pi}
      %e^{-r^{2}}
      %\left(\sum^{\infty}_{n=0} \frac{r^n e^{i n (\theta- \phi)}}{\sqrt{n!}} \right)
      %\left(\sum^{\infty}_{m=0} \frac{r^m e^{-i m (\theta- \phi)}}{\sqrt{m!}} \right)
      %\intertext{This product can be split into two separate sums, one in which
      %$ n=m $ and another in which $ n\ne m $.}
      %&=
      %\frac{e^{-r^{2}}}{\pi}
      %\left(\sum^{\infty}_{n=0} \cos\left(n\left(\theta-\phi\right)\right) \frac{r^{2n}}{n!} \right) +
      %\frac{e^{-r^{2}}}{2 \pi}
      %\left(\sum^{\infty}_{n=0} \sum^{\infty}_{m=0, m \ne n}
      %\frac{r^{m+n}}{\sqrt{m!n!}}
      %\left( e^{i n \left( \theta- \phi \right)} + e^{-i m \left( \theta- \phi \right)} \right)
      %\right)
      %\end{align}
      \begin{align}
         P_{\alpha}(\phi) &= \frac{1}{2 \pi} \braket{\phi \ket{\alpha} \bra{\alpha}
         \phi} \\
         & \propto | \braket{\alpha | \phi} |^2 \\
         &= | \sum^{\infty}_{n=0}  \braket{\alpha \ket{n} \bra{n} \phi} |^2 \\
         &= | e^{{-r^{2}}/2} \sum^{\infty}_{n=0}
         \frac{r^{n}e^{-i n \theta}}{\sqrt{n!}} e^{i n \phi}|^2
         \quad,\text{where $ \alpha = r e^{i \theta} $.} \\
      \end{align}
      This sum does not have a closed form. However, we can consider this sum in
      the limit of large $ r $. For this, we first must realize that the sum
      resembles a Poisson distribution over $ r $. However, for Poisson
      distributions for which the value of $ r $ is large this distribution
      (which is defined over discrete values of $ n $) is well-approximated by a
      Gaussian distribution\footnote{One can refer to any textbook on statistics
      for this transformation. A particular reference is
   \url{http://www.roe.ac.uk/japwww/teaching/astrostats/astrostats2012_part2.pdf}}
   (which is defined over continuous values of $ x $)
      as
      \[
         P(n) \propto \frac{\lambda^{n} e^{- \lambda}}{n!} \approx \frac{1}{\sqrt{2
         \pi \lambda}} \exp(-\frac{(x- \lambda)^2}{2 \lambda}) \enskip.
      \]
      We can now substitute the sum for an integral and the summand by the
      square root of the Gaussian distribution. It's important to note that for
      large values of $ r $ not only is the distribution well-approximated by a
      Gaussian distribution but the bounds can extend over all x (from $ -
      \infty \to \infty$) since the mean of the Gaussian is given by $ r^{2} $, the
      distribution narrows as $ 1/r^{2} $ and the tails converge to zero within a
      few standard deviations from the mean. Thus, the integrand will be
      negligibly small for values of $ x $ close to zero. Performing the
      integral and computing the norm square of the result gives the uncertainty
      distribution over $ \phi $.
      \begin{align}
         P(\phi) &\propto \left| \int_{-\infty}^{\infty} \frac{1}{(2 \pi r^{2})^{.25}}
         e^{- \frac{\left( x - r^2 \right)^2}{4 \lambda}} e^{i x (\phi
         -\theta)} dx \right|^2
         \intertext{Solving this integral using a computer algebra system
         (stressed for time) yields}
         &=  2 \sqrt{2 \pi } r e^{-2 r^2 (\theta -\phi )^2} \label{eq:gaussian}
      \end{align}
      The code which computed this result is:
      \begin{listing}
         \caption{Gaussian Fourier Transform}
         \begin{minted}[bgcolor=bg,escapeinside=||]{mathematica}
a = Power[2 |$\pi$| r^2, -1/4]
   Integrate[Exp[-(x - r^2)^2/(4*r^2)] Exp[ I x (|$\theta$| - |$\phi$|)], {x, -Infinity, Infinity}];
Simplify[a*Conjugate[a], Assumptions -> {|$\theta$|, |$\phi$|, r} |$\in$| Reals && r > 0]
         \end{minted}
      \end{listing}
      Now, to determine the uncertainty associated with the phase we need to
      compute the following quantity:
      \[
         \int_{0}^{2 \pi} P(\phi) (\phi^{2}) d \phi -
         \left(\int_{0}^{2 \pi} P(\phi) \phi d \phi\right)^2
      \].
      I tried using Mathematica to calculate this quantity. However, the
      expression is so ugly that I don't think it's worth showing it, here.
      Another way to think about the variance of this distribution is to realize
      that this distribution over phase is centered at $ \theta $ and has a
      width of $ \frac{1}{4 r^2} $. This can be seen immediately by rewriting $
      P(\phi)$ by inserting all constants and refactoring.
      \begin{equation}
         P(\phi) = \frac{1}{2 \pi} 2 \sqrt{2 \pi } r e^{-2 r^2 (\theta -\phi
         )^2}  \sqrt{\frac{1}{2 \pi \frac{1}{4 r^2}}} e^{-\frac{(\theta -
      \phi)^2}{2 \frac{1}{4 r^2}}}
      \end{equation}
      This is clearly a Gaussian with variance $ \frac{1}{4 r^2} $ and mean $ \theta $.
      Now, $ r^2 $ is the squared magnitude of $ \alpha $. That is, it's the
      mean number of photons associated with a state $ \ket{\alpha} $. So, the
      uncertainty in the phase decreases with increasing $n$ as $ \frac{1}{4 n} $.
   \end{homeworkSection}
\end{homeworkProblem}
