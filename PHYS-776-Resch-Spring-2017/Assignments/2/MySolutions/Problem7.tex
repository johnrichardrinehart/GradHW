% Problem 7
\begin{homeworkProblem}
    Calculating the Wigner function for the state $ \frac{1}{\sqrt{2}} \left(
    \ket{0} + \ket{3} \right) $ involves first calculating the characteristic
    function.
    \begin{align}
        \tilde{W}(\eta) &= \Tr(\rho e^{\eta a^\dagger - \eta^* a}) \\
                        &= e^{- \left| \eta \right|^2/2} \Tr(e^{-\eta^* a} \rho
        e^{\eta a^\dagger}) \\
        &= e^{-\left| \eta \right|^2/2} \left( \bra{0} e^{\eta a^\dagger} e^{-
    \eta^* a} \ket{0} + \bra{0} e^{\eta a^\dagger} e^{- \eta^* a} \ket{n} +
    \bra{n} e^{\eta a^\dagger} e^{- \eta^* a} \ket{0} +
    \bra{n} e^{\eta a^\dagger} e^{- \eta^* a} \ket{n}
\right)
    \end{align}
    Now, each of these terms can be considered in the following general form.
    \begin{align}
        \bra{l} e^{\eta a^\dagger}e^{-\eta^*a} \ket{m} &=
        \bra{l} e^{\eta a^\dagger} \sum_{k=0}^m \frac{(-1)^k
        (\eta^*)^k \sqrt{m!}}{k!\sqrt{(m-k)!}} \ket{m-k} \\
        &= \bra{l}
        \sum_{j=0}^\infty \sum_{k=0}^m \frac{(-1)^k (\eta^*)^k \sqrt{m!}}{k!
            \sqrt{(m-k)!}} \frac{(\eta)^j \sqrt{(m-k+j)!}}{j! \sqrt{(m-k)!}}
            \ket{m-k+j}
    \end{align}
    So,
    \[
        \bra{0}e^{\eta a^\dagger}e^{-\eta^* a} \ket{0} = 1
    \]
    since $ m=0 $ (so $ k=0 $) and since $ l=0 $.
    \[
        \bra{0}e^{\eta a^\dagger}e^{-\eta^* a} \ket{n} =  \frac{(-1)^n
        (\eta^*)^n}{\sqrt{n!}}
    \]
    since $ m=n $ and $ l=0 $.
    \[
        \bra{n}e^{\eta a^\dagger}e^{-\eta^* a} \ket{0} =  \frac{
        (\eta)^n}{\sqrt{n!}}
    \]
    since $ m=0 $ (so $ k=0 $) and $ l=n $.
    \[
        \bra{n}e^{\eta a^\dagger}e^{-\eta^* a} \ket{n} =
        \sum_{i=0}^{n} \frac{(-1)^i (\eta^*)^i \eta^i n!}{j!^2 (n-j)!}
        = \sum_{i=0}^{n} \binom{n}{j} \frac{(-1)^i (\eta^*)^i \eta^i}{j!}
        = \sum_{i=0}^{n} \binom{n}{j} \frac{(-1)^i |\eta|^{2i}}{j!}
    \]
    Now,
    \[
        \tilde{W}(\eta) = e^{- \left| \eta \right|^2/2} \left( 1 +
            \frac{(\eta)^n + (-\eta^*)^n}{\sqrt{n!}} + \sum_{j=0}^n \binom{n}{j} \frac{(-1)^j
    \left| \eta \right|^{2j}}{j!} \right)
    \]
    where $ n=3 $ for the state in this problem. The Wigner function is
    \begin{align}
        W(\alpha) &= \frac{1}{\pi^2} \int d^2\eta e^{\eta^* \alpha - \eta
        \alpha^*} \tilde{W}(\eta)
        \intertext{Given the nature of the characteristic function it would
            probably be a good idea to express this integral in polar form.
            Expressing $ \eta \equiv r e^{i \theta} $ and $ \alpha \equiv a e^{i
        \phi} $ we can rewrite the integral as:}
        W(\alpha) &= \frac{1}{\pi^2} \int r dr d\theta e^{2 i a r \sin(\phi-\theta)}
        e^{-r^2/2}\left(1 + r^n \frac{e^{i \theta n} + (-1)^ne^{-i \theta
        n}}{\sqrt{n!}} + \sum^{n}_{j=0} \binom{n}{j} \frac{(-1)^j
r^{2j}}{j!}\right)
    \intertext{Using Mathematica to evaluate this integral for $ n=3 $ yields:}
    &= \frac{8}{3 \pi} a^2 e^{-2 a^2} \left(8 a^4-18 a^2+2 \sqrt{6} a \cos (3 \phi )+9\right)
    \end{align}
    This Wigner function is plotted below in Fig.~\ref{fig:Problem7a}.

    \begin{figure}[ht]
        \centering
        \input{Problem7a.tex}
        \caption{Wigner plot of $ \ket{\psi} = \frac{1}{\sqrt{2}} (\ket{0} +
        \ket{3}) $.}
        \label{fig:Problem7a}
    \end{figure}

    For fun, I've computed some other star plots. However, there is no
    analytical solution for general $ n $. I won't include the plots, here. But,
    for reference, I am including a numerical integration routine that will
    solve for arbitrary $ n $ and plot it in Julia, a high performance
    programming language designed for scientific computing. We recommend the
    reader install and run this script to become familiar with the power of
    Julia (the ease of MATLAB with the speed of C).

    \begin{minted}{julia}
import Plots
import Cubature
import Interpolations
import Base.Iterators
Plots.plotly()

# characteristic(n) returns the Wigner characteristic function for a star state of
# the form |0>+|n>
@everywhere function characteristic(n)
    # x is a vector of [r,theta] values
    return function f(x)
        e^(-x[1]^2/2)*
        (1
         +
         ((x[1]*e^(im*x[2]))^n+(-x[1]*e^(-im*x[2]))^n)/sqrt(factorial(n))
        +
        sum(j -> (-1)^j*x[1]^(2*j)*binomial(n,j)/factorial(j),0:n)
        )
    end
end

ch = RemoteChannel(1) #ch -> # of solved points in the α grid
put!(ch,1)
#init = RemoteChannel(1) # t -> a channel that stores the computing time
@everywhere init = now()
#put!(t,now())
#@everywhere st = now()

# integrand(n,x,y,c,s) takes 5 arguments:
# n - the n in |0>+|n>
# x, y - the x,y values of alpha = x+iy
# c - a channel storing the number of solved points
@everywhere function integrand(n,x,y,c;print_status=false)
    if print_status
        cnt = take!(c)
        duration = (now()-init).value/1000
        println("Working on point: (",x,",",y,"). ",
        round(100*cnt/(length(xs)*length(ys)),1), "%.")
        if duration > 60
            println("On point ",cnt," of ", length(xs)*length(ys),
            " in time ", round(duration/60,1), " minutes.")
        else
            println("On point ",cnt," of ", length(xs)*length(ys),
            " in time ", duration, " seconds.")
        end
        put!(c,cnt+1)
    end
    function f(X)
        p = atan2(y,x)
        a = sqrt(x^2+y^2)
        return real((1/(pi^2))*X[1]*e^(2*im*a*X[1]*sin(p-X[2]))*characteristic(n)(X))
    end
    return f
end

# solution grid
@everywhere xs = ys = -5.5:.2:5.5

# the cartesian product of all of the xs and ys
grid_idxs = Iterators.product(xs,ys)
# the solution grid (where the results go)
grid_vals = zeros(size(grid_idxs))

n = 6
@everywhere integral =
l -> Cubature.hcubature(integrand(n,l[1],l[2],ch),[0,0],[10,2*pi],abstol=1e-3)[1]
println("Starting the integrals.\n") # starting the integrals at each alpha = x+iy
grid_vals = pmap(integral, collect(grid_idxs)) # pmap is a parallel map (uses workers)

# interpolate the solutions to smooth it out.
itp = Interpolations.interpolate(grid_vals,
                  Interpolations.BSpline(Interpolations.Cubic(Interpolations.Line())),
                  Interpolations.OnGrid()
                 )
# scale the interpolation according to the actual grid
sitp = Interpolations.scale(itp, xs, ys)
# plot the interpolated data
p = Plots.plot(
               linspace(xs[1],xs[end],500),
               linspace(ys[1],ys[end],500),
               (x,y) -> sitp[x,y],
               t=:surface
              )
    \end{minted}
    It's clear in the polar form that the state $ \frac{1}{\sqrt{2}} \left(
    \ket{0} + \ket{n} \right) $ would have $ n $ peaks and $ n $ troughs. This
    appears in the angle which arises from the complex terms in $ W(\alpha) $.
\end{homeworkProblem}
