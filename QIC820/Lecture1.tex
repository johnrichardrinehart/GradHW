\date{September 15, 2015}

\begin{document}
\begin{enumerate}
    \item{Course web page : cs.uwaterloo.ca/~watrous/CS766/}
    \item{Office hours: Tuesday 1-3 @ QNC 3312 (or whenever)}
    \item{Email: John Watrous (watrous@uwaterloo.ca)}
\end{enumerate}
\begin{section}[Introductory Course Information]
    Grades wil be split up with 80% problem sets (of which there are 4)
    and 20% final project.

    Dr. Watrous has written a book posted on the course web page. Send
    errata to his email. Notes from the 2011 offering of the course are
    also on the course website.
    \begin{subsection}[Course Overview]
        Course addresses the mathematics of quantum information. Focuses
        on fundamental concepts and methods.  Application is not a focus
        of this course. This course will focus on quantum algorithms and
        ompelexity, quantum cryptography, error correction and fault
        tolerance, etc. There will be some discussion of Physics
        (how condensed matter relates to quantum information).
    \end{subsection}
    \begin{subsection}[Summary of Topics]
        \begin{enumerate}
            \item{Basic notions: states, measurements, channels}
            \item{Distance and similarity of states (trace distance or
                    fidelity, diamond norm/completely bounded trace
                norm)}
            \item{Quantum Entropy and related things (source coding,
                    Holevo's theorem). We'll prove a fundamental fact of
                quantum entropy: strong subadditivity)}
            \item{Entanglement Theory. What does it mean to be entangled
                    (separability), LOCC paradigm (distant labs
                paradigm), bound entanglement}
            \item{Random topics: Semidefinite programming, Harr
                measure, others.}
        \end{enumerate}
    \end{subsection}
    \begin{subsection}[What we should know]
        Read the first chapter of the book (Mathematical Preliminaries).

        We should ask questions. There will be sessions to ask the
        course TA ``dumb questions''.
    \end{subsection}
\end{section}
\begin{section}[Actual Content: Basic Notions of Quantum Information]
    Introducing the concept of a register: The intuitive understanding
    of a register is an abstraction of a physical system that stores a
    finite amount of information. For example, a qubit is a register. A
    qutrit or a collection of qubits or a quantum computer (no size
    restriction) are all, also, examples of registers.

    He's not going to try to define ``information'' or a ``physical
    system''. This is an ``intuitive definition''.

    Note that a finite system is a system comprising a finite number of
    elements/symbols (\alpha, \beta, \gamma, \cdots or a, b, c, \cdots
    or \sigma_0, \sigma_1, \sigma_2, \cdots).

    Imagine you have access to two registers and you place both on a
    table. You could consider both registers as a single register. This
    is similar to how multiple objects (e.g. bowling balls) can be
    considered as a single objects by considering their center of mass,
    moment of inertia, etc.

    So, a register X is either:
    a) a simple register (X = $\Sigma$ for some alphabet $\Sigma$

    b) a compound register (tuple of registers: X = (Y_1,, \cdots, Y_n)
    for Y_1,\cdots,Y_n being registers. Of course, X can not be a
    self-referential register. Y_1 and Y_2 and \cdots Y_n should be
    well-defined. X is not equal to $\Sigma$, it's identified with the
    set of states that it can store which is $\Sigma$. $\Sigma$ is the
    set of states that can be stored in the register.

    Example of a register: Allow $\Sigma = {0,1}$ and $\Gamma =
    {1,\cdots,n}$ for $n element of Z$. Z_1 = $\Sigma$,\cdots, Z_n =
    \Sigma. W = (Z_1,\cdots,Z_n$. Y = \Gamma$. X = (Y,W) =
    (Y,(Z_1,\cdots,Z_n)). I could draw a tree structure that relates
    these things. X is top node. Y and W are chldren nodes of X.
    Z_1,\cdots,Z_n are children nodes of W.

    \begin{subsection}[Defining a classical state set]
        The classical state of a register X is defined in this way:
        1. 
        If X is a simple register X = \Sigma then the classical state set
        is $\Sigma$. 
        If X = (Y_1,\cdots,Y_n) is a compound register then the
        classical state set of X is $\Sigma = Cartesian product of classical
        state sets of Y_\alpha$. This is the reason why the tensor product
        is identified as the way in which to combine quantum states.

        We can think about states in a different way than just defining the
        ``bits'' of a register. We can think about states in a probabilistic
        sense, too. This could happen, for example, if we wanted to consider
        the effect of noise on a state.

        A probabilistic state of a register X having clasical state set
        $\Sigma$ is a probability vector, $p \element P(\Sigma)$.

        p(a) is interpereted as the probability that X is in the classical
        state $a \element \Sigma$.

        Note that if $\Sigma$ is an alphabet then $C^\Sigma$ (all of the
        functions from $\Sigma$ to the complex numbers) or $R^\Sigma$ are
        Euclidean vector spaces over particular fields. We can define a norm
        and/or an inner product on these spaces.

        Note: P(\Sigma) means {p\element R^\Sigma such that \Sum_{p\Element
        \Sigma} P(x) = 1} // MISSING SOMETHING HERE
    \end{subsection}
\end{subsection}
\begin{subsection}[Some Underlying Mathematics]
    $C^\Sigma$ is a complex Euclidean space.
    An inner product on $C^\Sigma$ is defined as: for $u,v \element
    \Sigma: <u,v> = \Sigma_{a\element Sigma} \overbar{u(a)}v(a)$. This
    definition of an inner pdoruct is used to define the Euclidiean
    norm: $||u|| = \sqrt{<u,u>} = \sqrt{\Sigma_{a\element\Sigma}
    |u(a)|^2}$. Typical names for these spaces are x,y,z,w,\cdots

    Every register X has associated with it some complex euclidean space,
    \script{X}. Typically, we will adopt the convention that registers
    will be in arial font and the complex space associated with it will
    be in script font. $script{x} = C^\Sigma$ where $\Sigma$ is the
    classical state set of X.

    For complex Euclidean spaces \script{x} and \script{y} the set of
    all linear maps from \script{x} to \script{y} will be denoted as
    L(\script{x},\script{y}). We also use L(\script{x}) to denote all
    the linear maps from $\script{x}$ to itself $L(\script{x}) =
    L(\script{x},\script{x})$. A linear map will be
    referred to as an operator in this course. Of course, all the
    operators we discuss will be linear operators.

    The standard basis: Most of the information in this course is
    basis-dependent. If I have a space $\script{x} = C^\Sigma$ then for
    each a \element \Sigma, e_a is this vector:
    e_a(b) = 1 if b = a or 0 if b \ne a. \\ USE CASES BEFORE HERE
    i.e. this basis is an orthonormal basis.

    E_A = \KET{A} IN dirac notation. {e_a: a \element\Sigma} is the
    standard basis of $\script{x}$.
    
    Having picked a standard basis for scriptx and scripty we can
    associate linear operatosr with matrices. A \element L(scriptx,
    scripty) (scriptx is C^\Sigma and scripty is C^\Gamma) then M_{A}e(a,b)
    = <e_b,A e_a> = <b|A|a> for all a\element \Gamma and b\element Sigma

    However, int he future M_A wil be denoted as A(a,b) to mean <e_b,A
    e_a> instead of M_A (a,b).

    If I have a  linear opetar of the form A\elementL(\scriptx) then we
    define the trace to be $Tr(A) = \Sigma_{a \element Sigma} A(a,a)$

    If A\element L(\scriptx,\scripty) then we define A^* \element
    L(\scripty,\scriptx) as the unique operator such that <y,A x> =
    <A^*y,x> for all x \element \scriptx, y \element \scripty. This is
    the conjugate transpose of A: $A^*(b,a) = \overbar{A(a,b)}$ for all
    a\element Gamma, b \element \Sigma. This is also known as the
    adjoint of A. Typical notation: $A^*$ can be written as $A^\dagger$.

    An operator A\elementL(\scriptx) is positive semidefinite if and
    only if (iff) A = B*B f or some choice of B \element L(\scriptx).
    This is only one of many formulations of a positive semidefinite operator.
    Another way to define it is as a hermitian operator whose
    eigenvalues are all non-negative. A hermitian operator being one
    that has the property that A = A^*.
\end{subsection}
\begin{subsection}[Defining/Describing Quantum States]
    A quantum state of a register X is a positive semidefinite, trace 1
    operator. $p \element L(\scriptx)$ for $\scriptx$ being the complex
    Euclidean space associated with X. $p \element L(\scriptx)$ for
    $\scriptx$ being the complex Euclidean space associated with X. $p
    \element L(\scriptx)$ for $\scriptx$ being the complex Euclidean
    space associated with X. $\rho \element L(\scriptx)$ for $\scriptx$
    being the complex Euclidean space associated with X. $\rho$ is
    also known as a density operator. We use the following notation:
    D(\scriptx) = {\rho \element L(\scriptX): \rho is positive
    semidefinite and Tr(\rho) = 1}.
\end{subsection}
\end{section}
\end{document}
