\begin{document}
\begin{section}[Correction to Last Lecture]
    \scripttx = \mathc^\Sigma, \scripty = \mathc^\Gamma
    A\elementof L(\scriptx,\scripty)
    M_A: \Gamma x \Sigma -> \mathc (complex numbers)
    M_A (a,b) = <e_a,Ae_b> for all a \elementof \Gamma, b \elementof
    \Sigma
    Identify A with M_A, so A(a,b) = <e_a, A e_b> = <a|A|b> for a
    \elementof \Gamma, b \elementof \Sigma
\end{section}
\begin{section}[Lecture 2 Material]
    Additional notation: Suppose that we have \scriptx = \mathc^\Sigma
    and L(\scriptx), L(\scriptx) being all linear operators of the form
    A:\scriptx \rightarrow \scriptx
    e.g. Herm(\scriptx) = {H\element L(\scriptx): H = H^*}
    e.g. Pos(\scriptx) = {P \element L(\scriptx): P = A^*A for
        e.g. PD(\scriptx) = {\rho \element Pos(\scriptx): Tr(\rho) = 1}
        e.g. PU(\scriptx) = {U \element L(\scriptx): U*u = \math1_\scriptx}
    A\elementof L(\scriptx}

    e.g. Pos(\scriptx) = {P \element L(\scriptx): u^*Pu \ge 0 for all u
    \element \scriptx}
    Where, if u \element \scriptx, then u^* is the linear map u^*v =
    <u,v> (it
    is the adjoint of the linear map \alpha \rightarrow \alpha u)
    u \element \scriptx, u \element L (\mathc, \scriptx), u^* \element
    L(\scriptx,\mathc)


    If we have a register X whose associated complex Euclidean space is
    \scriptx, then a state of X is a density opoerator \rho \element
    D(\scriptx).

    Important fact: D(\scriptx) is a convex set. This means that if
    \rho_0, \rho_1 are two density operator and \lamda \element [0,1]
    then: \lamda \rho_0 + (1-\lambda)\rho_1 \element
    D(\scriptx). It's easy to show that the trace is 1 and that the
    result is positive semi-definite. Such a density operator could be
    constructed by considering an experiment whereby a coin flip,
    outcome weighted by lambda, determines whether or not you choose
    \rho_0 or \rho_1 to represent the state.

    Typically, states are introduced as unit vectors in introductory QM
    courses (pure states). However, with a density operator formalism
    you can consider pure states as a special case of density operators.
    Pure states ar states of the form uu^* for some unit vector u 
    \element \scriptx.

    Consider S(\scriptx) = {u\element \scriptx: ||u|| = 1}. This is the set
    of all unit vectors.

    The pure states are the extreme points on the convex set
    D(\scriptx). There would be no way to write a pure state as a convex
    combination of states.

    The set of density operators is also a compact set (closed and
    bounded). The density operators have a ``solid boundary''. The
    sequence of density operators converges to a density operator. The
    convex set is equal to its hull at extreme points. So, the density
    operator is a convex combination of pure states.

    More generally, a convex combination of density operators is
    \Sigma_{a\alement\Gamma}p(a)\rho_a for some arbitarary alphabet
    \Gamma, a probability vector p \element P(\Gamma), and {\rho_a : a
    \element Gamma} \properlycontainedin D(\scriptx). In this class, convex
    combination is finite. 

    \begin{Theorem}[Spectral Theorem - Unproved, just stated]
        Allow A \element L(\scriptx) to be a normal operator, let n =
        dim(\scriptx), and let \lambda_1,\cdots,\lambda_n be the
        eigenvalues of A. The theorem results in the fact that there
        exists an orthonormal basis {x_1,\cdots,x_n} \containedin
        \scriptx such taht A = \Sigma_{k=1}^n \lambda_k x_k x_k^*p.

        {x_1,\cdots,x_n} is a spectral decomposition of A. (A normal
        operator is one that has the property that AA^* = A^*A)

        Warning: This theorem only holds for NORMAL operators! For
        example {0 1; 0 0} is not a normal operator and can not use the
        spectral decomposition theorem. You can always write any
        operator A as A = H + i K, H,K \element Herm(\scriptx)
        H = \frac{A + A^*}{2}, K = \frac{A - A^*}{2i}. For A to be
        normal can also be stated as saying that [H,K] = HK-KH = 0.

    \end{Theorem}

    Consider compound registers. X = (Y_1,\cdots,Y_n). The set of
    classical states corresponding to X (the classical state set of X)
    is \Sigma =\Gamma_1 x \cdots x \Gamma_n (cartesian product), where
    \Gamma_1,\cdots,\Gamma_n are the classical state sets of
    Y_1,\cdots,Y_n. The complex Euclidean space associated with X is
    \scriptx = \mathc^\Sigma = \mathC^{\Gamma_1 x \cdots x \Gamma_n} =
    \mathC^\Gamma_1 \tensor \cdots \tensor \mathc^\Gamma_n = y_1 \tensor
    \cdots \tensor y_n.

    In general, the tensor product of vectors y_1 \element
    \scripty_1,\cdots, y_n \scripty_n is the (y_1\tensor \cdots \tensor
    y_n)(a_1,\cdots,a_n) = y_1(a_1)\cdots y_n(a_n). Not every element of
    \scripty_1 \tensor \cdots \tensor \scripty_2 can be written as y_1
    \tensor \cdots \tensor y_n, but \scripty_1 \tensor \cdots \tensor
    \scripty_n is spannesd by these vectors.

    We have a Similar definition of the tensor product for operators.
    A_1 \elementof
    L(\scriptx_1,\scripty_1),\cdots,A_n\elementL(\scriptx_n,\scripty_n).
    The tensor product of these operators is the operator A_1 \tensor
    \cdots \tensor A_n \element L(\scriptx_1 \tensor \cdots \tensor
    \scriptx_n, \scripty_1\tensor \cdots \tensor \scripty_n) defined as
    (A_1\tensor \cdots \tensor
    A_n)((a_1,\cdots,a_n),b_1,\cdots,b_n))=A_1(a_1,b_1)\cdots
    A_n(a_n,b_n).

    Alternatively, A_1\tensor \cdots \tensorA_n is the unique operator
    for which (A_1A\tensor \cdots \tensorA,_n)(x_1\tensor \cdots \tensor
    x_n) = (A,x_1)\tensor \cdots \tensorA(A_n,x_n) for all x_1 \element
    \scriptx_1,\cdots,x_n \element \scriptx_n.

    Consider a state of X (compound register) of the form \rho = \rho_1
    \tensor \cdots \tensor \rho_n for \rho_1 \element
    D(\scripty_1),\cdots,\rho_n\element D(\scripty_n) is called a product
    state. It represents independence amoung the systems.

    Consider a simple example. \Gamma_1 and \Gamma_2 = {0,1} and \Sigma
    = Gamma_1 x Gamma_2 (cart prod) = {00,01,10,11}. Y_1 and Y_2 are
    qubits and X = (Y_1,Y_2) (the pair of qubits).

    Note: In general, E_{a,b} is the matrix with a 1 in the entry (a,b),
    0 everywhere else.  E_{0,0} = {1 0; 0 0} \element D(\scripty_1).

    u = \frac{e_1+e_2}{\sqrt{2}}, u u^* = .5 * {1 1; 1 1} \element
    D(\scripty_2).

    Let \rho_1 = {1 0; 0 0} and \rho_2 = .5 * {1 1; 1 1}. Then \rho_1
    \tensor \rho_2 = {.5 .5 0 0 ; .5 .5 0 0; 0 0 0 0 ; 0 0 0 0}.

    We cans also consider \rho = .5 {1 0 0 0; 0 0 0 0; 0 0 0 0; 0 0 0
    1}. It's also a valid density opeartor. It's a convex combination of
    .5 E_{0,0}\tensor E_{0,0} + .5 E_{1,1}\tensor E_[1,1}.

    a final example is rho = .5{1 0 0 1; 0 0 0 0; 0 0 0 0; 1 0 0 1} = v
    v^* for v = \frac{1}{\sqrt{2}} (e_0\tensor e_0+e_1 \tensor e_1) =
    \frac{|00>+|11>}{\sqrt{2}}.

    Imagine we have X = (Y_1,\cdots,Y_n) in a particular state \rho =
    D(\scriptx) = D(\scripty_1 \tensor \cdots \tensor \scripty_n).
    Remove Y_k. The state of (Y_1,\cdots,Y_{k-1},Y_{k+1},\cdots,Y_n) is
    uniquely determined by \rho and is Tr_{\scripty_k}(\rho) (the
    partial trace of \rho).

    The partial trace is Tr_{\scripty_k} is the unique linear map that
    satisfies Tr_{\scripty_k}(Y_1 \tensor \cdots \tensor Y_n) =
    Tr(Y_k)y_1\tensor \cdots\tensor y__{k-1} \tensor y_{k+1} \tensor
    \cdots \tensor y_n. \rho_1 \tensor \cdots \tensor \rho_n (taking
    away Y_k) must be \rho_1 \tensor \cdots \tensor \rho_{k-1} \tensor
    \rho_{k+1}  \tensor \cdots \tensor \rho_n. It can'b te that removing
    \rho_k changes the other states.

    Tr_{\scripty_k}(\rho) = \Sigma_{b_k}(\math1  \tensor \cdots \tensor
    \e_{bx}^* \tensor \math1) \rho (\math1 \tensor \cdots
    \tensor e_{bk} \tensor \cdots \tensor \math1)

    We can alse go in the other direction (X,Y). Suppose we have a state
    \rho \elementt D(\scriptx) of X. We may ask what states \sigma
    \element D(\scriptx \tensor \scripty) are consistent with \rho,
    meaning Tr_{\scripty}(\sigma) = \rho. In general, there are many
    states \sigma that accomplish this. It is intersting to consider the
    case when \sigma is a pure state.

    \begin{Theorem}[Operator vector correspondence]
        Let \scriptx and \scriptty be complex Euclidean spaces and let P
        \element Pos(\scirptx). A purifications u \element (\scriptx
        \tensor \scripty) exists if and only if dim(\scripty) \ge
        rank(P). The rank is the dimension of the image (the dimension
        of the space spanned by the row or column vectors in any
        representation).

        Let's introduce the operator vector correspondence. Imagine we
        have two complex Euclidean spaces \scriptx \element
        \mathc^\Sigma, \scripty \element \mathc^\Gamma. Define a linear
        map vec: L(\scriptx,\scripty) \rightarrow \scripty \tensor
        \scriptx as follows: vec(E_{a,b}) = e_a \tensor e_b.
        Alternatively, vec(\ket{a}\bra{b}) = \ket{a}\ket{b} for all a
        \element Gamma, b \element \sigma. e.g. vec({\alpha \beta;
        \gamma \delta} = {\alpha; \beta; \gamma; \delta}.
    \end{Theorem}

    Introducing some identities: 1. <vec(A),vec(B)> = <A,B> (``def''
    over equals) Tr(A^* B). 2. (A\tensor B)vec(C) = vec(A C B^T).
    3. A,B \element L(\scripty,\scriptx) then
    Tr_{\scripty}(vec(A)vec(B^*) = AB^* and
    Tr_{\scriptx}(vec(A)vec(B^*)) = (B^*A)^T. 4. vec(uv^*) = u\tensor
    \overbar(v) (overbar means element-wise conjugate)

    \begin{Theorem}
        Definition: Suppose \scriptx and \scripty are complex Euclidean
        spaces and that P\element Pos(\scriptx). A vector u \element
        \scriptx \tensor \scripty is a purification if Tr_{\scripty}(u
        u^*) = P. We would like to know under which conditions a
        purifications exists. Also, if it exists, what relationships can
        we draw relating it to other purifications? Thankfully, we can
        answer these questions.

        Proof: Assume a purification u \element \scriptx \tensor
        \scripty of P exists. Let's let A \element L(\scripty,\scriptx)
        be the unique operator with u = vec(A) (since vec is one-to-one
        and onto; bijective). We have P = Tr_{\scripty}(u u^*) =
        Tr_{\scripty}(vec(A)vec(^*)) = A A^*. Therefore rank(P) =
        rank(AA^*) \le rank(A) \le dim(\scripty). That's the first part
        of the proof.

        To prove the second part suppose that dim(\scripty) \ge rank(P). 

        Using the spectral theorem, we may write P = \Sigma _k=1}^r
        \lambda_k x_k x_k^* (where r is the rank(P)) for
        \lamba_1,\cdots,\lambda_r > 0. We don't care about 0 eigenvalues
        because they don't change the sum.  {x_1,\cdots,x_r} will be an
        orthonormal set by the spectral theorem. In general, for any
        Hermitian H \properlycontainedin Herm(\mathc^n) we order
        eigenvalues \lambda_1 \ge \lambda_2 \ge \cdots \ge \lambda_n.
        Choose {y_1,\cdots,y_r} to be any other normal set. we can do
        this because the dimension of \scripty is as big as the rank of
        P. Let's define A = \Sigma_{k=1}^r \sqrt{\lambda_k} x_k y_k^*
        \element L(\scripty,\scriptx) we have A AA^* = \Sigma_{k=1}^r
        \lambda_k x_k x_k^* = P. Theefore vec(A) is a purification of P.
        Thus, there exists a purifaction. QED.
\end{Theorem}
\end{section}
