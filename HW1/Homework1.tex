 \documentclass[17pt]{article}
\usepackage{fancyhdr} % Required for custom headers
\usepackage{lastpage} % Required to determine the last page for the footer
\usepackage{extramarks} % Required for headers and footers
\usepackage{graphicx} % Required to insert images
\usepackage{lipsum} % Used for inserting dummy 'Lorem ipsum' text into the template
\usepackage{amsmath,amsthm,amsxtra}

% Margins
\topmargin=-0.45in
\evensidemargin=0in
\oddsidemargin=0in
\textwidth=6.5in
\textheight=9.0in
\headsep=0.25in 

\linespread{1.1} % Line spacing

% Set up the header and footer
\pagestyle{fancy}
\lhead{\hmwkAuthorName} % Top left header
\chead{\courseTitle\ : \hmwkTitle} % Top center header
\rhead{\firstxmark} % Top right header
\lfoot{\lastxmark} % Bottom left footer
\cfoot{} % Bottom center footer
\rfoot{Page\ \thepage\ of\ \pageref{LastPage}} % Bottom right footer
\renewcommand\headrulewidth{0.4pt} % Size of the header rule
\renewcommand\footrulewidth{0.4pt} % Size of the footer rule

\setlength\parindent{0pt} % Removes all indentation from paragraphs

%----------------------------------------------------------------------------------------
%	DOCUMENT STRUCTURE COMMANDS
%	Skip this unless you know what you're doing
%----------------------------------------------------------------------------------------

% Header and footer for when a page split occurs within a problem environment
\newcommand{\enterProblemHeader}[1]{
\nobreak\extramarks{#1}{#1 continued on next page\ldots}\nobreak
\nobreak\extramarks{#1 (continued)}{#1 continued on next page\ldots}\nobreak
}

% Header and footer for when a page split occurs between problem environments
\newcommand{\exitProblemHeader}[1]{
\nobreak\extramarks{#1 (continued)}{#1 continued on next page\ldots}\nobreak
\nobreak\extramarks{#1}{}\nobreak
}

\setcounter{secnumdepth}{0} % Removes default section numbers
\newcounter{homeworkProblemCounter} % Creates a counter to keep track of the number of problems
\newcommand{\homeworkProblemName}{}
\newenvironment{homeworkProblem}[1][Problem \arabic{homeworkProblemCounter}]{ % Makes a new environment called homeworkProblem which takes 1 argument (custom name) but the default is "Problem #"
\stepcounter{homeworkProblemCounter} % Increase counter for number of problems
\renewcommand{\homeworkProblemName}{#1} % Assign \homeworkProblemName the name of the problem
\section{\homeworkProblemName} % Make a section in the document with the custom problem count
\enterProblemHeader{\homeworkProblemName} % Header and footer within the environment
}{
\exitProblemHeader{\homeworkProblemName} % Header and footer after the environment
}

\newcommand{\problemAnswer}[1]{ % Defines the problem answer command with the content as the only argument
\noindent\framebox[\columnwidth][c]{\begin{minipage}{0.98\columnwidth}#1\end{minipage}} % Makes the box around the problem answer and puts the content inside
}

\newcommand{\homeworkSectionName}{}
\newenvironment{homeworkSection}[1]{ % New environment for sections within homework problems, takes 1 argument - the name of the section
\renewcommand{\homeworkSectionName}{#1} % Assign \homeworkSectionName to the name of the section from the environment argument
\subsection{\homeworkSectionName} % Make a subsection with the custom name of the subsection
\enterProblemHeader{\homeworkProblemName} % Header and footer within the environment
}{
\enterProblemHeader{\homeworkProblemName} % Header and footer after the environment
}

%----------------------------------------------------------------------------------------
%	NAME AND CLASS SECTION
%----------------------------------------------------------------------------------------

\newcommand{\hmwkTitle}{Assignment 1} % Assignment title
\newcommand{\hmwkDueDate}{Wendesday,\ January\ 30,\ 2013} % Due date
\newcommand{\courseTitle}{Physics 760: Electricity and Magnetism} % Course/class
\newcommand{\hmwkClassInstructor}{Dr. Stefan Kycia} % Teacher/lecturer
\newcommand{\hmwkAuthorName}{John Rinehart} % Your name
\newcommand{\sudentNumber}{20503440} % Your name
\newcommand{\position}{PhD student at ECE departent}

%----------------------------------------------------------------------------------------
%%-----------------------------------------------------------------------------------------

%%%%%%%%%%
\newcommand{\red}[1]{\textcolor[rgb]{1,0,0}{#1}}
\newcommand{\blu}[1]{\textcolor[rgb]{0,0,1}{#1}}
\newcommand{\bs}[1]{\boldsymbol{#1}}
%\newcommand{\V}[1]{\bm{#1}}
\newcommand{\V}[1]{\Vec{#1}}
\newcommand{\A}[1]{\Hat{#1}}
\newcommand{\W}[1]{\widehat{#1}}
\newcommand{\T}[1]{\widetilde{#1}}
\newcommand{\pd}[2]{\dfrac{\partial #1}{\partial #2}}
\newcommand{\fpd}[2]{\frac{\partial #1}{\partial #2}}
\newcommand{\pds}[1]{\dfrac{\partial}{\partial #1}}
\newcommand{\fpds}[1]{\frac{\partial}{\partial #1}}
\newcommand{\pdss}[1]{\dfrac{\partial^2}{\partial {#1}^2}}
\newcommand{\pdsss}[2]{\dfrac{\partial^2}{\partial #1 \partial #2}}
\newcommand{\pdt}[2]{\dfrac{\partial^2 {#1}}{\partial {#2}^2}}
\newcommand{\pdtt}[3]{\dfrac{\partial^2 {#1}}{\partial {#2} \partial {#3}}}
\newcommand{\dif}[2]{\frac{d{#1}}{d{#2}}}
\newcommand{\vt}[1]{\Vec{\mathcal{#1}}}
\newcommand{\VP}[1]{\Vec{\mathbf{#1}}}
\newcommand{\vp}[1]{\mathbf{#1}}
\newcommand{\phas}[1]{\angle{#1}^{\circ}}
\newcommand{\er}{\epsilon_{r}}
\newcommand{\mr}{\mu_{r}}
\newcommand{\Lrw}{\Longrightarrow}
\newcommand{\refeq}[1]{(\ref{#1})}
\newcommand{\abs}[1]{\left| #1\right|}
\newcommand{\ket}[1]{|#1\rangle}
\newcommand{\bra}[1]{\langle #1| }
\newcommand{\bracket}[2]{\langle#1|#2\rangle }


%%%---
\newcommand\ointint{\begingroup
\displaystyle \unitlength 1pt
\int\mkern-7.2mu
\begin{picture}(0,3)
\put(0,3){\oval(10,8)}
\end{picture}
\mkern-7mu\int\endgroup}
%%%----
\providecommand{\abs}[1]{\lvert#1\rvert}
\providecommand{\norm}[1]{\lVert#1\rVert}

%%%%%%%%%%%%%%%

%---Packeges------------------------------------------------------------------
 %------------------------------------------------------
\usepackage[utf8]{inputenc}
\usepackage[T1]{fontenc}
\usepackage[english]{babel}
\usepackage[latin1]{inputenc}
\usepackage[T1]{fontenc}
\usepackage{pstricks}
\usepackage[usenames,dvipsnames]{pstricks}
\usepackage{epsfig}
\usepackage{pst-grad} % For gradients \usepackage{pst-plot} % For axes
\usepackage{pifont}
\usepackage{amsfonts}
\graphicspath{{IMG/}}
\usepackage[utf8]{inputenc}
 \usepackage[OT1]{fontenc}
 \usepackage[absolute,overlay]{textpos}
 \usepackage{graphicx}
 \usepackage[bookmarks=false,pdffitwindow]{hyperref}
 \usepackage{tikz}
 \usepackage{xcolor}
 \usepackage{calc}
\usepackage{chngcntr}
%----------------------------------------------------------------
\numberwithin{equation}{section}
\renewcommand{\theequation}{\arabic{equation}}



%%------------------------------------------------------------------------------------------
%	TITLE PAGE
%----------------------------------------------------------------------------------------

\title{
\vspace{2in}
\textmd{\textbf{\courseTitle\\ \vspace{0.5in}\hmwkTitle}}\\
\vspace{0.5in}\large{{\hmwkClassInstructor}}
\vspace{3in}
}
\author{\textbf{\hmwkAuthorName}\\ 
}\\
\date{January 1, 2000} % Insert date here if you want it to appear below your name

%----------------------------------------------------------------------------------------

\begin{document}

\maketitle


\newpage
\tableofcontents
\newpage


%----Problems-------
\begin{homeworkProblem}

\begin{homeworkSection}{a}
Charge multipole expansions are given in terms of an integral over the charge density as follows: \[ q_{lm} = \int\limits_{\text{all space}} Y_{lm}^*(\theta',\phi')r'^l \rho(\vec{x'})d\tau' \]

Now, this homework problem has a charge density which can be expressed as follows: \[ \rho(\vec{x'}) = \frac{q\delta(|\vec{x'}|)\delta(\theta'-\pi/2)}{r'^2sin^2\theta'} \big( \delta(\phi') + \delta(\phi'- \pi/2) - \delta(\phi - \pi) - \delta(\phi - 3\pi/2) \big) \]

Additionally, the spherical harmonics, $Y_{lm}$s, can be written in terms of associated Legendre polynomials as: \[Y_{lm}^*(\cos\theta) = \gamma_{lm}P_{lm}(\cos\theta)\exp(-im\phi) \]

Given all of this.

\begin{align*}
	q_{lm} = \int\limits_0^{2\pi} \int\limits_0^\pi \int\limits_0^\infty& Y_{lm}(\theta',\phi')r'^l \\ &\bigg( \frac{q\delta(|\vec{x'}|)\delta(\theta'-\pi/2)}{r'^2sin^2\theta'} \big( \delta(\phi') + \delta(\phi'- \pi/2) - \delta(\phi - \pi) - \delta(\phi - 3\pi/2) \big) \bigg) r'^2 sin^2\theta' \de r' \de\theta' \de\phi' \\
\end{align*}
\begin{align*}
	 &q_{lm} = q a^l Y_{lm}^*(\pi/2,0)+Y_{lm}^*(\pi/2,\pi/2) -Y_{lm}^*(\pi/2,\pi)-Y_{lm}^*(\pi/2,3\pi/2) \\
   &Y_{lm}(\theta,\phi) = \gamma_{lm} P_{lm}(\cos\theta)\exp(-i m \phi) \textit{,}\quad \textit{So,} \quad Y_{lm}(\pi/2,a) = \gamma_{lm} P_{lm}(0)\exp(-i m a)
\end{align*}
For a = $0,\pi/2,\pi,3\pi/2$ the expression for $\exp(-i m a)$ can be reduced as $1,(-i)^m,(-1)^m,i^m$, respectively. Given this, the expression for the multipoles can be recast as follows:

\[
    q_{lm} = q a^l \gamma_{lm} P_{lm}(0)\big( (1-(-1)^m) - (i^m - (-i)^m) \big)
\]

This can be easily seen to be zero for even $m$. Consider $f(m) =  (1-(-1)^m) - (i^m - (-i)^m $. $f(1) = 2-2i,\, f(3) = 2+2i,\, f(5) = 2-2i,\, f(7) = 2+2i,\, \cdots$. Additionally, though, $P_{lm}(x)$ is odd for any combination of $l$ and $m$ that is also odd. Since $m$ is constrained to be odd for nonzero $q_{lm}$ then $l$ must be constrained to be odd as well for nonvanishing $q_{lm}$ Thus, the final expression for $q_{lm}$ can be reduced:

\[
	q_{2j+1,2k+1} = 2q a^{2j+1} \big\{ 1+(-i)^{2k+1} \big\}\gamma_{1,1}P_{2j+1,2k+1}(0)
\]

To meet the problem demands
\begin{center}
	\begin{align*}
		q_{1,1} &= 2qa^3(1-i)\gamma_{1,1}P_{1,1}(0) = -\sqrt{\frac{3}{2\pi}}qa^3(1-i) \\
		q_{1,-1} &= 2qa^3(1-i)\gamma_{1,-1}P_{1,-1}(0) = \sqrt{\frac{3}{2\pi}}qa^3(1+i)
	\end{align*}
\end{center}

\end{homeworkSection}

\begin{homeworkSection}{b}
	In a similar fashion as before, we can construct $\rho(r,\theta,\phi) = \frac{q\delta(\phi')}{r'^2\sin^2\theta'}\big( \delta(\theta')\delta(|\vec{r'}-a\hat{z}|) + \delta(\theta'-\pi)\delta(|\vec{r'}+a\hat{z}|)- 2\delta(\theta')\delta(|\vec{r'}|)$. Thus, the $q_{lm}s$ are given by:
	\[
	\int\limits_{\text{all space}}\frac{qr'^l\delta(\phi')}{r'^2\sin^2\theta'} Y^*_{l,m}(\theta',\phi')\bigg(\delta(|\vec{r'}|-a)\delta(\theta')+\delta(|\vec{r'}|-a|)\delta(\theta'-\pi)-2\delta(r')\delta(\theta')\big)
	\]
	
	Reducing this expression yields:
	\[
	q_{lm} = -2q\delta_{l,0} + qa^l(Y^*_{l,m}(0,0)+Y^*_{l,m}(\pi,0))
	\]
	
	Since, $Y_{l,m}(\theta,\phi) = \gamma_{l,m}P_{l,m}(\cos\theta)\exp(-im\phi)$ then $Y_{l,m}(0,0)$ can be written as $\gamma_{l,m}P_{l,m}(1)$ and $Y_{l,m}(\pi,0)$ can be written as $\gamma_{l,m}P_{l,m}(-1)$.
	
	\[
	q_{l,m} = -2q\delta_{l,0}+qa^l\gamma_{l,m}(P_{l,m}(1)+P_{l,m}(-1))
	\]
	Now, $P_{l,m}$ is odd if $l+m$ is odd. Thus, in order to have nonvanishing $q_{l,m}$ it must be the case that the sum of $l$ and $m$ must be even. Additionally, it must be the case the $P_{l,m}(1) \text{\,and\,} P_{l,m}(-1) \ne 0$. Note that for all associated Legendre polynomials for which m is nonzero, $P_{l,m}(\pm 1) = 0$. This can be seen in that $P_{l,l} = (-1)^l(2l-1)!!(1-x^2)^{l/2}$. This is a restatement of the fact that this charge distribution exhibits azimuthal symmetry.  Thus, $q_{0,0}$ dies and \Leftrightarrow$q_{1,-1}$ and $q_{1,1}$ die (as a consequence of $P_{1,\pm 1}(\pm 1) = 0$). Note that $q_{1,0}$ dies, too. Thus, the first set of nonvanishing $q_{l,m}$ are:
\begin{center}
	\begin{align*}
	q_{2,0} &= qa^2\gamma_{2,0}(P_{2,0}(1) + P_{2,0}(-1)) = qa^2\sqrt{\frac{5}{4\pi}}(1+1) = qa^2\sqrt{5/\pi}
	\end{align*}
\end{center}
	
\end{homeworkSection}
%\begin{homeworkSection}
%\end{homeworkSection}

\end{homeworkProblem}
\setcounter{equation}{0}
%---------------------
% Problem 1.2
\begin{homeworkProblem}

\textbf{Problem Description.}

\begin{homeworkSection}{}
   \begin{align}
      [a, D(\alpha)] &= \alpha D(\alpha) \\
                     &=
      \left[
         a, \exp\left(\frac{-\left| \alpha \right|^{2}}{2}\right)
         e^{\alpha a^{\dagger}}
         e^{-\bar{\alpha}a}
      \right] \\
      &=
       \exp\left(\frac{-\left| \alpha \right|^{2}}{2}\right)
       \left[
         a,
         e^{\alpha a^{\dagger}}
         e^{-\bar{\alpha}a}
      \right]
      \intertext{Identifying $ B = e^{\alpha a^{\dagger}} $ and $ C =
      e^{-\bar{\alpha}a} $}
      &=
       \exp\left(\frac{-\left| \alpha \right|^{2}}{2}\right)
       \left[ a, B C \right] \\
       &= \exp\left(\frac{-\left| \alpha \right|^{2}}{2}\right)
       \left(\left[ a,B \right]C + B[a,C]\right) \\
       &= \exp\left(\frac{-\left| \alpha \right|^{2}}{2}\right)
       \left(\left[ a,e^{\alpha a^{\dagger}} \right] e^{-\bar{\alpha}a} +
       \left[ a, e^{-\bar{\alpha}a}\right] e^{\alpha a^{\dagger}}\right)
       \intertext{The first commutator can be evaluated by expanding the
       exponential into in an infinite series and considering the infinite
    series of commutators. The second commutator evaluates to zero trivially.}
       &= \exp\left(\frac{-\left| \alpha \right|^{2}}{2}\right)
    \left[ a,e^{\alpha a^{\dagger}} \right] e^{-\bar{\alpha}a} \text{\quad where
    $ b = \alpha a^{\dagger} $} \\
\end{align}

\begin{align}
   \left[ a, e^{\alpha a^{\dagger}} \right] &= \left[ a, 1 + b +
   \frac{b^{2}}{2!} + \frac{b^{3}}{3!} + \ldots \right] \\
    &= [a,b] + [a,\frac{b^{2}}{2!}] + [a,\frac{b^{3}}{3!}] + \ldots \\
    &= \sum^{\infty}_{i=0}  \frac{\alpha^{i} [a,(a^{\dagger})^{i}]}{i!} \\
    %TODO: Prove the identity used here
    &= \sum^{\infty}_{i=1}  \frac{\alpha^{i} i (a^{\dagger})^{i-1}}{i!} \\
    &= \sum^{\infty}_{i=1}  \frac{\alpha^{i} (a^{\dagger})^{i-1}}{(i-1)!} \\
    &=  \alpha \sum^{\infty}_{i=1} \frac{\alpha^{i-1} (a^{\dagger})^{i-1}}{(i-1)!} \\
    &= \alpha \sum^{\infty}_{j=0} \frac{\alpha^{j} (a^{\dagger})^{j}}{j!} \\
    &= \alpha \sum^{\infty}_{j=0} \frac{\alpha^{j} (a^{\dagger})^{j}}{j!} \\
    &= \alpha \sum^{\infty}_{j=0} \frac{(\alpha a^{\dagger})^{j}}{j!} \\
    &= \alpha e^{\alpha a^{\dagger}}
\end{align}

Thus,
\[
   [a, D(\alpha)] =
   \alpha
   \exp\left(\frac{-\left| \alpha \right|^{2}}{2}\right)
   e^{\alpha a^{\dagger}}
   e^{-\bar{\alpha}a} =
   \alpha D(\alpha)
\]

Now, $ \left[ a, D(\alpha) \right] \ket{0} = \alpha \left( D(\alpha)
\ket{0} \right) $ from the above. However, $ \left[ a, D(\alpha) \right]
\ket{0} = (a D(\alpha) - D(\alpha) a) \ket{0}$, by definition of the
commutator. Since $ \alpha \ket{0} = 0 $ this expression reduces to $ \left[
a, D(\alpha) \right] = a D(\alpha) \ket{0} $. Thus, $ a D(\alpha) \ket{0} =
\alpha D(\alpha) \ket{0} $ and $ D(\alpha) \ket{0} $ is an eigenstate of $ a $
with eigenvalue $ \alpha $. This is exactly what we mean by the coherent state $
\ket{\alpha} $.
\end{homeworkSection}
\end{homeworkProblem}

\setcounter{equation}{0}
%--------------------- 
\begin{homeworkProblem}
   \subsection{Problem 3a}
   To prove the equality, I will try to cast the left side of the equation
   into the right side.

   \begin{align*}
      \int_{-\infty}^{\infty} x_{1}(t)x_{2}^{*}(t) dt
      &= \int_{-\infty}^{\infty}
      \Bigl(\frac{1}{2\pi}\int_{-\infty}^{\infty} X_{1}(i\omega) \exp(i \omega
      t)d\omega\Bigr)
      \Bigl(\frac{1}{2\pi}\int_{-\infty}^{\infty} X_{2}^{*}(i\omega) \exp(-i \omega
      t)d\omega\Bigr)dt \\
      &= \frac{1}{(2\pi)^2}\int_{-\infty}^{\infty}
      \Bigl(\int_{-\infty}^{\infty} X_{1}(i\omega) \exp(i \omega
      t)d\omega\Bigr)
      \Bigl(\int_{-\infty}^{\infty} X_{2}^{*}(i\omega) \exp(-i \omega
      t)d\omega\Bigr)dt
      \intertext{Now, what I'd like to do is change the order of integration, so
      that I can integrate over $ t $ and leave just the $ \omega $ integral.
      However, the integral in the first term of the product is over some
      particular $ \omega $.  This is generally, different than that for the second
      integral. I'll label the first one $ \omega' $ and the second $ \omega'' $.}
      &= \frac{1}{(2\pi)^2}\int_{-\infty}^{\infty}
      \Bigl(\int_{-\infty}^{\infty} X_{1}(i\omega') \exp(i \omega'
      t)d\omega'\Bigr)
      \Bigl(\int_{-\infty}^{\infty} X_{2}^{*}(i\omega'') \exp(-i \omega''
      t)d\omega''\Bigr)dt \\
      &= \frac{1}{(2\pi)^2}\int_{-\infty}^{\infty}\int_{-\infty}^{\infty}
      \Bigl(\int_{-\infty}^{\infty} X_{1}(i\omega') \exp(i \omega' t)
      X_{2}^{*}(i\omega'') \exp(-i \omega'' t)dt\Bigr) d\omega' d\omega'' \\
      &= \frac{1}{(2\pi)^2}
      \int_{-\infty}^{\infty}X_{2}^{*}(i\omega'')
      \int_{-\infty}^{\infty}X_{1}(i\omega')
      \Bigl(\int_{-\infty}^{\infty} \exp(i \omega' t)
      \exp(-i \omega'' t)dt\Bigr) d\omega' d\omega'' \\
      \intertext{Now, the inner integral looks like $ \int \exp(i \alpha t) dt $,
      which, as derived in the appendix, is $ 2\pi\delta(\alpha) $.}
      &= \frac{1}{2\pi}
      \int_{-\infty}^{\infty}X_{2}^{*}(i\omega'')
      \int_{-\infty}^{\infty}X_{1}(i\omega')
      \delta(\omega'-\omega'') d\omega' d\omega'' \\
      &= \frac{1}{2\pi}
      \int_{-\infty}^{\infty} X_{1}(i\omega') X_{2}^{*}(i\omega') d\omega'
   \end{align*}
   Except for a relabeling of $ \omega' $ to $ \omega $ this is the same
   expression that we were trying to prove.

   \subsection{Problem 3b}
   To prove the equality, I will try to cast the left side of the equation
   into the right side.

   \begin{align*}
      \int_{-\infty}^{\infty}x_T(t+\tau)x_T(t) dt
      &= \frac{1}{\left( 2\pi \right)^2}
      \int_{-\infty}^{\infty}
      \Bigl( \int_{-\infty}^{\infty}X_T(i\omega)\exp(i\omega (t+\tau)) d\omega \Bigr)
      \Bigl( \int_{-\infty}^{\infty}X_T(i\omega)\exp(i\omega t) d\omega \Bigr)
      dt
      \intertext{Doing something similar as before:}
      &= \frac{1}{\left( 2\pi \right)^2}
      \int_{-\infty}^{\infty}\int_{-\infty}^{\infty} \Bigl(
         \int_{-\infty}^{\infty}X_T(i\omega')\exp(i\omega' (t+\tau))
      X_T(i\omega'')\exp(i\omega'' t) dt \Bigr) d\omega' d\omega'' \\
      &= \frac{1}{\left( 2\pi \right)^2}
      \int_{-\infty}^{\infty}\int_{-\infty}^{\infty} \Bigl(
         \int_{-\infty}^{\infty}X_T(i\omega')\exp(i\omega' t)\exp(i\omega'\tau)
      X_T(i\omega'')\exp(i\omega'' t) dt \Bigr) d\omega' d\omega'' \\
      &= \frac{1}{\left( 2\pi \right)^2}
      \int_{-\infty}^{\infty}X_{T}(i\omega'')\int_{-\infty}^{\infty}
      X_{T}(i\omega')\exp(i\omega'\tau) \Bigl(
         \int_{-\infty}^{\infty}\exp(i\omega' t)
      \exp(i\omega'' t) dt \Bigr) d\omega' d\omega'' \\
      \intertext{Now, at this point, the math looks very similar to that in the
   previous part of this problem. However, there was no conjugation done on the
   time-domain signal. Thus, the Dirac delta distribution takes on the following
   form $ 2 \pi \delta(\omega'+\omega'') $. So, now, $ \omega'' \to -\omega' $
   instead of $ \omega' $. Using this identity as before to collapse the integral yields }
      &= \frac{1}{ 2\pi }
      \int_{-\infty}^{\infty}X_{T}(i\omega') X_{T}(-i\omega')\exp(i\omega'\tau)
      d\omega'\\
      \intertext{Now, this is all that can be said about $
      \int_{-\infty}^{\infty}x_T(t+\tau)x_T(t) dt $ in general. However, if it
      is know that $ x(t) \in \mathds{R} $, then $ X(i\omega) $ is conjugate
      symmetric about $ \omega = 0 $. That is, $ X(-i\omega) = X^{*}(i\omega) $.
      In this case, the above expression can be rewritten as:}
      &= \frac{1}{ 2\pi }
      \int_{-\infty}^{\infty}|X_{T}(i\omega')|^2 \exp(i\omega'\tau)
      d\omega'
   \end{align*}
   Except for a relabeling of $ \omega' $ to $ \omega $ this is the same
   expression that we were trying to prove.

\end{homeworkProblem}

\setcounter{equation}{0}
%----------------------
% Problem 4
\begin{homeworkProblem}
    To derive the Wigner function in the momentum basis we re-express the
    characteristic function in terms of of a trace over the momentum basis.

    \begin{align}
        \tilde{W}(u,v) = \Tr(\rho e^{- i u q - i v p}
                       &= \int \bra{p} \rho e^{i u q - i v p} \ket{p} \\
                       &= \int \bra{p} \rho e^{i u v / 2} e^{-i v p} e^{-i u q}
        \ket{p} dp
        \intertext{Now, $ e^{-i u q} \ket{p} = e^{-i u \hat{q}} \frac{1}{\sqrt{2\pi}} \int
            e^{i q p} \ket{q} dq = \frac{1}{\sqrt{2 \pi}} \int e^{i q p} e^{-i u
                q} dq \ket{q} = \frac{1}{\sqrt{2\pi}} \int e^{i q (p-u)}
            \ket{q} dq = \ket{p-u}$.}
            &= \int \bra{p} \rho e^{i u v / 2} e^{-i v p} \ket{p-u} dp \\
            &= \int \bra{p} \rho e^{i u v / 2} e^{-i v (p-u)} \ket{p-u} dp \\
            &= e^{i u v /2} \int e^{-i v (p-u)} \bra{p} \rho \ket{p-u} dp
            \intertext{Now, allow $ p = y+u/2 $.}
            &= e^{i u v /2} \int e^{-i v (y+u/2)} \bra{y+u/2} \rho \ket{y-u/2}
            dy \\
            &= \int e^{-i v y} \braket{y+u/2|\rho|y-u/2} dy
    \end{align}
    So, now we can express the Wigner function as
    \[
        W(q,p) = \frac{1}{(2\pi)^2} \int\int \tilde{W}(u,v) e^{iuq + ivp} du dv
        \enskip.
    \]
    \begin{align}
        W(q,p) &= \frac{1}{(2\pi)^2} \int\int\int e^{-i v y} \braket{y+u/2|\rho|y-u/2} dy e^{iuq + ivp} du dv \\
               &= \frac{1}{(2\pi)^2} \int\int\int \braket{y+u/2|\rho|y-u/2} dy
        e^{iuq + ivp - ivy} du dv \\
        &= \frac{1}{(2\pi)^2} \int\int \braket{y+u/2|\rho|y-u/2}
        e^{iuq}du dy \left( \int e^{-iv(y-p)} dv \right) \\
        &= \frac{1}{(2\pi)^2} \int\int \braket{y+u/2|\rho|y-u/2}
        e^{iuq}du dy \left(2 \pi \delta(y-p)\right) \\
        &= \frac{1}{2 \pi} \int \braket{p+u/2|\rho|p-u/2} e^{iuq} du
        \intertext{Relabelling $ u \to y $:}
        &= \frac{1}{2 \pi} \int \braket{p+y/2|\rho|p-y/2} e^{iyq} dy
    \end{align}
This is the Wigner function expressed in the momentum basis.
\end{homeworkProblem}

\setcounter{equation}{0}
%----------------------
\begin{homeworkProblem}
A simple capacitor is a device formed by two insulated conductors adjacent to each 
other. If equal and opposite charges are placed on the conductors, there will be a 
certain difference of potential between them. The ratio of the magnitude of the 
charge on one conductor to the magnitude of the potential difference is called the 
capacitance (in SI units it is measured in farads). Using Gauss's law, calculate the 
capacitance of 

\begin{homeworkSection}{(a)}

two large, flat, conducting sheets of area A, separated by a small distance d; 

Using a small Gaussian surface on either side of the sheet I find that the electric field looks like $E * 2A = \frac{\sigma a}{\epsilon_0}$, where the factor of 2A on the left side came from the fact that there is flux through both sides of my pill box. Thus, the electric field from a sheet of charge can be approximated in the large A, small d limit as $E = \frac{Q}{2 A \epsilon_0}$ pointing perpendicular to the plates. Thus, the electric field inside, due to both the positive and negative charge is $E_{inside} = \frac{Q}{A \epsilon_0}$. The potential gained by traveling from one plate to another is just $\int \vec{E}\cdot \vec{dl} = \frac{2Q d}{a \epsilon_0}$. This was obtained by taking a path perpendicular to the plates (the direction in which the field faces). The ratio of the charge on one plate to the change in potential is : $\frac{Q}{V} = \frac{A \epsilon_0}{d}$.

\end{homeworkSection}

\begin{homeworkSection}{(b)}

two concentric conducting spheres with radii a, b (b > a); 

Using the results from Problem 1, I know that the potential outside of a sphere of charge is $\frac{k Q}{r}$, where r is my distance from the center of the sphere of charge and Q is the charge on the sphere. Thus, the change in potential between distances a and b with $ b > a $ is $|\Delta V| = \frac{k Q} (\frac{1}{a} - \frac{1}{b})$, where the difference has been taken in order to make the result a positive quantity. So, the capacitance, which is given by $ C = \frac{Q}{V} = \frac{1}{4\pi \epsilon_0}\frac{1}{\frac{1}{a}-\frac{1}{b}} $.

\end{homeworkSection}

\begin{homeworkSection}{(c)}
two concentric conducting cylinders of length L, large compared to their radii a,b (b > a). 

Using a Gaussian cylinder as my test volume and assuming that the field is radially-distributed (reasonable given the conditions of the problem statement). Thus $E(r) 2 \pi r \Delta z = \frac{\sigma 2 \pi a \Delta z}{\epsilon_0}$. Here, $\sigma$ is the charge per unit area on the surface of the cylinder. Thus, $E(r) = \frac{\sigma a}{\epsilon_0 r}$. But, if we consider the charge per unit length $\lambda$ that runs along the length of the cylinder : $\sigma = \frac{\lambda}{2 \pi a}$. Thus $E(r) = \frac{\lambda}{2\pi \epsilon_0 r}$. $\lambda = \frac{Q}{L}$. The potential gained from moving out along the radius will go as the integral of $E(r)$ over r. Thus, the expression obtained will be a natural log. $\Delta V(r) = \frac{Q}{2\pi L \epsilon_0} ln(b/a)$ where b is a distance chosen to be larger than a so as to make the change in potential a positive quantity. Now, $C = \frac{Q}{V} = \frac{Q}{\frac{Q}{2\pi L \epsilon_0} ln(b/a)} = \frac{2 \pi L \epsilon_0}{ln(b/a)}$. 

\end{homeworkSection}

\begin{homeworkSection}{(d)}
What is the inner diameter of the outer conductor in an air-filled coaxial cable 
whose center conductor is a cylindrical wire of diameter 1 mm and whose 
capacitance is:

\begin{itemize}
	\item $ 3*10^{-11} \frac{F}{m}?$
	\item $ 3*10^{-12} \frac{F}{m}?$
\end{itemize}

To be general I will solve for the inner diameter as a function of an arbitrary capacitance. From problem 5 (c) I know that $C/L = \frac{2 \pi \epsilon_0}{ln(b/a)}$. Thus, $ b = a e^{\frac{2 \pi \epsilon_0}{C/L}}$. I am given b. $2\pi \epsilon_0 \approx 5.563*10^-11 F/m$. Thus, substituting the proper values: $b_{3*10^{-11} F/m} \approx 1 mm * e^{1.85} \approx 6.36 mm$ and $b_{3*10^{-12} F=m} \approx 1 mm * e^{18.5} \approx 108 km $

\end{homeworkSection}

\end{homeworkProblem}
\setcounter{equation}{0}
%----------------------
% Problem 1.5
\begin{homeworkProblem}[Problem 6]
   \begin{homeworkSection}{}
      If the state has \SI{10}{\deci\bel} of squeezing then that means the ratio
      of the variance in the $ q $ quadrant of the squeezed state to that of the
      vacuum is
      \[
         10 = -10 \log_{10}\left(\frac{\Delta q^{2}}{\Delta q_{0}^{2}}\right) \enskip.
      \]
      \begin{align}
         -1 &= \log_{10} \left( \frac{\Delta q^{2}}{\Delta {q_0}^{2}}
         \right) \\
         10^{-1} &= \frac{\Delta q^2}{\Delta q_{0}^{2}}
         \intertext{Now, from the notes, $ \braket{q} = 0 $ and $ \braket{q^2} =
            \frac{1}{2}\left( \cosh(2r) - \sinh(2r) \cos(\theta) \right)$ for
         squeezed states (where the squeezing parameter $ \xi = r e^{i \theta}
      $). The vacuum state can be considered a squeezed state with squeezing
   parameter $ \xi = 0 $.}
   &= \frac{\Delta q^2}{\frac{1}{2}} \\
   &= \cosh(2r) - \sinh(2r) \cos(\theta)
   \end{align}
   Now, this expression depends on both $ r $ and $ \theta $. That is, it
   depends on $ \xi $, not just its magnitude or phase. We want to consider how
   many photons are in this state. That is, we want to determine
   \[
    \braket{n} = \braket{0 | S^{\dagger}(\xi) a^{\dagger} a S(\xi) | 0}
    \enskip.
   \]
   In order to maximize the squeezing in the $ q $ quadrant $ \theta = 0 $.
   Assuming that the experimentalists wanted to determine just how much
   squeezing they could get, they would, according to this definition, squeeze
   in the $ q $ quadrant. Thus, we'll assume $ \xi $ is real.
   \[
      10^{-1} = \cosh(2r) - \sinh(2r)
   \]
   The value of $ r $ that satisfies this is given by
   \[
      e^{-2r} = 10^{-1} \enskip.
   \]
   \[
      r = -\frac{1}{2} \ln(10^{-1}) = \frac{1}{2} \ln(10)
   \]
   Using this to calculate the number of photons in the squeezed state:
   \begin{align}
      \braket{n} &= \braket{0 | S^{\dagger}(\xi) a^{\dagger} a S(\xi) | 0} \\
                 &= \braket{0 | S^{\dagger}(\xi) a^{\dagger} S
   S^{\dagger}(\xi) a S | 0}
   \intertext{Using expressions derived in the notes (3.121 and 3.121):}
   &= \braket{0 | \left( a^{\dagger}\cosh(r) - a e^{-i \theta} \sinh(r) \right)
\left( a \cosh(r) - a^{\dagger}e^{i \theta} \sinh(r) \right) | 0}
\intertext{Only the one term that acts with an annihilation operator before the
creation operator survives.}
&= \braket{0 | a a^{\dagger} \sinh^2(r) | 0} \\
&= \sinh^2(r)
   \end{align}
   So, the number of photons in the squeezed state is $ \sinh^2(r) = \sinh^2(.5
   \ln(10) \approx 2.025 $. Haha, no, I would not describe this as bright.
   \end{homeworkSection}
\end{homeworkProblem}

\setcounter{equation}{0}
%%%\newpage
%%%%-------------------------------------------
\begin{thebibliography}{9}

\bibitem{sakurai}
  J.~J.Sakurai,
  \emph{Modern Quantum Mechanics}.
  Addison Wesley, Massachusetts,
  Revised Edition.
%------------------------------------
\bibitem{greiner-relativistic_QM}
  W.~Gereiner,
  \emph{Relativistic Quantum Mechanics}.
  Springer,
  Third Edition,
  2000.
%--------------------------------------
\bibitem{taflove}
  A.~Taflove and S.C.~Hagness
  \emph{Computational Electrodynamics: The Finite Difference Time Domain Method}.
  Artech House,
  Second Edition, 2000.
  
  %----------------------------
  \bibitem{antonio004}
  A.~Soriano et.al,
  \emph{Analysis of the finite difference time domain technique to solve the Schr\"odinger's equation for quantum devices}
  Journal of App. Phys,
  vol 95,N 12,2004.
  %-------------------------------
  \bibitem{shibata}
  T.~Shibata
  \emph{Absorbing boundary conditions for the finite-difference time-domain calculation of the one-dimensional Schr\"odinger's equation}.
  Phys Rev B,
  vol 43, N 8, 1991.
  %--------------------------------------
\bibitem{kosloff}
 R.~ Kosloff and D.~Kosloff
  \emph{Absorbing boundaries for wave propagating problems}.
  Journal of Computational Phys,
  vol 63, 1986.
  %---------------------------
 \bibitem{majd}
 B.~Engquist and A.~Majd
  \emph{Absorbing boundaries for numerical solutions of the waves}.
  Math of Computation,
  vol 31, N 139, 1977. 


\end{thebibliography}


%----------------------------------------
\end{document}
