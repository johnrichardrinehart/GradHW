\begin{homeworkProblem}
In this problem we frequently use two basic properties of the Pauli matrices:
\begin{align}
\{\sigma_i,\sigma_j\}&=\sigma_i\sigma_j+\sigma_j\sigma_i=2\delta_{ij}\label{P1-1}\\
[\sigma_k,\sigma_l]&=\sigma_k\sigma_l-\sigma_l\sigma_k=2i\sum_m\varepsilon_{klm}\sigma_m \label{P1-2}
\end{align}
where $\varepsilon_{klm}$ is Levi-Civita symbol. 
%------------------a---------------------------
\begin{homeworkSection}{(a)}
This problem can be simply solved using the nice properties of the Pauli matrices. We first develop a simple instruction to a more general problem. We claim that a $2\times 2$ matrix $X$ (not necessarily Hermitian, nor unitary) can be written as:
\begin{equation}\label{P1-a}
X=\sum_{i}a_i\sigma_i+a_0=\vp{a}.\B{\sigma}+a_0
\end{equation} 
Actually we are going to show that the Pauli matrices and unity matrix form a complete set of independent basis to expand any arbitrary $2\times 2$ matrices defined in complex field ($\mathbb{C}$). First we prove that they are independent. Assume that they are expanding a null matrix, we will show all expansion coefficents are zero:
\begin{equation}
\sum_i\sigma_i\alpha_i+\alpha_0=0
\end{equation}     
If we calculate the anticommutation of above equation and one of the Pauli matrices and by employing \eqref{P1-1} we arrive at:
\begin{equation}\label{P1-e}
\left\{\sum_i\sigma_i\alpha_i+\alpha_0,\sigma_k\right\}=2\alpha_k+2\alpha_0\sigma_k=0
\end{equation}
Please note that all Pauli matrices are traceless, that is:
\begin{equation}\label{P1-tr}
\Tr(\sigma_i)=\sum_k\sigma_{kk}=0
\end{equation}
So we have:
\begin{equation}
\Tr\left(2\alpha_k+2\alpha_0\sigma_k\right)=4\alpha_k=0
\end{equation}
So $\alpha_k$s are zero and from \eqref{P1-e} $\alpha_0$ is zero as well. So this set of matrices are independent and consequently all $a_i$s in \eqref{P1-a} can be uniqely determined. To evaluate the expansion coefficients we use traceless property of the pauli matrices. First we have to prove the following identity:
\begin{equation}
\Tr\left(\sigma_i\sigma_j\right)=2\delta_{ij}
\end{equation}
Proof:
Using \eqref{P1-1} we have:
\begin{equation}
\Tr\left\{\sigma_i,\sigma_j\right\}=\Tr\left(\sigma_i\sigma_j\right)+\Tr\left(\sigma_j\sigma_i\right)=4\delta_{ij}
\end{equation}
The trace of multiplication of two matrices in independent of the order of multiplication:
\begin{equation}
\Tr\left(AB\right)=\sum_{k}(AB)_{kk}=\sum_{k}\sum_j A_{kj}B_{jk}=\sum_{j}\sum_k B_{jk}A_{kj}=\sum_{j}(BA)_{jj}=\Tr\left(BA\right)
\end{equation}
therefore:
\begin{equation}
\Tr\left\{\sigma_i,\sigma_j\right\}=2\Tr\left(\sigma_i\sigma_j\right)=4\delta_{ij}\quad\Lrw\quad \Tr\left(\sigma_i\sigma_j\right)=2\delta_{ij}
\end{equation}
Using this important identity we can evaluate $a_i$s in \eqref{P1-a}. By mutiplying the both sides of that equation by $\sigma_i$ we get:
\begin{align}
\Tr\left(\sigma_i X\right)&=2 a_i\label{P1-110}\\
\Tr\left(X\right)&=2a_0\label{P1-111}
\end{align}
Now we can return back to our main problem. The density matrix as a $2\time 2$ matrix (like $X$) can be writen as:
\begin{equation}
\rho=\ket{\Psi(t)}\bra{\Psi(t)}=\frac{1}{2}\left(\xi+\sum_i \Psi_i\sigma_i\right)
\end{equation}
Using \eqref{P1-110} and \eqref{P1-111} we have:
\begin{equation}
\xi=\Tr(\rho)=\Tr\left(\ket{\Psi(t)}\bra{\Psi(t)}\right)=\sum_{b}\bracket{b}{\Psi(t)}\bracket{\Psi(t)}{b}=\bracket{\Psi(t)}{\Psi(t)}=1
\end{equation}
In above equation $\ket{b}$ are the eigenstates a given Hermitian operator. For the other coefficients we obtain:
\begin{equation}
\Psi_i=\Tr\left(\sigma_i\rho\right)
\end{equation}
Now we can calculate every coefficient explicitly:
\begin{align}
&\Psi_1=\Tr\left[\left(\ket{0}\bra{1}+\ket{1}\bra{0}\right)\ket{\Psi(t)}\bra{\Psi(t)}\right]=2\Re\left\{\bracket{1}{\Psi(t)}\bracket{0}{\Psi(t)}^*\right\}\\
&\Psi_2=i\Tr\left[\left(\ket{0}\bra{1}-\ket{1}\bra{0}\right)\ket{\Psi(t)}\bra{\Psi(t)}\right]=2\Im\left\{\bracket{1}{\Psi(t)}\bracket{0}{\Psi(t)}^*\right\}\\
&\Psi_3=\Tr\left[\left(\ket{1}\bra{1}-\ket{0}\bra{0}\right)\ket{\Psi(t)}\bra{\Psi(t)}\right]=\bracket{1}{\Psi(t)}^2-\bracket{0}{\Psi(t)}^2
\end{align}
So we arrive at:
\begin{equation}\label{P1-220}
\rho=\frac{1}{2}\left(1+\vp{\Psi}.\B{\sigma}\right)
\end{equation}

\end{homeworkSection}
%------------------b---------------------------
\begin{homeworkSection}{(b)}
The equation of motion for the density matrix is:
\begin{equation}
i\hbar\pd{\rho}{t}=[\M{H},\rho]
\end{equation}
The Hamiltonian is given by:
\begin{equation}\label{P1-H}
\M{H}=\frac{1}{2}\left(1+\vp{H}.\B{\sigma}\right)
\end{equation}
By inserting \eqref{P1-H} and \eqref{P1-220} into the equation of motion we obtain:
\begin{equation}
i\hbar\frac{d}{dt}\left\{\frac{1}{2}\left(1+\vp{\Psi}(t).\B{\sigma}\right)\right\}=\frac{i\hbar}{2}\frac{d\vp{\Psi}}{dt}.\B{\sigma}=\frac{1}{4}\left[1+\vp{H}.\B{\sigma},1+\vp{\Psi}.\B{\sigma}\right]
\end{equation}
therefore:
\begin{equation}
\frac{d\vp{\Psi}}{dt}.\B{\sigma}=\frac{1}{2}\left[\sum_k H_i\sigma_k,\sum_l\Psi_l\sigma_l\right]=\frac{1}{2}\sum_{k,l}H_k\Psi_l\left[\sigma_k,\sigma_l\right]=\frac{i}{i\hbar}\sum_{k,l,m}\varepsilon_{klm}H_k\Psi _l\sigma_m
\end{equation}
From elementary vector analysis the Triple product of three vectors is:
\begin{equation*}
(\vp{A}\times\vp{B}).\vp{C}=\sum_{l,m}A_kB_lC_m\varepsilon_{klm}
\end{equation*} 
hence:
\begin{equation}
\B{\sigma}.\frac{d\vp{\Psi}}{dt}=\frac{1}{\hbar}\B{\sigma}.(\vp{H}\times\vp{\Psi})
\end{equation}
Since the Pauli matrices are linearly independent we immediately conclude:
\begin{equation}\label{P1-EM}
\frac{d\vp{\Psi}}{dt}=\frac{1}{\hbar}(\vp{H}\times\vp{\Psi})
\end{equation}
\end{homeworkSection}
\begin{homeworkSection}{(c)}
To solve \eqref{P1-EM} we simply assume that the coordinate system has been properly rotated so that the\textbf{real} vector $\vp{H}$ has been aligned in the $\A{z}$ direction. This assumption dosn't degrade the solution because the final solution is independent of the coordinate system describing the problem. A linear transformation $\M{T}$  exists that produces the desired rotation:
\begin{equation}
\vp{H}'=\vp{T}^{t}\vp{H}=\abs{\vp{H}}\A{z}\qquad\vp{\Psi}'=\Psi'_1\A{x}+\Psi'_2\A{y}+\Psi'_3\A{z}=\vp{T}^{t}\vp{\Psi} 
\end{equation}
From \eqref{P1-EM} we obtain:
\begin{equation}
\frac{d}{dt}\vp{\Psi}'=\frac{1}{\hbar}\abs{\vp{H}}\A{z}\times\vp{\Psi'}
\end{equation}
Three coupled equations are:
\begin{align}
&\frac{d\Psi'_1}{dt}=-\frac{1}{\hbar}\abs{\vp{H}}\Psi'_2\\
&\frac{d\Psi'_2}{dt}=\frac{1}{\hbar}\abs{\vp{H}}\Psi'_1\\
&\frac{d\Psi'_3}{dt}=0
\end{align}
From the fisrt two equations we get:
\begin{eqnarray}
\frac{d^2\Psi'_1}{dt^2}=-\omega^2\Psi'_1 &\Lrw & \Psi'_1(t)=\Psi'_1(0)\cos\omega t+\frac{1}{\omega}\dot{\Psi}'_1(0)\sin\omega t\\
\frac{d^2\Psi'_2}{dt^2}=-\omega^2\Psi'_2 &\Lrw &\Psi'_2(t)=\Psi'_2(0)\cos\omega t+\frac{1}{\omega}\dot{\Psi}'_2(0)\sin\omega t
\end{eqnarray}
where:
\begin{equation*}
\omega=\frac{\abs{\vp{H}}}{\hbar}
\end{equation*}
using the equation of motions we can write:
\begin{align}
&\Psi'_1(t)=\Psi'_1(0)\cos\omega t-\Psi'_2(0)\sin\omega t\\
&\Psi'_2(t)=\Psi'_2(0)\cos\omega t+\Psi'_1(0)\sin\omega t
\end{align}
$\Psi'_3(t)$ is simply a constant function.
 \end{homeworkSection}
\end{homworkProblem}