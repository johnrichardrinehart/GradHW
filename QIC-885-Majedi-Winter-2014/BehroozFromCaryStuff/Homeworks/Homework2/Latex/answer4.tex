\begin{homeworkProblem}
 We first prove the following identity:
 \begin{equation}\label{P4-comut}
 \comut{A}{BC}=B\comut{A}{C}+\comut{A}{B}C
 \end{equation}
Proof:
If we expand the right hand side of \eqref{P4-comut} we get:
\begin{equation}
B\comut{A}{C}+\comut{A}{B}C&=B(AC-CA)+(AB-BA)C=-BCA+ABC=\comut{A}{BC}
\end{equation}
We have used the associative axiom of multiplication
\begin{homeworkSection}{(a)}
\begin{equation}
\comut{\A{x}^2}{\A{p}}=\A{x}\comut{\A{x}}{\A{p}}+\comut{\A{x}}{\A{p}}\A{x}=2i\hbar\A{x}
\end{equation}
\end{homeworkSection}
%------b---------------
\begin{homeworkSection}{(b)}
We define:
\begin{equation}
\mathcal{A}_n=\comut{\A{x}^n}{\A{p}}
\end{equation}
Using \eqref{P4-comut}, we can draw a recursive experision for $\mathcal{A}_n$  
\begin{equation}
\mathcal{A}_{n}=\comut{\A{x}^n}{\A{p}}=\comut{\A{x}\A{x}^{n-1}}{\A{p}}=\A{x}\comut{\A{x}^{n-1}}{\A{p}}+\comut{\A{x}}{\A{p}}\A{x}^{n-1}=\A{x}\mathcal{A}_{n-1}+i\hbar\A{x}^{n-1}
\end{equation}
So:
\begin{equation}
\mathcal{A}_n=\A{x}\mathcal{A}_{n-1}+i\hbar\A{x}^{n-1}\qquad , \quad \mathcal{A}_1=i\hbar
\end{equation}
We cliam that:
\begin{equation}\label{P4-110}
\mathcal{A}_n=in\hbar\A{x}^{n-1}
\end{equation}
Proof by induction:
\begin{enumerate}
\item
The theory works for $n=1$:
$$\mathcal{A}_1=i\hbar\A{x}^0$$
\item
Assume \eqref{P4-110} works for $N=n$ ,that is, $\mathcal{A}_n=in\hbar\A{x}^{n-1}$ we show it also works for $N=n+1$

\begin{equation*}
\mathcal{A}_{n+1}=\A{x}\mathcal{A}_{n}+i\hbar\A{x}^{n}=
\A{x}in\hbar\A{x}^{n-1}+i\hbar\A{x}^{n}=i(n+1)\hbar\A{x}^{n-1}
\end{equation*}
\end{enumerat}
\end{homeworkSection}
%---------c---------
\begin{homeworkSection}{(c)}
It's assumed that $g(\A{x})$ is a well-defined function of position and we can expand this function as the power series of $\A{x}$ as:
\begin{equation}
g(\A{x})=\sum_{n=0}^{\infty}g_n\A{x}^n
\end{equation}  

As amatter of the fact power series representation is the natural way of  representing any function of an operator. So we obtain:
\begin{equation}
\comut{g(\A{x})}{\A{p}}=\comut{\sum_{n=0}^{\infty}g_n\A{x}^n}{\A{p}}=\sum_{n=0}^{\infty}g_n\comut{\A{x}^n}{\A{p}}
\end{equation} 
From part (c) we have:
\begin{equation}
\comut{g(\A{x})}{\A{p}}=\sum_{n=0}^{\infty}in\hbar g_n\A{x}^{n-1}=i\hbar\left.\frac{dg(\zeta)}{d\zeta}\right|_{\zeta=\A{x}}
\end{equation}
\end{homeworkSection}
%-----d-----------------------
\begin{homeworkSection}{(d)}
\begin{equation}\label{P4-220}
\comut{\A{x}}{H}=\comut{\A{x}}{\frac{\A{p}^2}{2m}+V(\A{x})}=\frac{1}{2m}\comut{\A{x}}{\A{p}^2}+\comut{\A{x}}{V(\A{x})}
\end{equation}
$\A{x}$ and $V(\A{x})$ commute and the second term on the right hand side of \eqref{P4-220} vanishes and just the first term remains. Using \eqref{P4-comut} we obtain:
\begin{equation}
\comut{\A{x}}{\A{p}^2}=\comut{\A{x}}{\A{p}}\A{p}+\A{p}\comut{\A{x}}{\A{p}}=2i\hbar\A{p}
\end{equation}
At the end of the day:
\begin{equation}
\comut{\A{x}}{H}=\frac{i\hbar}{m}\A{p}
\end{equation}

\end{homeworkSection}

%----e---
\begin{homeworkSection}{(e)}
\begin{equation}\label{P4-330}
\comut{\A{p}}{H}=\comut{\A{x}}{\frac{\A{p}^2}{2m}+V(\A{x})}=\frac{1}{2m}\comut{\A{p}}{\A{p}^2}+\comut{\A{p}}{V(\A{x})}
\end{equation}

$\A{p}$ and $\A{p}^2$ commute and just the second term on the right hand side of \eqref{P4-330} remains. Using the result of part (c) we get:
\begin{equation}
\comut{\A{p}}{H}=\comut{\A{p}}{V(\A{x})}=-\comut{V(\A{x})}{\A{p}}=-i\hbar\left.\frac{dV(\zeta)}{d\zeta}\right|_{\lambda=\A{x}}
\end{equation}
\end{homeworkSection}








\end{homeworkProblem}