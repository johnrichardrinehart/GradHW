\begin{homeworkProblem}
\begin{homeworkSectiom}{(a)}
If the the state $\ket{\Psi}$ is a normalized linear combination of two basis as 
$$\ket{\Psi}=\alpha\ket{0}+\beta\ket{1}$$
then the normalization condition is translated as:
\begin{equation}
\bracket{\Psi}{\Psi}=
\begin{pmatrix}
\alpha\\
\alpha
\end{pmatrix}^\dagger
\begin{pmatrix}
\alpha\\
\alpha
\end{pmatrix}
=\abs{\alpha}^2+\abs{\beta}^2=1
\end{equation} 

where $\dagger$ is \textit{transpose conjugate} operator. So $\alpha$ and $\beta$ are related to each other by:
\begin{equation}
\abs{\alpha}^2+\abs{\beta}^2=1
\end{equation}

\end{homeworkSection}
%-----b-----
\begin{homeworkSection}{(b)}
The Hamiltonian of the system is represented by the following matrix:
\begin{equation}
\A{H}=
\begin{pmatrix}
a & b\\
b & a
\end{pmatrix}
\end{equation}
Time independent Schr\"odinger equation in the abstract space is:
\begin{equation}
H\ket{\Psi}=E\ket{\Psi}
\end{equation}
This abstract equation can be represented in $\mathbb{C}^2$ space as the following eigenvalue problem:
\begin{equation}
\begin{pmatrix}
a & b\\
b & a
\end{pmatrix}
\begin{pmatrix}
\alpha\\
\beta
\end{pmatrix}
=E
\begin{pmatrix}
\alpha\\
\beta
\end{pmatrix}
\quad\Lrw\quad
\begin{pmatrix}
a-E & b\\
b & a-E
\end{pmatrix}
\begin{pmatrix}
\alpha\\
\beta
\end{pmatrix}
=0
\end{equation}
Putting the determinant the matrix of the coefficients equal to zero ensures us to have non-trivial solutions:
\begin{equation}
\det
\begin{pmatrix}
a-E & b\\
b & a-E
\end{pmatrix}
=0
\quad\Lrw\quad
(a-E)^2=b^2\quad\Lrw\quad E=a\pm b
\end{equation} 
and the associated eigenvectors are:
\begin{equation}
E_1=a+b\quad\Lrw\quad 
\begin{pmatrix}
a & b\\
b & a
\end{pmatrix}
\begin{pmatrix}
\alpha\\
\beta
\end{pmatrix}
=(a+b)
\begin{pmatrix}
\alpha\\
\beta
\end{pmatrix}
\quad\Lrw\quad
\alpha=\beta
\end{equation}
\begin{equation}
E_2=a-b\quad\Lrw\quad 
\begin{pmatrix}
a & b\\
b & a
\end{pmatrix}
\begin{pmatrix}
\alpha\\
\beta
\end{pmatrix}
=(a-b)
\begin{pmatrix}
\alpha\\
\beta
\end{pmatrix}
\quad\Lrw\quad
\alpha=-\beta
\end{equation}
So two orthonormal eigenstates are:
\begin{align}
&E=a+b\quad\Lrw\quad\ket{\Psi_1}=\frac{1}{\sqrt{2}}\left(\ket{1}+\ket{2}\right)\\
&E=a-b\quad\Lrw\quad\ket{\Psi_2}=\frac{1}{\sqrt{2}}\left(\ket{1}-\ket{2}\right)
\end{align}

\end{homeworkSection}
\begin{homeworkSection}{(c)}
Time-dependent solution to the schr\"odinger equation can be obtained just by applying the time evolution operator on an arbitrary state:
\begin{equation} 
i\hbar\fpds{t}\ket{\Psi}=H\ket{\Psi}\quad\Lrw\quad \ket{\Psi(t)}=\exp\left(\frac{-iHt}{\hbar}\right)\ket{\Psi(t=0)}
\end{equation}
By decomposing $\ket{\Psi(t=0)}=\alpha\ket{1}+\beta\ket{2}$ to the energy eigenstates we get:
\begin{equation}
\ket{\Psi(t)}=\exp\left(\frac{-iHt}{\hbar}\right)\left\{\sum_{n}\ket{\Psi_n}\bra{\Psi_n}\right\}\ket{\Psi(t=0)}=
\sum_{n}\exp\left(\frac{-iE_nt}{\hbar}\right)\ket{\Psi_n}\bracket{\Psi_n}{\Psi(t=0)}
\end{equation}
Hence:
\begin{equation}\label{P1-1}
\ket{\Psi(t)}=\frac{1}{2}(\alpha+\beta)\exp\left(\frac{-i(a+b)t}{\hbar}\right)\left(\ket{1}+\ket{2}\right)
+\frac{1}{2}(\alpha-\beta)\exp\left(\frac{-i(a-b)t}{\hbar}\right)\left(\ket{1}-\ket{2}\right)
\end{equation}

\end{homeworkSection}

%-----C-----
\begin{homeworkSection}{(c)}
If the system starts out at $t=0$ in state $\ket{0}$, the state after time $t$ can be ptredicted by \eqref{P1-1}. By putting $\alpha=1$ and $\beta=0$ we obtain:

\begin{multline}
\ket{\Psi(t)}=\frac{1}{2}\exp\left(\frac{-iat}{\hbar}\right)\left\{\exp\left(\frac{-ibt}{\hbar}\right)\left(\ket{1}+\ket{2}\right)+\exp\left(\frac{+ibt}{\hbar}\right)\left(\ket{1}-\ket{2}\right)\right\}\\
=\exp\left(\frac{-iat}{\hbar}\right)\left\{\cos\left(\frac{bt}{\hbar}\right)\ket{1}-i\sin\left(\frac{bt}{\hbar}\right)\ket{2}\right\}
\end{multline}

\end{homeworkSection}

\end{homeworkProblem}