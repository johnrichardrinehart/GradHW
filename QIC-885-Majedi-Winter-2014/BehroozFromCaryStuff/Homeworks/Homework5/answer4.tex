\begin{homeworkProblem}
\begin{homeworkSection}{(a)}
The expreimental setup has been shown in figure \ref{fig-HBT}. As shown in the figure the original beam splitter accepts two input beams and and produces two other beams which are a linear combination of the inputs.  
%------------------------------
\begin{figure}[!h]
\centering
% Generated with LaTeXDraw 2.0.8
% Wed Apr 10 19:05:51 EDT 2013
% \usepackage[usenames,dvipsnames]{pstricks}
% \usepackage{epsfig}
% \usepackage{pst-grad} % For gradients
% \usepackage{pst-plot} % For axes
\scalebox{0.8} % Change this value to rescale the drawing.
{
\begin{pspicture}(0,-3.454375)(8.822812,3.454375)
\psline[linewidth=0.04](1.7809376,1.610625)(7.2409377,1.610625)(7.2409377,-0.829375)(7.2609377,-0.849375)
\psline[linewidth=0.04cm](7.2409377,-1.789375)(7.2409377,-0.829375)
\psframe[linewidth=0.04,dimen=outer,fillstyle=gradient,gradlines=2000,gradmidpoint=1.0](7.7409377,-0.349375)(6.7609377,-0.889375)
\psline[linewidth=0.04](2.7809374,2.590625)(2.7409375,-2.409375)(4.9609375,-2.409375)(4.9809375,-2.429375)
\rput{90.0}(2.8515625,-7.6503124){\psframe[linewidth=0.04,dimen=outer,fillstyle=gradient,gradlines=2000,gradmidpoint=1.0](5.7409377,-2.129375)(4.7609377,-2.669375)}
\psframe[linewidth=0.04,dimen=outer,fillstyle=solid,fillcolor=yellow](7.9209375,-1.749375)(6.6209373,-3.049375)
\psline[linewidth=0.04cm](5.5009375,-2.409375)(6.6209373,-2.409375)
\rput{47.79893}(2.1081147,-1.5408384){\psframe[linewidth=0.04,linecolor=red,dimen=outer,fillstyle=solid,fillcolor=red](3.6595416,1.697305)(1.9257599,1.5192075)}
\usefont{T1}{ptm}{m}{n}
\rput(5.2523437,-3.159375){$D_1$}
\usefont{T1}{ptm}{m}{n}
\rput(8.292344,-0.639375){$D_2$}
\usefont{T1}{ptm}{m}{n}
\rput(7.3503127,-3.299375){timer-counter}
\usefont{T1}{ptm}{m}{n}
\rput(2.7189062,3.280625){Input #2}
\usefont{T1}{ptm}{m}{n}
\rput(0.884375,1.800625){Input #1}
\usefont{T1}{ptm}{m}{n}
\rput(2.7623436,2.880625){$\vp{E}_2$}
\usefont{T1}{ptm}{m}{n}
\rput(0.84234375,1.400625){$\vp{E}_1$}
\usefont{T1}{ptm}{m}{n}
\rput(2.3423438,-0.359375){$\vp{E}'_1$}
\usefont{T1}{ptm}{m}{n}
\rput(5.5023437,1.940625){$\vp{E}'_2$}
\end{pspicture} 
}

\caption{\small HBT experiment}
\label{fig-HBT}
\end{figure} 
%-----------------------------
According to \cite{fox} the classical EM field in the output braches can be expressed as below:
\begin{align}
&\vp{E}'_1=\frac{\vp{E}_1-\vp{E}_2}{\sqrt{2}}\\
&\vp{E}'_2=\frac{\vp{E}_1+\vp{E}_2}{\sqrt{2}}
\end{align}

These transformtion can be applied to the quantized version of the field. To do this we shoukd just use the same linear combinations for annihilation and creation operators. If we use the same notaions to signify these operators we can write:
\begin{align}
&a'_{1}=\frac{a_1-a_2}{\sqrt{2}}\label{P4-100}\\
&a'_{2}=\frac{a_1+a_2}{\sqrt{2}}\label{P4-101}
\end{align}
the second-order correlation function is then
obtained from:
\begin{equation}
g^{(2)}(\tau)=\frac{\Ex{n'_1(t)n'_2(t+\tau)}}{\Ex{n'_1(t)}\Ex{n'_2(t+\tau)}}
\end{equation}
Using number operator we can write:
\begin{equation}
g^{(2)}(0)=\frac{\Ex{a'^\dagger_1 a'_1a'^\dagger_2 a'_2}}{\Ex{a'^\dagger_1 a'_1}\Ex{a'^\dagger_2 a'_2}}
\end{equation}
We can rewrite this equation in terms of unprimed ladder operators. Actually we have to use this new representation because we know about the number state of the input braches. It's assumed that there is no photon in one of the inputs and the second one is in $\ket{n}$ state. If $\ket{\phi}$
 describes the number state of the inputs we can write:
 \begin{equation}
 \ket{\phi}=\ket{n,0}  
 \end{equation}
 Using \eqref{P4-100} and \eqref{P4-101} we have:
 \begin{equation}
 g^{(2)}(0)=\frac{\Ex{(a_1^\dagger-a_2^\dagger)(a_1-a_2)(a_1^\dagger+a_2^\dagger)(a_1+a_2)}}{\Ex{(a_1^\dagger-a_2^\dagger)(a_1-a_2)}\Ex{(a_1^\dagger+a_2^\dagger)(a_1+a_2)}}
 \end{equation} 
  Using the fact that
\begin{equation*}
a_2\ket{\phi}=0
\end{equation*}
we can extremely simplify all calculation:
 \begin{align}
 &\bra{\phi}{(a_1^\dagger-a_2^\dagger)(a_1-a_2)}\ket{\phi}=\bra{\phi}a_1^\dagger a_1\ket{\phi}=\bra{\phi}\A{N}_1\ket{\phi}\\
 &\bra{\phi}{(a_1^\dagger+a_2^\dagger)(a_1+a_2)}\ket{\phi}=\bra{\phi}a_1^\dagger a_1\ket{\phi}=\bra{n}\A{N}_1\ket{n}
 \end{align}
 For the third term we have to work more:
 \begin{equation}\label{P4-300}
\bra{\phi}{(a_1^\dagger-a_2^\dagger)(a_1-a_2)(a_1^\dagger+a_2^\dagger)(a_1+a_2)}}\ket{\phi}
= \bra{\phi}a_1^\dagger(a_1-a_2)(a_1^\dagger+a_2^\dagger)a_1\ket{\phi} 
 \end{equation}
 If we expand the left hand side of \eqref{P4-300} we will encounter to four terms. Using commuation relations we can simplify every term on the right hand side of above equation:
 \begin{align}
 \bra{\phi} a_1^\dagger a_1 a_1^\dagger a_1\ket{\phi} &=\bra{\phi}\A{N}_1^2\ket{\phi}\\
  \bra{\phi}a_1^\dagger a_1 a_2^\dagger a_1\ket{\phi} &=\bra{\phi}a_2^\dagger a_1^\dagger a_1 a_1\ket{\phi}=0\\
 \bra{\phi}a_1^\dagger a_2 a_1^\dagger a_1\ket{\phi}&=\bra{\phi}a_1^\dagger  a_1^\dagger a_1 a_2\ket{\phi}=0\\
 \bra{\phi}a_1^\dagger a_2 a_2^\dagger a_1\ket{\phi}&=\bra{\phi}a_1^\dagger a_2^\dagger a_2 a_1\ket{\phi}+
 \bra{\phi}a_1^\dagger [a_2, a_2^\dagger] a_1\ket{\phi}
 =-\bra{\phi}\A{N}_1\ket{\phi}
  \end{align}  
In all expression we have use the fact that $\A{a}_1$ and $\A{a}_2$ commute. At the end of the day we arrive at the following equation:
 \begin{equation}
 g^{(2)}(0)=\frac{\Ex{\A{N}_1^2-\A{N}_1}}{\Ex{\A{N_1}}^2}
 \end{equation}
 \end{homeworkSection}
%------b--------------
\begin{homeworkSection}{(b)}
If the input is in photon number state $\ket{n}$ we have:
\begin{equation}
 g^{(2)}(0)=\frac{\Ex{\A{N}_1^2-\A{N}_1}}{\Ex{\A{N_1}}^2}
=\frac{n^2-n}{n^2}=\frac{n-1}{n}
\end{equation}

\end{homeworkSection}
\end{homeworkProblem}